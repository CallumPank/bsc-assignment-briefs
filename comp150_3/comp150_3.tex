\documentclass{../fal_assignment}
\graphicspath{ {../} }

\usepackage{enumitem}
\setlist{nosep}
\usepackage[T1]{fontenc} % http://tex.stackexchange.com/a/17858
\usepackage{url}
\usepackage{todonotes}

\title{Evaluation --- Teamwork Review}
\author{Dr Michael Scott}

\begin{document}

\maketitle
\begin{marginquote}
    ``It is the long history of humankind: those who learned to collaborate and improvise most effectively have prevailed''
    
    --- Charles Darwin
\end{marginquote}
\marginpicture{MakeyMakey.jpg}{
    The \emph{MaKey~MaKey} allows a multitude of materials to be used to create videogame controllers.
}
\section*{Introduction}

In this assignment, you critically reflect on your teamwork and team dynamic across the group project, using this as a lens to review you own team-working skills. You will also develop a plan of action to overcome any challenges; thereby, improving the quality of your output in future group projects.

Such reflection and planning is an extremely important part of learning games development. Research shows that deliberate practice is very effective at nurturing expertise in software engineering. Everyone properly adopting this technique eventually succeeds, despite the challenging nature of the subject.

This assignment is formed of several parts:

\begin{enumerate}[label=(\alph*)]
    \item \textbf{Write} a weekly team blog, authored \textbf{collaboratively}, that must:
    	\begin{enumerate}[label=\roman*.]
    		\item \textbf{describe} the progress of the team project;
    		\item \textbf{assess} any obstacles or challenges that the team has encountered;
    		\item and \textbf{outline} specific actions to take to overcome them.
	\end{enumerate}
    \item \textbf{Write} a draft 500-word report that must:
    	\begin{enumerate}[label=\roman*.]
    		\item \textbf{identify two} key teamwork skills that you consider weaknesses;
    		\item \textbf{assess} your application of \textbf{each} of these skills, \textbf{describing how} they affected your work \textbf{and suggesting why} they arose;
    		\item and then \textbf{identify how} to improve \textbf{each} of these skills, with reference to SMART actions.
	\end{enumerate}
    \item \textbf{Write} a final 500-word report that must:
    	\begin{enumerate}[label=\roman*.]
    		\item \textbf{revise} any issues raised by your tutor or your peers.
	\end{enumerate}
\end{enumerate}

\subsection*{Assignment Setup}

This assignment is a \textbf{reflective writing task} and so regular reflection is expected. Use the existing \texttt{comp150-desktop-game} repository for your weekly blog. Also, fork the GitHub repository at the following URL:

\indent \url{https://github.com/Falmouth-Games-Academy/comp150-evaluation}

Use the existing directory structure and, as required, extend this structure with sub-directories. The \texttt{readme.md} file is not needed. 

Modify the \texttt{.gitignore} to the defaults for \textbf{TeX}. Please, also ensure that you add editor-specific files and folders to \texttt{.gitignore}. 

\subsection*{Part A}

Part A consists of a \textbf{multiple formative submissions}. This work is \textbf{collaborative} and will be assessed on a \textbf{threshold} basis. The following criteria are used to determine a pass or fail:

\begin{enumerate}[label=(\alph*)]
	\item Progress is described with adequate detail;
	\item Challenges are clearly explained;
	\item Both team-related and technical challenges are addressed;
	\item All reports afford appropriate anonymity (see Additional Guidance section);
	\item Reflections are constructive rather than blaming and/or punitive;
	\item Proposed actions to overcome challenges are SMART.
\end{enumerate}

To complete Part A, one member of the team should write a brief report each term-time week in the \texttt{readme.md} document hosted in the existing \texttt{comp150-desktop-game} repository. Separate each week with a heading. It is recommended that the team discuss the content of the report together and that the scribe be changed each week.  Ensure that the report is pushed to GitHub prior to each scheduled sprint review session. Then, attend the scheduled review session.

You will receive immediate \textbf{informal feedback}.

\subsection*{Part B}

Part B is a \textbf{single formative submission}. This work is \textbf{individual} and will be assessed on a \textbf{threshold} basis. The following criteria are used to determine a pass or fail:

\begin{enumerate}[label=(\alph*)]
	\item Submission is timely;
	\item Enough work is available to conduct a meaningful review;
	\item A broadly appropriate review of a peer's work is submitted.
\end{enumerate}

To complete Part B, prepare a draft version of the report. Use the marking rubric to inform the structure of the document. Ensure that the TeX source and compiled \texttt{*.pdf} are pushed to GitHub and a pull request is made prior to the scheduled review session. Then, attend the scheduled review session.

You will receive \textbf{peer feedback} within 3 working days after the code review session.

\subsection*{Part C}

Part C is a \textbf{single summative submission}. This work is \textbf{individual} and will be assessed on a \textbf{criterion-referenced} basis. The following criteria are used to allocate marks:

\begin{enumerate}[label=(\alph*)]
	\item Appropriateness and Specificity of Selection of Key Teamwork Skills;
	\item Adequacy of Self-Appraisal in Relation to Key Teamwork Skills;
	\item Depth of Reflection on Key Teamwork Skills;
	\item Appropriateness of Plan for the Future;
	\item Quality of Academic Writing.
\end{enumerate}

You will receive \textbf{formal feedback} three weeks after the final deadline.

\section*{Additional Guidance}

This is a reflective writing task, and as such you are expected to write in a reflective style. Make use of first-person plural and follow a similar style. However, take care to assume an as objective position as possible. There must be sufficient analysis and reference to evidence in order to support your claims. More so than would be expected in a personal reflective report.

Teamwork analysis is \textbf{not} the place where you indulge in witch-hunts and blame-games. While it is inevitable that incidents occur, this is not an excuse to blame other students for shortcomings in a project. This is not only poor form, but is indicative of the report's author themselves having poor teamworking skills. Incidents should be taken as an opportunity for reflection. Why wasn't a contingency in place? Why had nobody pair programmed a critical user story? Why wasn't there a backup? Where was the collective responsibility? These are opportunities for growth and personal development.

A common concern is lack of engagement by other members of the team. Such concerns can be discussed, but must be \textbf{anonymous}. No other student should be identified in either the report or the blog. Moreover, concerns must be sufficiently analysed. For example, why was the design of the game not re-scoped if it became clear that one of the specialists on the team was unlikely to deliver key components upon which a proposed design relied?

The continuous cycle of reflection and planning is the cornerstone of deliberate practice. It is extremely important to help you learn. Avoid treating this activity as an afterthought. Further to this, avoid delegating the blog a single member of the team. It should represent the output from a team discussion that occurs at the end of each week. 

Furthermore, do not place too much emphasis on the end-product (i.e., the game itself). A short description alongside some screenshots and pseudocode is more than sufficient. It is the reflection and analysis that is important. Namely, \textit{how} and \textit{why} your team's working practice is influencing the quality of the game in more abstract terms. As such, presenting your work in terms of `lessons learned' can be appropriate. This will likely lead to higher quality submissions in the future and such lessons could form some of the components of your final report. Critically, however, lessons learned form a strong foundation for improving your teamwork.

When choosing which key skills to address in your report, be specific.
Avoid choosing general skills (e.g.\ communication, time management) that are clearly important for all teams.
Instead focus on which specific weaknesses are a priority for \textbf{you}, as a team and as individuals.

As with previous evaluations, the most common mistake when planning future actions to take is being too general. It is, therefore, important to consider SMART goals: specific; measurable; achievable; relevant; and time-bound. Teamwork is critically important in the games industry, whether this be through an out-sourcing, remote collaboration, or indeed pair-programming. So, do ensure that you engage with this process of reflection and planning to improve your teamworking skills.

\section*{FAQ}

\begin{itemize}
	\item 	\textbf{What is the deadline for this assignment?} \\ 
    		Falmouth University policy states that deadlines must only be specified on LearningSpace. Please examine the assignment area where you located this document.
    			    		
	\item 	\textbf{When do I need to get my weekly reports signed-off?} \\ 
    		At the end of each Sprint. The blog must be up to date in order to achieve the threshold marks for that Sprint.
    		
	\item 	\textbf{What should I do to seek help?} \\ 
    		You can email your tutor for informal clarifications. For informal feedback, make a pull request on GitHub. 
    		
    	\item 	\textbf{Is this a mistake?} \\ 	
    		If you have discovered an issue with the brief itself, the source files are available at: \\
    		\url{https://github.com/Falmouth-Games-Academy/bsc-assignment-briefs}.\\
    		 Please make a pull request and comment accordingly.
\end{itemize}

\section*{Additional Resources}

\begin{itemize}
    \item Belbin, R.M. (2012) Team roles at work. Routledge.
    \item Baker, D.P. and Salas, E. (1992) Principles for Measuring Teamwork Skills. Human Factors, 34(4), pp.469-475.
    \item Williams, L. and Kessler, R. (2002) Pair programming Illuminated. Addison-Wesley.
\end{itemize}

\begin{markingrubric}
%
    \firstcriterion{Satisfactory Preparation of Weekly Reports}{10\%}
        \gradespan{5}{\fail At least one weekly blog post has not been submitted, is incomplete, or is unsatisfactory.}
        \grade 		All weekly blog posts have been signed-off by your tutor by the deadline.
%
    \criterion{Appropriateness, Specificity, and Relevance of Selection of Key Teamwork Skills}{10\%}
        \grade\fail 	Fewer than two appropriate key skills are mentioned.
        \par 		Little or no focus on teamwork.
        \grade 		At least two appropriate key skills are mentioned.
        \par 		Some focus on teamwork.
        \grade 		At least two appropriate key skills are mentioned.
        \par 		Much focus on teamwork.        
        \grade 		At least two appropriate key skills are mentioned.
        \par 		Significant focus on teamwork.    
        \par 		At least one key skill is specific and career relevant.
        \grade 		At least two appropriate key skills are mentioned.
        \par 		Exemplary focus on teamwork.   
        \par 		At least two key skills are specific and career relevant.
        \grade 		At least two appropriate key skills are mentioned.
        \par 		Exemplary focus on teamwork.   
        \par 		At least two key skills are specific, career relevant, and a priority.
%
    \criterion{Adequacy of Self-Criticism in Relation to Key Teamwork Skills}{20\%}
        \grade\fail 	No self-criticism is made.
        \grade 		Little self-criticism is made.
        \grade 		Some self-criticism is made.
        \grade 		Much self-criticism is made.
        \grade 		A significant level of self-criticism is made.
            \par 		Some of the self-criticism is accurate and pertinent.
        \grade 		An exceptional level of self-criticism is made.
            \par 		Much of the self-criticism is accurate and pertinent.
%
    \criterion{Depth of the Reflection on the Application of Key Teamwork Skills}{20\%}
        \grade\fail 	No reflection is evident.
        \grade 		Little reflection is evident.
        \grade 		Some reflection is evident.
        \grade 		Much reflection is evident.
        \par 		Some depth of insight is demonstrated.
        \grade 		Significant reflection is evident.
        \par 		Much depth of insight is demonstrated.
        \grade 		Exemplary reflection is evident.
        \par 		Significant depth of insight is demonstrated.
%
    \criterion{Appropriateness of Plan for Future Development}{25\%}
        \grade\fail 	No generally appropriate plans are proposed.
        \grade 		At least one generally appropriate plan is proposed.
        \grade 		At least two generally appropriate plans are proposed.
        \grade 		At least two specific and achievable plans are proposed. 
        \par 		At least one of the plans is also relevant.
        \grade 		At least two specific, relevant, and achievable plans are proposed. 
        \par 		At least one of the plans is also measurable and time-bound.
        \grade 		At least two specific, measurable, achievable, relevant, and time-bound plans are proposed. 
%
    \criterion{Appropriateness of Reflective Writing Style}{5\%}
        \grade\fail 	Demonstrates no evidence of ability in reflective writing.
        \grade 		Demonstrates evidence of little ability in reflective writing.
        \grade 		Demonstrates evidence of some ability in reflective writing.  
        \grade 		Demonstrates evidence of partial mastery of reflective writing.
        \grade 		Demonstrates evidence of mastery in reflective writing.
        \grade 		Demonstrates significant evidence of mastery in reflective writing.
%
    \criterion{Appropriateness of Spelling and Grammar}{5\%}
        \grade\fail 	Substantial spelling and/or grammar errors.
        \grade 		Many spelling and/or grammar errors.
        \grade 		Some spelling and/or grammar errors.  
        \grade 		Few spelling and/or grammar errors.
        \grade 		Nearly no spelling and/or grammar errors.
        \grade 		No spelling and/or grammar errors.
%
    \criterion{Appropriateness of Essay Structure}{5\%}
        \grade\fail 	There is no structure, or the structure is unclear.
        \grade 		There is little structure.
        \grade 		There is some structure.
        \par 		A few sentences and paragraphs are well constructed.
        \grade 		There is much structure.
        \par 		Some sentences and paragraphs are well constructed.
        \par 		There is a clear introduction and conclusion.
        \grade 		There is much structure, highlighting the key skills.
        \par 		Most sentences and paragraphs are well constructed.
        \par 		There is a clear and well-constructed introduction and conclusion.
        \grade 		There is much structure, highlighting the key skills.
        \par 		All sentences and paragraphs are well constructed.
        \par 		There is a clear and well-constructed introduction and conclusion.
\end{markingrubric}

\end{document}