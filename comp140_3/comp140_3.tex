\documentclass{../fal_assignment}
\graphicspath{ {../} }

\usepackage{enumitem}
\setlist{nosep} % Make enumerate / itemize lists more closely spaced
\usepackage[T1]{fontenc} % http://tex.stackexchange.com/a/17858
\usepackage{url}
\usepackage{todonotes}

\title{Evaluation --- Usability Analysis}
\author{Dr Michael Scott}

\begin{document}

\maketitle
\begin{marginquote}
    ``My mantras: focus and simplicity. Simple can be harder than complex. You have to work hard to get your thinking clean to make it simple. But it's worth it in the end because once you get there, you can move mountains.''
    
    --- Steve Jobs
\end{marginquote}
\marginpicture{MakeyMakey.jpg}{
    The \emph{MaKey~MaKey} allows a multitude of materials to be used to create videogame controllers.
}
\section*{Introduction}

In this assignment, you will evaluate a game interface. Specifically, the novel input device that you have designed and the game it is designed for. You will apply a heuristic analysis technique and report the results.

Design is an art and a science. Humans are mimetic creatures that share much physiology and build shared understandings through culture. This provides the foundation for usability, removing obstacles to players' ability to play. Hence, an ability to assess usability is important for commercial success.

This assignment is formed of several parts:

\begin{enumerate}[label=(\alph*)]
    \item \textbf{Write} a set of heuristics which:
    	\begin{enumerate}[label=\roman*.]
    		\item \textbf{identify} properties of good design;
	\end{enumerate}
    \item \textbf{Conduct} a heuristic analysis, with the support of a group, in which you:
    	\begin{enumerate}[label=\roman*.]
    		\item \textbf{apply} the heuristics to \textbf{evaluate} the game controller;
      		\item \textbf{record} feedback from at least three peer-evaluators;
    		\item and then \textbf{review} the feedback.
	\end{enumerate}
    \item \textbf{Write} a draft 500-word report that must:
    	\begin{enumerate}[label=\roman*.]
    		\item \textbf{briefly describe} the game controller \textbf{and} the evaluation;    
    		\item \textbf{recommend TWO} specific improvements;	
    		\item and then \textbf{justify} those improvements.
	\end{enumerate}
    \item \textbf{Write} a final 500-word report that must:
    	\begin{enumerate}[label=\roman*.]
    	          \item \textbf{revise} any concerns raised by peers and tutors. 
	\end{enumerate}
\end{enumerate}

\subsection*{Assignment Setup}

This assignment is an \textbf{evaluative writing task}. Fork the GitHub repository at the following URL:

\indent \url{https://github.com/Falmouth-Games-Academy/comp140-evaluation}

Use the existing directory structure and, as required, extend this structure with sub-directories. Ensure that you maintain the \texttt{readme.md} file.

Modify the \texttt{.gitignore} to the defaults for \textbf{TeX}. Please, also ensure that you add editor-specific files and folders to \texttt{.gitignore}. 

\subsection*{Part A}

Part A consists of a \textbf{single formative submission}. This work is \textbf{individual} and will be assessed on a \textbf{threshold} basis. The following criteria are used to determine a pass or fail:

\begin{enumerate}[label=(\alph*)]
	\item An appropriate set of heuristics has been defined;
\end{enumerate}

To complete Part A, list your heuristics in the \texttt{readme.md} document.  Show these to your tutor.  If acceptable, this will be signed-off. 

You will receive immediate \textbf{informal feedback}.

\subsection*{Part B}

Part B consists of a \textbf{single formative submission}. This work is \textbf{collaborative} and will be assessed on a \textbf{threshold} basis. The following criteria are used to determine a pass or fail:

\begin{enumerate}[label=(\alph*)]
	\item Guidance to peer-evaluators is sufficient;
	\item Photographic evidence illustrates that the analysis was conducted.
	\item A broadly appropriate review of a peer's work is submitted.
\end{enumerate}

To complete Part B, prepare your prototype for evaluation. Ensure that the heuristics and sufficient guidance are added to the \texttt{readme.md} document. Also ensure that these have been pushed to GitHub prior to the scheduled session. Then, attend the scheduled session. During the session, take photographs and add these photographs to the \texttt{readme.md} document.

You will receive \textbf{peer feedback} within 3 working days after the session.

\subsection*{Part C}

Part C is a \textbf{single formative submission}. This work is \textbf{individual} and will be assessed on a \textbf{threshold} basis. The following criteria are used to determine a pass or fail:

\begin{enumerate}[label=(\alph*)]
	\item Submission is timely;
	\item Enough work is available to conduct a meaningful review;
	\item A broadly appropriate review of a peer's work is submitted.
\end{enumerate}

To complete Part C, prepare a draft version of the report. Ensure that the TeX source has been pushed to GitHub and a pull request is made prior to the scheduled session. Then, attend the scheduled session.

You will receive \textbf{peer feedback} within 3 working days after the session.

\subsection*{Part D}

Part D is a \textbf{single summative submission}. This work is \textbf{individual} and will be assessed on a \textbf{criterion-referenced} basis. The following criteria are used to allocate marks:

\begin{enumerate}[label=(\alph*)]
	\item Appropriateness of Heuristics;
	\item Adequacy of Procedure;
	\item Depth of Analysis;
	\item Appropriateness of Design Recommendations;
	\item Adequacy of Use of Figures and Tables;
	\item Quality of Academic Writing;
\end{enumerate}

To complete Part D, revise the report based on the feedback you have received. Then, upload the report to the LearningSpace. Please note, the LearningSpace will only accept a single \texttt{*.pdf} file.

You will receive \textbf{formal feedback} three weeks after the final deadline.

\section*{Additional Guidance}

Normally, reviewers are experts. They will have used heuristics in the past and will be familiar with them. Here, however, reviewers will be your peers. Like you, they are novices. For this reason, it is important that you provide clear guidance to support them. Write instructions clearly in the \texttt{readme.md} document. Be aware that any interference you make during the evaluation itself may bias the results. Further to this, reviewers may have differing opinions. Critically analyse the reviews before making any recommendations.

It is worthwhile helping out your peers at the same time they are helping you out. Working in your existing groups and rotating responsibilities can make the procedure very efficient. This should be done in the review session itself, but if there is insufficient time then you may conduct additional reviews outside of class. If, however, there are any concerns about conducting the review, then please ask before leaving the review session.

A key challenge with this assignment is identifying appropriate heuristics and adapting them to the field of games. Many heuristics do not require any change at all to be applicable to games, while others may require removal or a minor tweak in terminology. If you propose additional heuristics, ensure that there is sufficient rationale to do so. That is, cite relevant research.

The analysis itself is quite formulaic. Refer to and follow the guidelines published by Jakob Nielsen. Although his book chapter and website are not a simple checklist, they are quite comprehensive. One suggestion is to create a diagram to illustrate the procedure. This will not only aid in its correct application during the review session, but will help you to reduce the number of words in your report. 

So long as you follow the procedure accurately, it should be straightforward to identify several design flaws. It only takes four reviewers to catch most well-known flaws. You only need to address one or two of these flaws through the recommended design changes, as the same flaw could be targeted through several changes. Do not attempt to fix everything in such a short report.

\section*{FAQ}

\begin{itemize}
	\item 	\textbf{What is the deadline for this assignment?} \\ 
    		Falmouth University policy states that deadlines must only be specified on LearningSpace. Please examine the assignment area where you located this document.
    		
    	\item 	\textbf{Am I supposed to come up with my own heuristics from scratch?} \\ 
    		No. Use and adapt existing research. Re-word existing heuristics to make them more relevant to games.	
    		
	\item 	\textbf{What should I do to seek help?} \\ 
    		You can email your tutor for informal clarifications. For informal feedback, make a pull request on GitHub. 
    		
    	\item 	\textbf{Is this a mistake?} \\ 	
    		If you have discovered an issue with the brief itself, the source files are available at: \\
    		\url{https://github.com/Falmouth-Games-Academy/bsc-assignment-briefs}.\\
    		 Please make a pull request and comment accordingly.
\end{itemize}

\section*{Additional Resources}

\begin{itemize}
    \item Norman, D. (2013) The Design of Everyday Things. Revised Edition. MIT Press.
    \item Przybylski, A.K., Deci, E.L., Rigby, C.S., and Ryan, R. M. (2014) Competence-Impeding Electronic Games and Players' Aggressive Feelings, Thoughts, and Behaviors. Journal of Personality and Social Psychology, 106(3), pp. 441-457.
    \item Nielsen, J. (2002) Heuristic Evaluation. Usability Inspection Methods, 17(1), pp. 25-62. 
    \item Peters, T. (2008) Design: Tom Peters Essentials. Gabal Verlag GBMH.
    \item \url{https://www.nngroup.com/topic/heuristic-evaluation/}
    \item \url{http://gameaccessibilityguidelines.com/}
\end{itemize}

\begin{markingrubric}
%
    \firstcriterion{Satisfactory Completion of Heuristic Analysis Procedure}{10\%}
        \gradespan{5}{\fail Analysis has not been signed-off by your tutor.}
        \grade 		Analysis has been signed-off by your tutor.
%
    \criterion{Satisfactory Preparation of Draft for Peer-Review}{5\%}
        \gradespan{5}{\fail Either no draft is available for review or the review provided is unsatisfactory.}
        \grade 		Participated in peer review and provided an appropriate review.
%
    \criterion{Appropriateness of Heuristics}{20\%}
        \grade\fail 	No heuristics are listed.
        \grade 		A set of heuristics is listed.
        \par 		It is not clear how the heuristics have been derived, or there is a lack of academic rigor.
        \grade 		A set of heuristics is listed.
        \par 		The heuristics have been copied verbatim from a scholarly source, or has been adapted poorly.
        \grade 		An appropriate set of heuristics is listed.
        \par 		The heuristics have been adapted from one or more scholarly sources.
        \grade 		An appropriate set of heuristics is listed.
        \par 		The heuristics have been adapted from one or more scholarly sources.
        \par 		Appropriately justified adaptation for use in a games domain has been attempted.        
        \grade 		An appropriate set of heuristics is listed.
        \par 		The heuristics have been adapted from one or more scholarly sources.
        \par 		Rigorously justified adaptation for use in a games domain has been attempted.    
%
    \criterion{Adequacy of Procedure}{5\%}
        \grade\fail 	No description of procedure.
        \grade 		A very weak procedure is evidenced.
        \grade 		A weak weak procedure.
        \grade 		A sufficient procedure is evidenced.
        \grade 		An appropriate procedure is evidenced.
        \grade 		An well-conducted procedure is evidenced.
%
    \criterion{Depth of Analysis}{20\%}
        \grade\fail 	No analysis.
        \grade 		Little analysis.
        \grade 		Some analysis.
        \grade 		Much analysis.
        \par 		Some depth of insight is demonstrated.
        \grade 		Significant analysis.
        \par 		Much depth of insight is demonstrated.
        \grade 		Exemplary analysis.
        \par 		Significant depth of insight is demonstrated.
%
    \criterion{Appropriateness of Design Recommendations}{20\%}
        \grade\fail 	No design changes are recommended.
        \grade 		At least one generally appropriate design change is proposed.
        \grade 		At least one specific and achievable design changes are proposed. 
        \grade 		At least two generally appropriate design changes are proposed.
        \par  		At least one specific and achievable design changes are proposed. 
        \grade 		At least two specific and achievable design changes are proposed. 
        \par  		At least one of the proposed changes is a significant improvement and well-justified.
        \grade 		At least two specific and achievable design changes are proposed. 
        \par  		The proposed changes are both significant improvements and well-justified.
%
    \criterion{Adequacy of Use of Figures and Tables}{5\%}
        \grade\fail 	No tables or figures are used.
        \grade 		Very poor tables or figures are present.
        \grade 		Poor tables or figures are present. 
        \grade 		Sufficiently designed tables and figures are present. 
        \grade 		Appropriately designed tables and figures are present.
        \grade 		Well designed tables and figures are present. 
%
    \criterion{Appropriateness of Academic Writing}{5\%}
        \grade\fail 	No evidence for partial-mastery of academic writing.
        \par 		The reference section is missing.
        \grade 		Some evidence for partial-mastery of academic writing.
        \par 		The reference section is incomplete and/or malformed.
        \grade 		Much evidence for partial-mastery of academic writing.
        \par 		The reference section is complete and well-formed in either ACM or IEEE format.
        \par 		Most in-text citations and quotations are correct.
        \grade 		Some evidence for mastery of academic writing.
        \par 		The reference section is complete and well-formed in either ACM or IEEE format.
        \par 		All in-text citations and quotations are correct.
        \grade 		Much evidence for mastery of academic writing.
        \par 		The reference section is complete and well-formed in either ACM or IEEE format.
        \par 		All in-text citations and quotations are correct.
        \grade 		Significant evidence for mastery of academic writing.
        \par 		The reference section is complete and well-formed in either ACM or IEEE format.
        \par 		All in-text citations and quotations are correct.
%
    \criterion{Appropriateness of Spelling and Grammar}{5\%}
        \grade\fail 	Substantial spelling and/or grammar errors.
        \grade 		Many spelling and/or grammar errors.
        \grade 		Some spelling and/or grammar errors.  
        \grade 		Few spelling and/or grammar errors.
        \grade 		Almost no spelling and/or grammar errors.
        \grade 		No spelling or grammar errors.
%
    \criterion{Appropriateness of Report Structure}{5\%}
        \grade\fail 	There is no structure, or the structure is unclear.
        \grade 		There is little structure.
        \grade 		There is some structure.
        \par 		A few sentences and paragraphs are well constructed.
        \grade 		There is much structure.
        \par 		Some sentences and paragraphs are well constructed.
        \par 		There is a clear introduction and conclusion.
        \grade 		There is much structure, highlighting the recommendations.
        \par 		Most sentences and paragraphs are well constructed.
        \par 		There is a clear and well-constructed introduction and conclusion.
        \grade 		There is much structure, highlighting the recommendations.
        \par 		All sentences and paragraphs are well constructed.
        \par 		There is a clear and well-constructed introduction and conclusion.
\end{markingrubric}

\end{document}