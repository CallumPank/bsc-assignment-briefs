\documentclass{../fal_assignment}
\graphicspath{ {../} }

\usepackage{enumitem}
\usepackage[T1]{fontenc} % http://tex.stackexchange.com/a/17858
\usepackage{url}
\usepackage{todonotes}

\title{Agile Essay}
\author{Dr Michael Scott}

\begin{document}

\maketitle

\section*{Introduction}

\begin{marginquote}
  ``Individuals and Interactions over Processes and Tools''
  
  ``Working Software over Comprehensive Documentation''
  
  ``Customer Collaboration over Contract Negotiation''
  
  ``Responding to Change over Following a Plan''
    
    --- Agile Manifesto
\end{marginquote}

In this assignment, you will conduct research on the agile development methodology in order to bring an academic perspective into your working practice. Specifically, you will address the following three questions: (1) What is the agile philosophy? (2) To what extent is the agile philosophy suited to the games industry? (3) What form should the application of agile principles take in the games industry? You will then present your research findings as a brief 15-minute presentation and a 1000-word academic essay.

This work has several aims. Firstly, to develop your written communication skills. Employers in the games industry demand a high standard. Bugs resulting from miscommunicated requirements are a large avoidable cost. Secondly, to develop your research skills. Most critically, transitioning from using just textbooks and websites (i.e., at school) to using more rigorous library resources such as academic peer-reviewed papers (i.e., at university). Thirdly, to develop your knowledge of working practices in the games industry.

This assignment is formed of four parts: A, B, C, and D. You will be arranged into small peer support groups. You must:

\begin{enumerate}[label=(\alph*)]
    \item Prepare a short 100-200 word proposal \textbf{and} a reference list which must:
    	\begin{enumerate}[label=\roman*.]
    		\item \textbf{identify} a specific research question which you intend to address;
    		\item \textbf{describe} the way in which you intend to address the question;
    		\item and then \textbf{list} at least \textbf{SIX} appropriate academic references to use to support your research.
	\end{enumerate}
    \item Prepare a 10-15 minute group presentation that must:
    	\begin{enumerate}[label=\roman*.]
    		\item \textbf{describe}, with the support of your peers, the key findings of your research;
    		\item and \textbf{discuss how}, with the support of your peers, these findings apply to your group's working practice.
	\end{enumerate}
    \item Prepare a draft 1000-word essay which will:
    	\begin{enumerate}[label=\roman*.]
    		\item \textbf{address EACH} of the \textbf{THREE} research questons.
	\end{enumerate}
    \item Prepare a final 1000-word essay which will:
    	\begin{enumerate}[label=\roman*.]
    		\item \textbf{revise} any issues raised by your tutor and/or your peers.
	\end{enumerate}
\end{enumerate}

Part A consists of a \textbf{single formative individual} submission. This is not grade-bearing; however, submission is mandatory. Failure to meet with your tutor will result in a grade capped at 40\% (D-). Appropriate feedback will be provided orally in a tutorial to ensure you take an appropriate direction.

 %

\marginpicture{MakeyMakey.jpg}{
    The \emph{MaKey~MaKey} allows a multitude of materials to be used to create videogame controllers.
}

Part B is a \textbf{single collaborative formative} submission and will be assessed on a \textbf{threshold} basis. The threshold is set at 5\%. This means that 5\% of the total marks available for the coursework overall are awarded on a pass or fail basis. In other words, satisfactory submissions will be awarded 5\%. However, unsatisfactory submissions will receive 0\%.

The following criteria are used to determine a pass or fail in Part B:

\begin{enumerate}[label=(\alph*)]
	\item All three questions are adequetely addressed by the team;
	\item There is evidence of some academic rigor;
	\item The discussion demonstrates insight into the relationship between theory and practice.
\end{enumerate}

Part C is a \textbf{single individual formative} submission and will be assessed on a \textbf{threshold} basis. The threshold is set at 5\%. This means that 5\% of the total marks available for the coursework overall are awarded on a pass or fail basis. In other words, satisfactory submissions will be awarded 5\%. However, unsatisfactory submissions will receive 0\%.

A pass is determined through attendence to the peer-review session with a draft and submission of a satisfactory peer-review.

Part D is a \textbf{single individual summative} submission and will be assessed on a \textbf{criterion-referenced} basis. This submission is expected to take students from the threshold of 15\% (F) up to the maximum of 100\% (A*). This means that 85\% of the total marks available for the coursework overall will be awarded.

The following criteria are used to allocate marks:

\begin{enumerate}[label=(\alph*)]
	\item Appropriateness of Referenced Articles;
	\item Relevance to and Focus on the Research Questions;
	\item Depth of Insight into the Agile Philosophy;
	\item Specificity, Verifiability, and Accuracy of Claims;
	\item Adequacy of Analysis of Research Articles;	
	\item Adequacy of Discussion on Transfer to the Games Industry;
	\item Quality of Academic Writing;
\end{enumerate}

\begin{marginquote}
    ``Learn from yesterday, live for today, hope for tomorrow. The important thing is not to stop questioning.''
    
    --- Albert Einstein
\end{marginquote}
\marginpicture{guitarhero}{
    Rhythm games such as \emph{Guitar Hero} and \emph{Rock Band} are excellent examples of games
    which make use of unique input devices to enhance gameplay.
}

\section*{Submission Instructions}

\subsection*{Part A}

Part A must be completed as a formative submission on GitHub. Fork the GitHub project at the following URL:

\indent \url{https://github.com/Falmouth-Games-Academy/comp150-agile}

Write your proposal in the \texttt{readme.md} file. Provide a reference list using the \texttt{*.bib} file in the repository. You will need to show this to your tutor in a personal tutorial session prior to Week 3, at which point Part A will be signed-off. 

You will receive feedback immediately in the session.

\subsection*{Part B}

Part B must be completed as a single PDF document, prepared in either LaTeX (i.e., using Beamer) or some other presentation software. A single PDF document must be submitted to the LearningSpace by the final submission deadline shown on LearningSpace. Please note that the LearningSpace will only accept a single PDF document. 

You will receive feedback immediately in the session.

\subsection*{Part C}

Part C must be completed as a single PDF document, prepared in LaTeX. This should be uploaded to GitHub, with an accompanying pull request being made, prior to the relevant review session as scheduled in the course schedule. 

You will receive feedback shortly after the session.

\subsection*{Part D}

Part D must be completed as a single PDF document, prepared in LaTeX. The LaTeX source files should be hosted on GitHub in the \texttt{comp150-agile} repository. A single PDF document must be submitted to the LearningSpace by the final submission deadline shown on LearningSpace. Please note that the LearningSpace will only accept a single PDF document.

You will receive formal feedback three weeks after the submission deadline shown on LearningSpace.

\begin{marginquote}
    ``Luck is not a factor. Hope is not a strategy. Fear is not an option.''
    
    --- James Cameron
    
    \marginquoterule

        ``We keep moving forward, opening new doors, and doing new things, because we're curious and curiosity keeps leading us down new paths.''
    
    --- Walt Disney
\end{marginquote}

\marginpicture{fishing}{
    The \emph{Dreamcast Fishing Controller}, released as a peripheral for the game \emph{Sega Bass Fishing}.
    Even peripherals which appeal to only a small audience can enjoy moderate commercial success.
}

\section*{Additional Guidance}

Use your experience from the previous essays. Identify weaknesses and feed-\textit{forward}. University is an opportunity for improvement and an effective way to do this is to compare past and current performance.

Areas where students tend to lose marks are: depth of insight; analytical skill; and evaluative skill. Depth of insight implies rigorous research, addressing one key challenge in much detail, rather than several challenges with weaker research and/or in less detail. Adequete analysis implies going beyond mere descrption, perhaps through: performing calculatons, comparing sources, or even deploying reasoning to generate new insights. Adequete evaluation implies making appropriate reference to evidence and ensuring that evidence is of appropriate quality. Further to this, sound and valid arguments are constructed, criticising the claims made by other authors.

Stick to the research questions. You have but 1000-words! Depth over breadth. Quality over quantitiy. Stick to the point and write concisely. Your ability to recall facts is \textbf{not} under assessment! Your ability to construct an argument through critical analysis and making it relevant to practice \textbf{is}.

\section*{Additional Resources}

\begin{itemize}
    \item Keith, C. (2010) Agile Game Development with Scrum. Pearson Education.
    \item http://agilemanifesto.org/
\end{itemize}

\begin{markingrubric}
%
    \firstcriterion{Satisfactory Preparation of Presentations and Peer-Reviews}{10\%}
        \gradespan{2}{\fail At least one weekly blog post has not been submitted, is incomplete, or is unsatisfactory.}
        \gradespan{2}{Either Part B or Part C are passed.}
        \gradespan{2}{Both Part B and Part C are passed.}
%
    \criterion{Appropriateness of Referenced Articles}{10\%}
        \grade\fail 	No relevant article is referenced.
        \grade 		At least three relevant sources are referenced.
        \grade 		At least six relevant sources have been referenced.
        \par		Where appropriate, some sources report scholarly research.
        \grade 		At least eight relevant sources have been referenced.
        \par		Where appropriate, most articles report scholarly research.
        \grade 		At least ten relevant sources have been referenced.
        \par		Where appropriate, all sources report scholarly research.
        \par		Some appropriate seminal and highly reputed sources have been referenced.      
        \grade 		At least ten relevant sources have been referenced.
        \par		Where appropriate, all articles report scholarly research.
        \par		Many appropriate seminal and highly reputed sources have been referenced.   
%
    \criterion{Relevance to and Focus on the Research Questions}{5\%}
        \grade\fail 	No focus on the research questions.
        \grade 		Little focus on the research questions.
        \grade 		Some focus on the research questions.
        \grade 		Much focus on the research questions.
            \par 		Research questions are explicitly defined.
        \grade 		Significant focus on the research questions.
            \par 		Research questions are explicitly or otherwise clearly defined.
            \par 		The conclusion explicitly refers back to the research question.
        \grade 		Extensive focus on the research questions.
            \par 		Research questions are explicitly or otherwise clearly defined.
            \par 		The conclusion explicitly or otherwise clearly refers back to the research question.
%
    \criterion{Depth of Insight into the Agile Philosophy}{20\%}
        \grade\fail 	No depth of insight into the agile philosophy.
        \grade 		Little depth of insight into the agile philosophy.
        \grade 		Some depth of insight into the agile philosophy.
        \par 		Reference to the agile manifesto or related work.
        \grade 		Much depth of insight into the agile philosophy.
        \par 		Articulation of the agile manifesto and related work.
        \grade 		Significant depth of insight into the agile philosophy.
        \par 		Exploration of the agile manifesto with reference to appropriate related work.
        \grade 		Exemplary depth of insight into the agile philosophy.
        \par 		Critical insight into the agile manifesto with support from related work.
%
    \criterion{Specificity, Verifiability, and Accuracy of Claims}{5\%}
        \grade\fail 	No citations to evidence to claims.
        \par 		Substantial errors and/or misinterpretations.
        \grade 		Few claims have a clear source of evidence.
        \par 		Significant errors and/or misinterpretations.
        \grade 		Some claims have a clear source of evidence.
        \par 		Many errors and/or misinterpretations.
        \grade 		Many claims have a clear source of evidence.
        \par 		Some errors and/or misinterpretations.
        \grade 		Most claims have a clear source of evidence.
        \par 		Few errors and/or misinterpretations.
        \grade 		All claims have a clear source of evidence.
        \par 		Almost no errors and/or misinterpretations.
%
    \criterion{Adequacy of Analysis of Research Articles}{20\%}
        \grade\fail 	No analysis has been presented.
        \grade 		Little analysis has been presented.
        \grade 		Some analysis has been presented. 
        \grade 		Much analysis has been presented.
        \grade 		Significant analysis has been presented.
        \grade 		Exemplary analysis has been presented.
%
    \criterion{Adequacy of Discussion on Transfer to the Games Industry}{15\%}
        \grade\fail 	No transfer to the games industry.
        \grade 		Little transfer to the games industry.
        \grade 		Some transfer to the games industry. 
        \par 		Appropriate references to the games industry and/or game development practice. 
        \grade 		Much transfer to the games industry.
        \par 		Appropriate argument suggesting effective game development practice. 
        \grade 		Significant transfer to the games industry.
        \par 		Relevant criticism of game development practices, demonstrating insight into pitfalls and arguing for possible solutions. 
        \grade 		Exemplary transfer to the games industry.
        \par 		Relevant criticism of game development practices, demonstrating insight into key pitfalls and effectively defending appropriate solutions with evidence. 
%
    \criterion{Appropriateness of Academic Writing}{5\%}
        \grade\fail 	No evidence for partial-mastery of academic writing.
        \par 		The reference section is missing.
        \grade 		Some evidence for partial-mastery of academic writing.
        \par 		The reference section is incomplete and/or malformed.
        \grade 		Much evidence for partial-mastery of academic writing.
        \par 		The reference section is complete and well-formed in either ACM or IEEE format.
        \par 		Most in-text citations and quotations are correct.
        \grade 		Some evidence for mastery of academic writing.
        \par 		The reference section is complete and well-formed in either ACM or IEEE format.
        \par 		All in-text citations and quotations are correct.
        \grade 		Much evidence for mastery of academic writing.
        \par 		The reference section is complete and well-formed in either ACM or IEEE format.
        \par 		All in-text citations and quotations are correct.
        \grade 		Significant evidence for mastery of academic writing.
        \par 		The reference section is complete and well-formed in either ACM or IEEE format.
        \par 		All in-text citations and quotations are correct.
%
    \criterion{Appropriateness of Spelling and Grammar}{5\%}
        \grade\fail 	Substantial spelling and/or grammar errors.
        \grade 		Many spelling and/or grammar errors.
        \grade 		Some spelling and/or grammar errors.  
        \grade 		Few spelling and/or grammar errors.
        \grade 		Almost no spelling and/or grammar errors.
        \grade 		No spelling or grammar errors.
%
    \criterion{Appropriateness of Essay Structure}{5\%}
        \grade\fail 	There is no structure, or the structure is unclear.
        \grade 		There is little structure.
        \grade 		There is some structure.
        \par 		A few sentences and paragraphs are well constructed.
        \grade 		There is much structure.
        \par 		Some sentences and paragraphs are well constructed.
        \par 		There is a clear introduction and conclusion.
        \grade 		There is much structure, highlighting the argument.
        \par 		Most sentences and paragraphs are well constructed.
        \par 		There is a clear and well-constructed introduction and conclusion.
        \grade 		There is much structure, highlighting the argument.
        \par 		All sentences and paragraphs are well constructed.
        \par 		There is a clear and well-constructed introduction and conclusion.
\end{markingrubric}

\end{document}