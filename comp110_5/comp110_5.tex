\documentclass{../fal_assignment}
\graphicspath{ {../} }

\usepackage{enumitem}
\usepackage[T1]{fontenc} % http://tex.stackexchange.com/a/17858
\usepackage{url}
\usepackage{todonotes}

\title{Semester Two Reflective Report}
\author{Dr Michael Scott}

\begin{document}

\maketitle
%\begin{marginquote}
%    ``Students come into programming classes with a broad range of backgrounds ---
%    some have experience in several programming languages, others have never programmed before in their life.
%    
%    Being able to engage with the community and support each other is important.
%    Upload your code to GitHub and receive feedback from experienced peers.
%    Review your peers' work yourself and really consider what `quality' actually means
%    and what `good' source code looks like.
%    Debate, argue, and question others about it ---
%    an open and sustained discourse is an excellent way to learn ---
%    for both beginners and adepts!''
%\end{marginquote}
\begin{marginquote}
    ``Remember, learning to program can take a surprising amount of time \& effort --- students may get there at different rates, but all students who put in the time \& effort get there eventually. Making good use of [reflection and deliberate practice] are an essential part of this process.''
    
    --- Professor Quintin Cutts
\end{marginquote}
\marginpicture{MakeyMakey.jpg}{
    The \emph{MaKey~MaKey} allows a multitude of materials to be used to create videogame controllers.
}
\section*{Introduction}

In this assignment, you critically reflect on your progress across the Semester. This involves reviewing key weaknesses that influence the quality of your work. From this, you develop plans of continuing professional development.

Such reflection and planning is an extremely important part of learning games development. Research shows that deliberate practice is very effective at nurturing expertise in software engineering. Everyone properly adopting this technique eventually succeeds, despite the challenging nature of the subject.

This assignment is formed of several parts:

\begin{enumerate}[label=(\alph*)]
    \item \textbf{Write} a series of brief weekly reports that will:
    	\begin{enumerate}[label=\roman*.]
    		\item \textbf{describe} your progress;
    		\item \textbf{assess} any problems or issues that you have encountered;
    		\item and then \textbf{outline} some specific actions to take to overcome these problems.
	\end{enumerate}
    \item \textbf{Write} a draft 500-word report that must:
    	\begin{enumerate}[label=\roman*.]
    		\item \textbf{identify three} key skills that you consider weaknesses;
    		\item \textbf{assess} your application of \textbf{each} of these skills, \textit{describing how} they affected the quality of your submissions \textbf{and suggesting why} they became challenges;
    		\item and then \textbf{identify how} to improve \textbf{each} of these skills, with reference to SMART actions.
	\end{enumerate}
    \item \textbf{Write} a final 500-word report that must:
    	\begin{enumerate}[label=\roman*.]
    		\item \textbf{revise} any issues raised by your tutor or your peers.
	\end{enumerate}
\end{enumerate}

\subsection*{Assignment Setup}

This assignment is a \textbf{reflective writing task} and so regular reflection is expected. Fork the GitHub repository at the following URL:

\indent \url{https://github.com/Falmouth-Games-Academy/comp110-evaluation}

Use the existing directory structure and, as required, extend this structure with sub-directories. Ensure that you maintain the \texttt{readme.md} file. 

Modify the \texttt{.gitignore} to the defaults for \textbf{TeX}. Please, also ensure that you add editor-specific files and folders to \texttt{.gitignore}. 

\subsection*{Part A}

Part A consists of a \textbf{multiple formative submissions}. This work is \textbf{individual} and will be assessed on a \textbf{threshold} basis. The following criteria are used to determine a pass or fail:

\begin{enumerate}[label=(\alph*)]
	\item Progress has been described with adequate detail;
	\item Problems and issues have been clearly explained and assessed;
	\item There is evidence of reflection;
	\item At least one appropriate SMART action has been outlined;
\end{enumerate}

To complete Part A, write a report each term-time week in the \texttt{readme.md} document. Separate each week with a heading. Show these to your tutor at three week intervals.  If acceptable, these will be signed-off. 

You will receive immediate \textbf{informal feedback}.

\subsection*{Part B}

Part B is a \textbf{single formative submission}. This work is \textbf{individual} and will be assessed on a \textbf{threshold} basis. The following criteria are used to determine a pass or fail:

\begin{enumerate}[label=(\alph*)]
	\item Submission is timely;
	\item Enough work is available to conduct a meaningful review;
	\item A broadly appropriate review of a peer's work is submitted.
\end{enumerate}

To complete Part B, prepare a draft version of the reflective report. Use the marking rubric to inform the structure of the document. Ensure that the TeX source and compiled \texttt{*.pdf} are pushed to GitHub and a pull request is made prior to the scheduled code review session. Then, attend the scheduled code review session.

You will receive \textbf{peer feedback} within 3 working days after the code review session.

\subsection*{Part C}

Part C is a \textbf{single summative submission}. This work is \textbf{individual} and will be assessed on a \textbf{criterion-referenced} basis. The following criteria are used to determine a pass or fail:

\begin{enumerate}[label=(\alph*)]
	\item Appropriateness and Specificity of Selection of Key Skills;
	\item Adequacy of Self-Appraisal in Relation to Key Skills;
	\item Depth of Reflection on Key Skills;
	\item Appropriateness of Plan for the Future;
	\item Quality of Academic Writing;
\end{enumerate}

To complete Part C, revise the reflective report based on the feedback you have received. Then, upload the reflective report to the LearningSpace. Please note, the LearningSpace will only accept a single \texttt{*.pdf} file.

You will receive \textbf{formal feedback} three weeks after the final deadline.

\section*{Additional Guidance}

Reflection is taking time to examine thoughts, feelings, beliefs, values, attitudes and assumptions in the context of a specific topic, situation, problem, issue, or process. Part of reflection is relating these varied understandings to your experience. Then analysing how and why something arose. Building upon this, you can predict future performance and consequently propose ways to improve future performance.

A common mistake made by beginners to reflective writing is to merely describe the context and/or the experience. Avoid this. Description is not particularly important. It is the analysis and evaluation of an experience which is important. This is because it will reveal insights about yourself and your actions, which will help you to focus on the most relevant weaknesses.

Programming is a skill that requires a significant investment in time and energy to develop. Of course, regular practice is important. However, quality of practice is more important than quantity of practice.

Deliberate practice is a continuous cycle of reflection, planning and practice. This means that reflection is a forethought rather than an afterthought. Avoid leaving reflection to the last minute. Focus on the ongoing development process early, and indeed, on the process itself. The quality of the end product may indicate the existence of challenges, but it is only through deep reflection on working practice that it becomes clear why those challenges arose and how to avoid them.

Effective deliberate practice is: conscious and intentional; designed with current skill in mind, forcing discomfort while avoiding frustration; provides relevant measures to track progress; and follows a repeatable structure.

A common mistake when planning such practice is being too general. Consider SMART actions: specific; measurable; achievable; relevant; and time-bound. Also note, problem solving and designing are particularly important programming skills and approaches to developing skills in these areas can be relevant.

\section*{FAQ}

\begin{itemize}
	\item 	\textbf{What is the deadline for this assignment?} \\ 
    		Falmouth University policy states that deadlines must only be specified on LearningSpace. Please examine the assignment area where you located this document.
    			    		
	\item 	\textbf{When do I need to get my weekly reports signed-off?} \\ 
    		Meet with your tutor for an individual tutorial at least once every three weeks during term-time. Show them your weekly reports on GitHub. This can be done by making a pull-request prior to the meeting. 
    		
	\item 	\textbf{What should I do to seek help?} \\ 
    		You can email your tutor for informal clarifications. For informal feedback, make a pull request on GitHub. 
    		
    	\item 	\textbf{Is this a mistake?} \\ 	
    		If you have discovered an issue with the brief itself, the source files are available at: \\
    		\url{https://github.com/Falmouth-Games-Academy/bsc-assignment-briefs}.\\
    		 Please make a pull request and comment accordingly.
\end{itemize}

\section*{Additional Resources}

\begin{itemize}
    \item Ericsson, K.A., Krampe, R.T., and Tesch-Romer, C. (1993)The Role of Deliberate Practice in the Acquisition of Expert Performance. Psychological Review, 100(3), 363-406.
    \item Bolton, G.E.J. (2014) Reflective Practice: Writing and Professional Development. SAGE Publications: London.
\end{itemize}

\begin{markingrubric}
%
    \firstcriterion{Satisfactory Preparation of Weekly Reports}{10\%}
        \gradespan{5}{\fail At least one weekly report has not been submitted, is incomplete, or is unsatisfactory.}
        \grade 		All weekly reports have been signed-off by your tutor by the deadline.
%
    \criterion{Appropriateness, Specificity, and Relevance of Selection of Key Skills}{10\%}
        \grade\fail 	Less than two appropriate key skills are mentioned.
        \grade 		At least two appropriate key skills are mentioned.
        \grade 		At least three appropriate key skills are mentioned.
        \grade 		At least three appropriate key skills are mentioned.
        \par 		At least two of the key skills are both specific and relevant.
        \grade 		At least three appropriate key skills are mentioned.
        \par 		At least three of the key skills are both specific and relevant.
        \grade 		At least three appropriate key skills are mentioned.
        \par 		At least two of the key skills are both specific and a priority.
%
    \criterion{Adequacy of Self-Criticism in Relation to Key Skills}{20\%}
        \grade\fail 	No self-criticism is made.
        \grade 		Little self-criticism is made.
        \grade 		Some self-criticism is made.
        \grade 		Much self-criticism is made.
        \grade 		A significant level of self-criticism is made.
            \par 		Some of the self-criticism is accurate and pertinent.
        \grade 		An exception level of self-criticism is made.
            \par 		Much of the self-criticism is accurate and pertinent.
%
    \criterion{Depth of the Reflection on the Application of Skills}{20\%}
        \grade\fail 	No reflection is evident.
        \grade 		Little reflection is evident.
        \grade 		Some reflection is evident.
        \grade 		Much reflection is evident.
        \par 		Some depth of insight is demonstrated.
        \grade 		Significant reflection is evident.
        \par 		Much depth of insight is demonstrated.
        \grade 		Exemplary reflection is evident.
        \par 		Significant depth of insight is demonstrated.
%
    \criterion{Appropriateness of Plan for Future Development}{25\%}
        \grade\fail 	No appropriate plans are proposed.
        \grade 		At least one generally appropriate plan is proposed.
        \grade 		At least two specific and achievable plans are proposed. 
        \grade 		At least three specific and achievable plans are proposed. 
        \par 		At least two of the plans are also relevant.
        \grade 		At least three specific, relevant, and achievable plans are proposed. 
        \par 		At least two of the plans are also measurable and time-bound.
        \grade 		At least three specific, measurable, achievable, relevant, and time-bound plans are proposed. 
%
    \criterion{Appropriateness of Reflective Writing Style}{5\%}
        \grade\fail 	Demonstrates no evidence of ability in reflective writing.
        \grade 		Demonstrates evidence of little ability in reflective writing.
        \grade 		Demonstrates evidence of some ability in reflective writing.  
        \grade 		Demonstrates evidence of partial mastery of reflective writing.
        \grade 		Demonstrates evidence of mastery in reflective writing.
        \grade 		Demonstrates significant evidence of mastery in reflective writing.
%
    \criterion{Appropriateness of Spelling and Grammar}{5\%}
        \grade\fail 	Substantial spelling and/or grammar errors.
        \grade 		Many spelling and/or grammar errors.
        \grade 		Some spelling and/or grammar errors.  
        \grade 		Few spelling and/or grammar errors.
        \grade 		Nearly no spelling and/or grammar errors.
        \grade 		No spelling and/or grammar errors.
%
    \criterion{Appropriateness of Essay Structure}{5\%}
        \grade\fail 	There is no structure, or the structure is unclear.
        \grade 		There is little structure.
        \grade 		There is some structure.
        \par 		A few sentences and paragraphs are well constructed.
        \grade 		There is much structure.
        \par 		Some sentences and paragraphs are well constructed.
        \par 		There is a clear introduction and conclusion.
        \grade 		There is much structure, highlighting the key skills.
        \par 		Most sentences and paragraphs are well constructed.
        \par 		There is a clear and well-constructed introduction and conclusion.
        \grade 		There is much structure, highlighting the key skills.
        \par 		All sentences and paragraphs are well constructed.
        \par 		There is a clear and well-constructed introduction and conclusion.
\end{markingrubric}

\end{document}