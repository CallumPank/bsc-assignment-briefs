\documentclass{../fal_assignment}
\graphicspath{ {../} }

\usepackage{enumitem}
\setlist{nosep} % Make enumerate / itemize lists more closely spaced
\usepackage[T1]{fontenc} % http://tex.stackexchange.com/a/17858
\usepackage{url}
\usepackage{todonotes}

\title{API Hacking}
\author{Dr Edward Powley}

\begin{document}

\maketitle
%\begin{marginquote}
%    ``Students come into programming classes with a broad range of backgrounds ---
%    some have experience in several programming languages, others have never programmed before in their life.
%    
%    Being able to engage with the community and support each other is important.
%    Upload your code to GitHub and receive feedback from experienced peers.
%    Review your peers' work yourself and really consider what `quality' actually means
%    and what `good' source code looks like.
%    Debate, argue, and question others about it ---
%    an open and sustained discourse is an excellent way to learn ---
%    for both beginners and adepts!''
%\end{marginquote}
\marginpicture{MakeyMakey.jpg}{
    The \emph{MaKey~MaKey} allows a multitude of materials to be used to create videogame controllers.
}
\section*{Introduction}

In this assignment, you are required to design and implement a component for a game.
The component must be based on one or more \textbf{existing third-party APIs}:
for example web services, hardware device SDKs, or open-source libraries.
Your component must be integrated into one of the games being developed by students on the BA Digital Games course,
either by using Unity's native plugin interface to invoke code written in C++,
or by implementing the plugin directly in C\#.

This assignment is formed of \textbf{three} parts.
There is a \textbf{single summative submission} via LearningSpace.
Begin by forking the GitHub project at the following URL:
\begin{center}
\url{https://github.com/Falmouth-Games-Academy/comp140-api-hacking}
\end{center}

\subsection*{A. \emph{Propose} a game component}

On GitHub, edit the \texttt{readme.md} file to contain a description of your proposed game component.
Your proposal should:
\begin{enumerate}
    \item \textbf{Identify} the game into which your component will be integrated;
    \item \textbf{Describe} the component that will be created;
    \item \textbf{Show} the key user stories, in the form of a Trello task board;
    \item \textbf{Justify} your choice of technologies (implementation language, APIs, etc.).
\end{enumerate}
Your proposal will be assessed on:
\begin{enumerate}
    \item \textbf{Scope}:
        is the proposed component non-trivial, but feasible given the duration of the project?
    \item \textbf{Appropriateness of component}:
        will the proposed component enhance the selected game?
    \item \textbf{Appropriateness of technologies}:
        is the choice of implementation language and APIs sensible and well-justified?
\end{enumerate}

\textbf{Formative submission:} Discuss your proposal and task board with tutors in class.

\subsection*{B. \emph{Implement} the component}

You will build your prototype component over \textbf{three sprints}.
At the beginning of each sprint, populate the sprint backlog on your Trello board.
You should aim to have a `potentially shippable' component at the end of each sprint;
that is, a component which does not have any major flaws or half-finished features that prevent it from being tested.
For the first two sprints, the component should be working either within the game or in a separate testing environment;
for the final sprint, the component should be fully integrated into the game.

Your implementation will be assessed on the criteria listed in the marking rubric.

\textbf{Formative submission:} Participate in the sprint review sessions in class.
Check your code into your forked GitHub repository regularly,
and make a pull request whenever you require assistance or feedback.

\subsection*{C. \emph{Demonstrate} your component}

Bring an executable of the game, with your component integrated, to the demo session in class.
Be prepared to discuss it with tutors and peers.

This part is not assessed, but you will receive feedback which will be useful to you on future projects.

\textbf{Formative submission:} Participate in the demo session.

\subsection*{Summative submission (electronic)}

Create a zip file containing the following:

\begin{itemize}
\item A markdown file named \texttt{readme.md} containing your \textbf{proposal} from part~A,
    and any other \textbf{documentation} you feel is appropriate
    (e.g.\ special instructions for testing the component).
\item Screenshots of your \textbf{Trello task board}.
    As a minimum, include one from the beginning and one from the end of each sprint,
    and screenshots documenting any checklists or other information within your cards.
\item All \textbf{source code} created for part~B.
    You do \textbf{not} need to provide the entire source code for the game.
    Any code which was \textbf{not written by you} must be \textbf{clearly identified} as such,
    in the \texttt{readme.md} file and/or as comments in the code files themselves.
\item A \textbf{Windows executable} version of the game with your component integrated.
    Build the project in Unity, and include both the \texttt{*.exe} file and the \texttt{*\_Data} directory.
\end{itemize}

The recommended way to produce this zip file is to check all of the above into your GitHub repository
throughout the course of the project,
and then use the ``Download Zip'' function on the GitHub website.
Any material you do not wish to upload to GitHub (e.g.\ the executable)
must be added to the zip file manually before uploading.

\textbf{Material not included in the zip file will not be marked},
even if it is available online, and even if tutors have seen it in class.
Please ensure that all material you want the markers to consider is included in your submission.

Upload your zip file to the appropriate submission queue on LearningSpace.
Note that LearningSpace accepts only a single zip file per submission,
with a maximum file size of 1GB.

\begin{marginquote}
    ``Alive with new thinking, buzzing with opportunity, connected with the best in the business,
    Falmouth University is the perfect place to start shaping your creative career.
    Thousands of people from around the globe come to us every year,
    graduating to become the brightest stars in art, design, media, performance and writing industries.

    ``[...]
    Falmouth has forged its position as one of the most highly regarded creative arts institutions across the globe.''
    
    --- Falmouth University website
\end{marginquote}
\marginpicture{guitarhero}{
    Rhythm games such as \emph{Guitar Hero} and \emph{Rock Band} are excellent examples of games
    which make use of unique input devices to enhance gameplay.
}
\section*{Additional Guidance}

Falmouth University is nationally and internationally renowned as an arts institution.
Despite the fact that you are studying for a Bachelor of Science degree in a technical discipline,
you are still expected to strive for the same level of \textbf{innovation and creative flair}
as your fellow students in other departments.
All assignments on this course involve a mix of technical and creative activities;
this assignment is more heavily weighted towards the creative than the assignments you have completed thus far.
On this assignment, a competent execution of an unimaginative idea is unlikely to achieve higher than a C grade overall,
as opposed to an imperfect execution of a unique and ambitious concept
--- bear this in mind when working on your design.
One approach to promoting creativity is
\textbf{divergent thinking}: generation of ideas by exploring many possible solutions.
Often the most interesting ideas are \textbf{subversive}: they deliberately go against the
accepted or most obvious solution

The history of video games is littered with failed peripherals which consumers simply did not want,
which were perceived as expensive gimmicks rather than legitimate enhancements to gameplay.
Your creativity should be balanced by \textbf{commercial awareness}:
your design should be informed by your research into products that have succeeded and failed
in the past, and what underexploited niches exist in the present.
An A$^*$ project would be a highly divergent idea, but one that has clear commercial viability.
Do not be too discouraged if you fall short of this: this is a tall order even for the professionals!

We have given you some of the materials you need: a MaKey~MaKey kit, crocodile clip leads and conductive paint.
You will need to add your own materials to produce a \textbf{functional} physical prototype.
A ``Blue Peter'' style prototype made from household items is fine,
as is something made out of modelling clay, construction toys etc.
However you should still choose your materials carefully, as overly flimsy construction may
lose you marks on the functionality criterion.

You may also wish to connect electronic components such as LEDs, buzzers, photoresistors etc to the MaKey~MaKey,
or even use a different, more flexible hardware platform such as Arduino.
However you are discouraged from spending large sums of money on extra hardware,
and doing so is \textbf{not required} to achieve a high mark.
If you choose to go down this route,
it is possible to purchase an Arduino and a selection of electronic components online for 
around the price of a textbook (\textsterling 20 -- \textsterling 30).\footnote{
    Note that the MaKey~MaKey kits provided in class are version~1.2, which, unlike earlier versions, is not based on Arduino.
    Any tutorials you may find online for reprogramming the MaKey~MaKey firmware using the Arduino IDE
    are unfortunately not applicable to this version.
}

You should aim to demonstrate a high level of \textbf{sophistication}
in the technical execution of your prototype.
An important part of sophistication is having the insight to choose the right tool for the job:
if a simpler technique fulfils all the requirements, use it.
The use of unnecessarily complicated techniques, serving only to showcase one's own cleverness,
is a dangerous habit for a software developer.

The sole purpose of the \textbf{video demonstration} is to aid moderators and external examiners,
who are not present for the demo session,
in assessing your work.
Your video does \textbf{not} need to be entertaining or highly polished:
a smartphone or webcam video of you or someone else using the controller is sufficient.

%Your \textbf{weekly reports} should document the iterations you make on your design and prototyping.
%The emphasis in this assignment is on creativity and rapid iteration,
%so do not be afraid to ``go back to the drawing board'' if a prototyped idea does not work as well as anticipated.
%However it is important to document (and learn from) your failures, even more so than your successes.
%
%You are strongly encouraged to make use of sketches, diagrams, photographs, screenshots
%and short videos to document your design and prototyping process.
%Many indie game developers use such work-in-progress material as an important tool for promotion
%and community engagement; for example, search Twitter for the hashtag \texttt{\#screenshotsaturday}.
%Videos should ideally be short (5--30 second) demonstrations of functions of your prototype;
%you may narrate if you wish, but it is not required.

\begin{marginquote}
    ``The first 90 percent of the code accounts for the first 90 percent of the development time.
    
    ``The remaining 10 percent of the code accounts for the other 90 percent of the development time.''
    
    --- Tom Cargill
    
    \marginquoterule
    
    ``Hofstadter's Law:
    
    ``It always takes longer than you expect, even when you take into account Hofstadter's Law.''
    
    --- Douglas Hofstadter
\end{marginquote}
\marginpicture{fishing}{
    The \emph{Dreamcast Fishing Controller}, released as a peripheral for the game \emph{Sega Bass Fishing}.
    Even peripherals which appeal to only a small audience can enjoy moderate commercial success.
}
\section*{Additional Resources}

\begin{itemize}
    \item Wilkinson, K. and Petrich, M. (2014) The Art of Tinkering: Meet 150 Markers Working at the Intersection of Art, Science \& Technology. Weldon Owen: London.
    \item Alicia Gibb. Building Open Source Hardware: DIY Manufacturing for Hackers and Makers. Addison Wesley, 2014. 
    \item Jeremy Blum. Exploring Arduino: Tools and Techniques for Engineering Wizardry. John Wiley, 2013. 
    \item Kelly, K. (2014) Cool Tools: A Catalogue of Possibilities. Cool Tools.
    \item Hatch, M. (2013) The Maker Movement Manifesto: Rules for Innovation in the New World of Creators, Hackers, and Tinkerers. McGraw Hill: New York.
    \item \url{http://makeymakey.com/howto.php}
\end{itemize}

\begin{markingrubric}
    \firstcriterion{Sprint reviews}{Threshold \par 5\% + 5\%}
        \gradespan{2}{\fail None of the sprints are delivered, or no `reasonable' peer reviews are submitted.}
        \gradespan{2}{A `potentially shippable' component is produced at the end of the final sprint.
            \par A `reasonable' review of at least one peer's work is provided in one of the review sessions.}
        \gradespan{2}{A `potentially shippable' component is produced at the end of all three sprints.
            \par A `reasonable' review of at least one peer's work is provided in each of the review sessions.}
    \criterion{Design of the solution}{15\%}
        \grade\fail User stories are not provided, or the design does not correspond to the user stories.
        \grade Few user stories are distinguishable and easily measured.
            \par The correspondence between design and user stories is tenuous.
        \grade Some user stories are distinguishable and easily measured.
            \par The design somewhat corresponds to the user stories.
        \grade Most user stories are distinguishable and easily measured.
            \par The design corresponds to the user stories.
        \grade Nearly all user stories are distinguishable and easily measured.
            \par The design clearly corresponds to the user stories.
        \grade All user stories are distinguishable and easily measured.
            \par The design clearly and comprehensively corresponds to the user stories.
    \criterion{Commercial awareness}{10\%}
        \grade\fail No commercial awareness is demonstrated.
        \grade Emerging commercial awareness is demonstrated.
            \par There is no evidence of market research.
        \grade Some commercial awareness is demonstrated.
            \par Market research is present, but with significant gaps.
        \grade Much commercial awareness is demonstrated.
            \par Market research is extensive, but with some gaps.
        \grade Significant commercial awareness is demonstrated.
            \par Market research is comprehensive.
        \grade Exemplary commercial awareness is demonstrated.
            \par Market research is comprehensive and insightful.
    \criterion{Innovation and creative flair}{30\%}
        \grade\fail Demonstrates no evidence of innovation and/or creativity.
        \grade Demonstrates evidence of emerging innovation and/or creativity.
            \par The solution is purely derivative of existing products.
            \par There is no evidence of divergent thinking.
        \grade Demonstrates evidence of progressing innovation and/or creativity.
            \par The solution is mostly derivative, with some attempts at innovation.
            \par There is evidence of an attempt at divergent thinking.
        \grade Demonstrates evidence of partial mastery of innovative and creative practice.
            \par The solution is an interesting and somewhat innovative product.
            \par There is some evidence of divergent thinking.
        \grade Demonstrates some evidence of mastery of innovative and creative practice.
            \par The solution is a novel and innovative product.
            \par There is much evidence of divergent thinking.
        \grade Demonstrates much evidence of mastery of innovative and creative practice.
            \par The solution is a unique and innovative product.
            \par There is significant evidence of divergent thinking.
    \criterion{Functionality of physical prototype}{10\%}
        \grade\fail A physical prototype is not produced, or the prototype is completely non-functional.
        \grade The physical prototype is barely functional.
            \par There are serious technical and/or constructional flaws.
        \grade The physical prototype is somewhat functional.
            \par There are obvious technical and/or constructional flaws.
        \grade The physical prototype is mostly functional.
            \par There are minor technical and/or constructional flaws.
        \grade The physical prototype is functional.
            \par There are superficial technical and/or constructional flaws.
        \grade The physical prototype is functional.
            \par The technical execution and physical construction are flawless.
    \criterion{Sophistication: \par Software \par Electronics \par Physical construction}{20\%}
        \grade\fail The solution lacks even a basic level of sophistication in any of the three areas.
        \grade The solution is basic and unsophisticated in all three areas.
            \par Little insight has been demonstrated in any area.
        \grade The solution is moderately sophisticated in one of the areas, but lacking in the other two.
            \par Emerging insight has been demonstrated in at least one of the areas.
        \grade The solution is moderately sophisticated in two of the noted areas, but lacking in the third.
            \par Much insight has been demonstrated in at least one of the areas.
        \grade The solution combines somewhat sophisticated software, electronics and physical construction.
            \par Significant insight has been demonstrated in at least two of these areas.
        \grade The solution combines highly sophisticated software, electronics and physical construction.
            \par Exemplary insight has been demonstrated in all three areas.
    \criterion{Professional practice}{5\%}
        \grade\fail GitHub has not been used.
        \grade Material has only been checked into GitHub a few times before the deadline.
        \grade Material has been checked into GitHub at least once per sprint.
        \grade Material has been checked into GitHub several times per sprint.
        \grade Material has been checked into GitHub several times per sprint.
            \par Commit messages are clear, concise and relevant.
        \grade Material has been checked into GitHub several times per sprint.
            \par Commit messages are clear, concise and relevant.
            \par There is evidence of engagement with peers (e.g.\ voluntary code review).
\end{markingrubric}

\end{document}
