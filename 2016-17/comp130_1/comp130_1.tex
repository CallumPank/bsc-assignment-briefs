\documentclass{../fal_assignment}
\graphicspath{ {../} }

\usepackage{enumitem}
\setlist{nosep} % Make enumerate / itemize lists more closely spaced
\usepackage[T1]{fontenc} % http://tex.stackexchange.com/a/17858
\usepackage{url}
\usepackage{todonotes}

\title{Worksheet Tasks}
\author{Dr Ed Powley}

\begin{document}

\maketitle

\section*{Introduction}

\begin{marginquote}
``As soon as we started programming, we found out to our surprise that it wasn't as easy to get programs right as we had thought. Debugging had to be discovered. I can remember the exact instant when I realized that a large part of my life from then on was going to be spent in finding mistakes in my own programs.''
\par --- Maurice Wilkes
\marginquoterule
\par ``C++ is history repeated as tragedy. Java is history repeated as farce.''
\par --- Scott McKay
\end{marginquote}
\marginpicture{flavour_pic}{
    The \textbf{MetaMakers} Institute has been set up to undertake cutting edge Computational Creativity research here at \textbf{Falmouth University}. The main focus of the institute is to develop, combine and extend artificial intelligence techniques in order to build truly inspirational software, and we undertake philosophical enquiries into what it means for software to be creative. 
}

In this assignment, you are required to \textbf{design}, \textbf{annotate}, and \textbf{write} a series of computer programs according to instructions.

In order for programmers to communicate with each other regarding the technical aspects of a game development project, they must have good computational thinking skills, a strong foundational knowledge of computing principles, applied knowledge of program design notations and annotations, and a working knowledge of particular programming constructs (often as a result of writing their own versions). Such knowledge and skills take time and a sustained effort to develop. For this reason, you will work consistently across the semester by completing a series of bite-sized worksheets.

This assignment is formed of \textbf{five} parts.
Each part corresponds to one worksheet, addressing the following five learning objectives:
\begin{enumerate}[label=(\Alph*)]
	\item \textbf{Apply} the basic principles of programming in a new language;
	\item \textbf{Implement} algorithms from their notations such as flowcharts and pseudocode;
	\item \textbf{Apply} theoretical concepts such as object-orientated design patterns;
	\item \textbf{Express} the architecture of a game using industry-standard notation;
	\item \textbf{Extend} the functionality of a broad game architecture context.
\end{enumerate}

For each worksheet you must:
\begin{enumerate}[label=(\roman*)]
    \item \textbf{Read} the instructions in the worksheet;
    \item \textbf{Complete} all of the problems presented in the worksheet;
    \item \textbf{Bring} your solution to class on the date specified on the worksheet,
    	where it will be marked.
\end{enumerate}

\subsection*{Assignment Setup}

This assignment consists of \textbf{five formative submissions}, followed by a \textbf{single summative submission}.
You will receive \textbf{immediate informal feedback} after each formative submission.

Each worksheet contains detailed submission instructions; you will generally be required to \textbf{fork} a repository on GitHub and submit a \textbf{pull request} containing your solution. You will be instructed how to do this in class as part of module COMP150.

At the end of the semester you will be required make a final summative submission of all five of your worksheet solutions.
Prepare a \textbf{single \texttt{.zip} file} containing your five worksheet submissions \textbf{in five separate folders}, and upload it to the appropriate submission area on LearningSpace.
This submission is for archival purposes only; at this stage your work has already been marked and you have received feedback, and you should \textbf{not} submit any new, unmarked work via LearningSpace.
This final submission is subject to the usual university policies regarding late submission or non-submission,
as detailed in the course handbook ---
even if you have met all the formative deadlines for individual worksheets,
failure to make a submission via LearningSpace by the summative deadline will be subject to penalties.

\section*{Additional Guidance}

Make a submission on time and you will get a basic pass on that worksheet,
even if your solution is incorrect or incomplete.
A solution meeting all of the correctness and/or functionality criteria on the worksheet is required to demonstrate basic proficiency,
with higher grades contingent on your solution being of a high quality.
The individual worksheets give more guidance as to what constitutes ``quality'' for that particular exercise,
but bear in mind that a major purpose of these worksheets is to assess your ability to communicate
complex computational ideas in English, in notation and in program code.
Thus pay particular attention to the precision and clarity of your written communication,
and the readability and maintainability of your source code.

It is very important to keep up with the worksheets. Missing a deadline results in an automatic mark of 0\% for that worksheet.
The underlying skills being developed are also critically important to your progression as a programmer, so do not neglect the work.
Do not underestimate the time it takes to complete tasks that may appear trivial when you first see them.
Do not leave work until the last minute! With programming in particular, trying to ``cram'' the work just before the deadline is a sure path to failure. Aim for consistent, steady progress over the course of the semester.

Nobody learns in a vacuum: you are allowed, and indeed encouraged, to discuss your work with your peers. However you must be very careful to avoid falling into \textbf{academic misconduct}, in particular \textbf{plagiarism}. If any part of your solution is \textbf{not your own individual work}, you must make this as clear as possible in your submission, for example in source code comments.

\section*{FAQ}

\begin{itemize}
	\item 	\textbf{What is the deadline for this assignment?} \\ 
			Each worksheet has its own formative deadline, specified on that worksheet and also communicated in class.
    		Falmouth University policy states that summative deadlines must only be specified on the MyFalmouth system.
    		
	\item 	\textbf{What should I do to seek help?} \\ 
    		You can email your tutor for informal clarifications. For informal feedback, make a pull request on GitHub. 
    		
	\item 	\textbf{How will I receive feedback on my work?} \\ 
    		You will be given verbal feedback on your work during the session in which it is marked.
    		If you require more in-depth feedback or discussion, please book an appointment with your tutor.
    		
    	\item 	\textbf{Is this a mistake?} \\ 	
    		If you have discovered an issue with the brief itself, the source files are available at: \\
    		\url{https://github.com/Falmouth-Games-Academy/bsc-assignment-briefs}.\\
    		 Please make a pull request and comment accordingly.
\end{itemize}

\section*{Additional Resources}

Please see individual worksheets.

%\begin{itemize}
%    \item Keith, C. (2010) Agile Game Development with Scrum. Pearson Education.
%    \item http://agilemanifesto.org/
%\end{itemize}

\begin{markingrubric}
%    \firstcriterion{Timely submission}{40\%}
%        \gradespan{5}{\fail Nothing is submitted by the deadline, or the submission is not a reasonable attempt at the
%        	problems set forth in the worksheet.
%        	\par \textbf{This results in a mark of 0\% for this worksheet, regardless of other criteria.}}
%        \gradespan{1}{A reasonable attempt at the worksheet is submitted on time.}
%
    \firstcriterion{Worksheets}{20\% $\times$ 5 worksheets}
        \grade\fail	A reasonable attempt at the worksheet is not submitted by the deadline.
        \grade 		The submission is on time.
        \par		The submission is a reasonable attempt, but is incomplete or incorrect.
        \grade 		The submission is on time.
        \par 		The submission is complete and correct.
        \par		The submission has significant quality issues.
        \grade 		The submission is on time.
        \par 		The submission is complete and correct.
        \par		The submission has some quality issues.
        \grade 		The submission is on time.
        \par 		The submission is complete and correct.
        \par		The submission has very few quality issues.
        \grade 		The submission is on time.
        \par 		The submission is complete and correct.
        \par		The submission has almost no quality issues.
\end{markingrubric}

\end{document}