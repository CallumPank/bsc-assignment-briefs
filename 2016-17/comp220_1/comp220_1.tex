\documentclass{../fal_assignment}
\graphicspath{ {../} }

\usepackage{enumitem}
\setlist{nosep} % Make enumerate / itemize lists more closely spaced
\usepackage[T1]{fontenc} % http://tex.stackexchange.com/a/17858
\usepackage{url}
\usepackage{todonotes}

\title{Portfolio of game engine components --- Graphics}
\author{Dr Ed Powley}

\begin{document}

\maketitle

\section*{Introduction}

\begin{marginquote}
``Bad programming is easy. [People] can learn it in 21 days, even if they are dummies...[Good programming requires a] willingness to devote a large portion of one's life to deliberative practice...So go ahead, buy that book; you'll probably get some use out of it. But you won't change your life or your real expertise as a programmer in 21 days...How about working hard to continually improve over 24 months? Well, now you're starting to get somewhere...''
\par --- Peter Norvig
\end{marginquote}
\marginpicture{MakeyMakey.jpg}{
    The \emph{MaKey~MaKey} allows a multitude of materials to be used to create videogame controllers.
}

In this assignment, you are required to \textbf{design} and \textbf{implement} a C++ program using SDL and OpenGL
which demonstrates the type of 3D computer graphics techniques that appear in a modern game engine.

Justification bla bla bla.

This assignment is formed of several parts:
\begin{enumerate}[label=(\Alph*)]
	\item \textbf{Propose} a design for your game demo. Your proposal should:
		\begin{enumerate}[label=(\roman*)]
			\item \textbf{describe} the concept of your demo;
			\item \textbf{explain} how your concept meets the requirements;
			\item \textbf{do} another thing probably.
		\end{enumerate}
	\item \textbf{Formulate} the mathematical foundations for your demo;
	\item \textbf{Implement} the thing;
	\item \textbf{Demonstrate} the thing.
\end{enumerate}

\subsection*{Assignment Setup}

This assignment is a \textbf{programming} task. Fork the GitHub repositories at the following URL:

\indent \url{https://github.com/Falmouth-Games-Academy/comp220-portfolio}

Use the existing directory structure and, as required, extend this structure with sub-directories.
Ensure that you maintain the \texttt{readme.md} file.

Modify the \texttt{.gitignore} to the defaults for \textbf{Visual Studio}.
Please, also ensure that you add editor-specific files and folders to \texttt{.gitignore}. 

\subsection*{Part A}

Part A consists of a \textbf{single formative submission}.
This work is \textbf{individual} and will be assessed on a \textbf{threshold} basis.
Your pitch should address the following questions:

\begin{itemize}
	\item What is the title and high concept of the game?
	\item What is the intended aesthetic?
	\item What is the core mechanic? 
	\item What makes the game fun?
	\item Is there a market for this type of game? Who is the target audience?
	\item What are the unique selling points?
	\item Is the scope appropriate for the product development time-frame?
\end{itemize}

To complete Part A, prepare the handout using any word processing tool. There is no submission.

Show the handout to your \textbf{tutor} for immediate \textbf{informal feedback}.

\section*{Additional Guidance}

Bla bla bla.

\section*{FAQ}

\begin{itemize}
	\item 	\textbf{What is the deadline for this assignment?} \\ 
    		Falmouth University policy states that summative deadlines must only be specified on LearningSpace. Please examine the assignment area where you located this document.
    		
	\item 	\textbf{What should I do to seek help?} \\ 
    		You can email your tutor for informal clarifications. For informal feedback, make a pull request on GitHub. 
    		
    	\item 	\textbf{Is this a mistake?} \\ 	
    		If you have discovered an issue with the brief itself, the source files are available at: \\
    		\url{https://github.com/Falmouth-Games-Academy/bsc-assignment-briefs}.\\
    		 Please make a pull request and comment accordingly.
\end{itemize}

\section*{Additional Resources}

\begin{itemize}
    \item Keith, C. (2010) Agile Game Development with Scrum. Pearson Education.
    \item http://agilemanifesto.org/
\end{itemize}

\begin{markingrubric}
	\firstcriterion{sdgagdsa}
%
    \criterion{Appropriateness of User Stories and Sprint Plans}{5\% $\dagger$}
        \grade\fail No user stories and/or sprint plans are provided.
        \grade Few user stories are distinguishable and easily measured.
            \par Sprint plans provide little support for the project.
        \grade Some user stories are distinguishable and easily measured.
            \par Sprint plans provide some support for the project.
        \grade Most user stories are distinguishable and easily measured.
            \par User stories correspond to the game design.
            \par Sprint plans provide much support for the project.
        \grade Nearly all user stories are distinguishable and easily measured.
            \par User stories clearly correspond to the game design.
            \par Sprint plans provide effective support for the project.
        \grade All user stories are distinguishable and easily measured.
            \par User stories clearly and comprehensively correspond to the game design.
            \par Sprint plans provide exemplary support for the project.
%
    \criterion{Functional Coherence}{5\% $\dagger$}
        \grade\fail No gameplay elements have been implemented and/or the code fails to compile or run.
        \grade Few gameplay elements have been implemented.
            \par There are many obvious and serious bugs.
        \grade Some gameplay elements have been implemented.
            \par There are some obvious bugs.
        \grade Many gameplay elements have been implemented.
            \par There is some evidence of feature creep.
            \par There are few obvious bugs.
        \grade Almost all gameplay elements have been implemented.
            \par There is little evidence of feature creep.
            \par There are some minor bugs.
        \grade All gameplay elements have been implemented.
            \par There is no evidence of feature creep.
            \par Bugs, if any, are purely cosmetic and/or superficial.
%
    \criterion{Performance}{5\% $\dagger$}
        \grade\fail The executable has unacceptably many performance issues.
        \grade The executable has many performance issues.
        \grade The executable has some performance issues.
        \grade The executable has very few performance issues.
        \grade The executable has almost no performance issues.
        \grade The executable has no performance issues.
            \par Measures have been taken to degrade gracefully on less powerful hardware.
%
    \criterion{Sophistication}{10\% $\dagger$}
        \grade\fail No insight into the appropriate use of programming constructs is evident from the source code.
            \par No attempt to structure the program is evident (e.g. one monolithic source file).
        \grade Little insight into the appropriate use of programming constructs is evident from the source code.
            \par The program structure is poor.
        \grade Some insight into the appropriate use of programming constructs is evident from the source code.
            \par The program structure is adequate.
        \grade Much insight into the appropriate use of programming constructs is evident from the source code.
            \par The program structure is appropriate.
        \grade Significant insight into the appropriate use of programming constructs is evident from the source code.
            \par The program structure is effective. There is high cohesion and low coupling.
        \grade Exemplary insight into the appropriate use of programming constructs is evident from the source code.
            \par The program structure is very effective. There is high cohesion and low coupling.
%
    \criterion{Maintainability}{20\% $\dagger$}
        \grade\fail There are no comments, or comments are misleading.
            \par Most variable names are unclear or inappropriate.
            \par Code formatting hinders readability.
        \grade The code is only sporadically commented, or comments are unclear.
            \par Some identifier names are unclear or inappropriate.
            \par Code formatting is inconsistent or does not aid readability.
        \grade The code is well commented.
            \par Some identifier names are descriptive and appropriate.
            \par An attempt has been made to adhere to a consistent formatting style.
             \par There is little obvious duplication of code or of literal values.           
        \grade The code is reasonably well commented.
            \par Most identifier names are descriptive and appropriate.
            \par Most code adheres to a consistent formatting style.
             \par There is almost no obvious duplication of code or of literal values.   
        \grade The code is reasonably well commented, with some Doxygen-compatible module documentation.
            \par Almost all identifier names are descriptive and appropriate.
            \par Almost all code adheres to a consistent formatting style.
             \par There is no obvious duplication of code or of literal values. Some literal values can be easily ``tinkered''. 
        \grade The code is very well commented, with comprehensive Doxygen-compatible module documentation.
            \par All identifier names are descriptive and appropriate.
            \par All code adheres to a consistent formatting style.
             \par There is no obvious duplication of code or of literal values. Most literal values are, where appropriate, easily ``tinkered'' outside of the source.  
%
    \criterion{Portability and Navigability}{5\% $\dagger$}
        \grade\fail Game will not execute at all on another machine for reasons related to code portability which cannot be fixed easily due to its poor structure.
            \par The provided template has not been followed.
        \grade There were challenges executing the game, but these were resolvable.
            \par The directory structure inside the submitted zip file is unclear.
            \par The provided template has not been followed.
        \grade Several portability issues are present.
            \par The directory structure inside the submitted zip file is somewhat confusing.
            \par The provided template has mostly been followed.
        \grade Some portability issues are present.
            \par The directory structure inside the submitted zip file is adequate.
            \par The provided template has been followed.
        \grade Few portability issues are present.
            \par The directory structure inside the submitted zip file is mostly sensible.
            \par The provided template has been followed.
        \grade Almost no portability issues are present.
            \par The directory structure inside the submitted zip file is sensible.
            \par The provided template has been followed.
%
    \criterion{Use of Version Control}{10\%}
        \grade\fail GitHub has not been used.
        \grade Material has been checked into GitHub less frequently than once per sprint.
            \par All code has been checked into the Master branch.
        \grade Code has been checked into GitHub at least once per sprint.
            \par An attempt has been made to use branches.
        \grade Code has been checked into GitHub several times per sprint.
            \par Commit messages are clear, concise and relevant.
            \par Branches are used sensibly.
        \grade Code has been checked into GitHub several times per sprint.
            \par Commit messages are clear, concise and relevant.
            \par Branches are used somewhat effectively.
            \par There is evidence of engagement with peers (e.g.\ code review).
        \grade Code has been checked into GitHub several times per sprint.
            \par Commit messages are clear, concise and relevant.
            \par Branches are used effectively.
            \par There is significant evidence of engagement with peers (e.g.\ code review).
\end{markingrubric}

\subsection*{Contract --- to be given as a separate document, but in here for now}

You must design and implement a game demo making use of 3D graphics techniques.
The demo must allow the player to move around a simple 3D environment in first-person perspective,
and must have a simple gameplay objective.

The demo must meet the following requirements:

\begin{itemize}
	\item A scene containing at least one textured mesh and at least one light source.
	\item Standard first-person movement controls: move with the WASD keys, look around with the mouse.
		Extra controls (e.g.\ shoot, jump, interact) may be added if required.
	\item A \textbf{simple} gameplay objective such as collecting objects, killing enemies, or solving a puzzle.
	\item Any \textbf{three} of the following graphics and simulation techniques:
		\begin{itemize}
			\item Procedural generation of complex meshes;
			\item Rendering of semi-transparent materials;
			\item Skeletal animation;
			\item Collision detection;
			\item Integration of a physics engine (e.g.\ Bullet, PhysX);
			\item Particle effects (e.g.\ fire, smoke, dust, explosions, water, blood);
			\item An advanced real-time lighting effect (e.g.\ shadow casting, real-time reflections, volumetric lighting, bloom);
			\item A stylised full-screen filter (e.g.\ cel shading, halftone, CRT monitor effect, motion blur);
			\item Other advanced rendering or simulation techniques of your choice, to be discussed with your tutor.
		\end{itemize}
\end{itemize}

\end{document}