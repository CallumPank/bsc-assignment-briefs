\documentclass{../fal_assignment}
\graphicspath{ {../} }

\usepackage{enumitem}
\setlist{nosep} % Make enumerate / itemize lists more closely spaced
\usepackage[T1]{fontenc} % http://tex.stackexchange.com/a/17858
\usepackage{url}
\usepackage{todonotes}

\title{Ethics \& Professionalism Essay}
\author{Dr Michael Scott}

\newcommand{\proposalWordCount}{200}
\newcommand{\essayWordCount}{1000}
\newcommand{\presentationMinutes}{30}
\newcommand{\minReferenceCount}{12 }

\begin{document}

\maketitle

\section*{Introduction}

\begin{marginquote}
  ``Beware of bugs in the above code; I have only proved it correct, not tried it!''
  
   --- Donald Knuth
   
    \marginquoterule
    
    ``I'm not a real programmer. I throw together things until it works then I move on. The real programmers will say: Yeah, it works but you're leaking memory everywhere. Perhaps we should fix that!''
    
    --- Rasmus Lerdorf
    
    \marginquoterule
    
    ``I'm sorry Dave, I'm afraid I can't do that.''
    
    --- HAL 9000
      
\end{marginquote}
\marginpicture{flavour_pic}{
    Edward Tufte makes the case that the 2003 Columbia disaster was caused through the misuse of PowerPoint for communication.
}

In this assignment, you will research ethical and professional issues in the games development community and bring an academic perspective to its culture. Specifically, to explore: 

\begin{enumerate}[label=(\roman*)]
    \item Which ethical and professional challenges are prominent;
    \item and how professionals can conduct themselves in an ethical manner.
\end{enumerate}

The culture of the games industry will not only play a key role in shaping your future job satisfaction, but it will also be instrumental in determining the form of the cultural artefacts that you produce. You need to explore your values and determine what is important to you. Issues in the games industry span a wide range of topics, from \textit{EA Spouse} to \textit{Hot Coffee}, from the \textit{Edge Fiasco} to \textit{GamerGate}, from \textit{ConFlag Censorship} to \textit{Bondi Snubs}. Not to mention, controversy surrounding issues such as anti-social behaviour, violence, and addiction. Notions of ethical principles, codes of conduct, and professional bodies (e.g., BCS, ACM, IGDA, etc.) will aid such explorations. 

This assignment is formed of several parts:

\begin{enumerate}[label=(\Alph*)]
    \item \textbf{Write} a \proposalWordCount-word proposal \textbf{with} references which must:
    	\begin{enumerate}[label=\roman*.]
    		\item \textbf{justify} the importance of \textbf{one} ethical and/or professional issue;
    		\item and then \textbf{cite} at least \textbf{\minReferenceCount} appropriate academic references.
	\end{enumerate}
    \item \textbf{Present}, as a \textbf{group}, a \presentationMinutes-minute summary of your research that will:
    	\begin{enumerate}[label=\roman*.]
    		\item \textbf{clarify} each issue \textbf{and} its importance;
    		\item \textbf{identify} appropriate research questions \textbf{and} avenues for research;
    		\item \textbf{debate how} each issue could be resolved;
    		\item and \textbf{discuss how} these issues affect the development community.
	\end{enumerate}
    \item \textbf{Write} a draft \essayWordCount-word essay which will:
    	\begin{enumerate}[label=\roman*.]
    		\item  \textbf{address} the research question.
	\end{enumerate}
    \item \textbf{Write} a final \essayWordCount-word essay which will:
    	\begin{enumerate}[label=\roman*.]
    		\item \textbf{revise} any issues raised by your tutor and/or your peers.
	\end{enumerate}
\end{enumerate}

\todo[inline]{\textbf{Note:} All research questions must be distinctive. Members of the same group must \textbf{not} target the same research question.}

\subsection*{Assignment Setup}

This assignment is an \textbf{academic writing task}. Fork the GitHub repository at the following URL:

\indent \url{https://github.com/Falmouth-Games-Academy/comp230-ethics}

Use the existing directory structure and, as required, extend this structure with sub-directories. Ensure that you maintain the \texttt{readme.md} file.

Modify the \texttt{.gitignore} to the defaults for \textbf{TeX}. Please, also ensure that you add editor-specific files and folders to \texttt{.gitignore}. 

\subsection*{Part A}

Part A consists of a \textbf{single formative submission}. This work is \textbf{individual} and will be assessed on a \textbf{threshold} basis. The following criteria are used to determine a pass or fail:

\begin{enumerate}[label=(\alph*)]
	\item Submission is timely;
	\item Research question is appropriate and distinctive;
	\item At least \minReferenceCount  academic peer-reviewed sources are cited.
\end{enumerate}

To complete Part A, write your proposal in the \texttt{readme.md} document and then prepare the reference list using a \texttt{*.bib} file.  Show these to your tutor.  If acceptable, this will be signed-off. 

You will receive immediate \textbf{informal feedback}.

\subsection*{Part B}

Part B is a \textbf{single formative submission}. This work is \textbf{collaborative} and will be assessed on a \textbf{threshold} basis. The following criteria are used to determine a pass or fail:

\begin{enumerate}[label=(\alph*)]
	\item Research questions are adequately addressed;
	\item Some evidence of academic rigour;
	\item Some insight into the relationship between theory and practice.
\end{enumerate}

To complete Part B, prepare a presentation, and practice your debate and discussion. Prepare your slideshow collaboratively in TeX. Use the combined reference list of the group to broadly discuss each individual research question. Help each other. Ensure that the source code and related assets are pushed to GitHub prior to the scheduled session. Then, attend the scheduled session.

You will receive \textbf{peer feedback} within 3 working days after the session.

\subsection*{Part C}

Part C is a \textbf{single formative submission}. This work is \textbf{individual} and will be assessed on a \textbf{threshold} basis. The following criteria are used to determine a pass or fail:

\begin{enumerate}[label=(\alph*)]
	\item Submission is timely;
	\item Enough work is available to conduct a meaningful review;
	\item A broadly appropriate review of a peer's work is submitted.
\end{enumerate}

To complete Part C, prepare a draft version of the essay. Ensure that the source code and related assets are pushed to GitHub and a pull request is made prior to the scheduled session. Then, attend the scheduled session.

You will receive \textbf{peer feedback} within 3 working days after the session.

\subsection*{Part D}

Part D is a \textbf{single summative submission}. This work is \textbf{individual} and will be assessed on a \textbf{criterion-referenced} basis. Please refer to the marking rubric at the end of this document for further detail.

To complete Part D, revise the essay based on the feedback you have received. Then, upload the essay to the LearningSpace. Please note, the LearningSpace will only accept a single \texttt{.pdf} file.

You will receive \textbf{formal feedback} three weeks after the final deadline.

\section*{Additional Guidance}

As you progress into your second year, you will discover that a much greater level of intellectual independence is expected of you. Sessions now focus on student-driven dialogues where important issues are explored instead of merely presenting material. Your tutor is there to highlight opportunities for learning and to facilitate the dialogue. Not to provide answers. It is, therefore, critically important that all students research the topic for each dialogue in advance of attending. These are indicated to you in the session schedule on the LearningSpace.

Again, identifying the most appropriate ethical or professional challenge to address and then developing an appropriate research question is the most challenging aspect of these assignments. It is very unlikely that you will settle on the first research question that you propose. Question arise out of research and gaps in the literature. Furthermore, the question should relate to issues in the games industry. An example might be: ``What legal powers does an employee have to remedy a situation where a company refuses to credit their contribution to a game they worked on?''. You will need to discuss your question with your tutor and your peers to help focus it.

Areas where students tend to lose marks are: depth of insight; analytical skill; and evaluative skill. Depth of insight implies rigorous research, addressing one key challenge in much detail, rather than several challenges with weaker research and/or in less detail. Adequate analysis implies going beyond mere description, perhaps through: performing calculations, comparing sources, or even deploying reasoning to generate new insights. Adequate evaluation implies making appropriate reference to evidence and ensuring that evidence is of appropriate quality. Further to this, sound and valid arguments are constructed, criticising the claims made by other authors.

Focus on answering your research question. You have but 1000-words! Depth over breadth. Quality over quantity. Write concisely. Your ability to recall facts is not under assessment, your ability to construct an argument through critical analysis and making it relevant to practice is.

\section*{FAQ}

\begin{itemize}
	\item 	\textbf{What is the deadline for this assignment?} \\ 
    		Falmouth University policy states that deadlines must only be specified on the MyFalmouth system.
    		
	\item 	\textbf{What should I do to seek help?} \\ 
    		You can email your tutor for informal clarifications. For informal feedback, make a pull request on GitHub. 
    		
    	\item 	\textbf{Is this a mistake?} \\ 	
    		If you have discovered an issue with the brief itself, the source files are available at: \\
    		\url{https://github.com/Falmouth-Games-Academy/bsc-assignment-briefs}.\\
    		 Please make a pull request and comment accordingly.
\end{itemize}

\section*{Additional Resources}

\begin{itemize}
    \item Baase, S. (2012) A Gift of Fire: Social, Legal, and Ethical Issues for Computing Technology. Pearson Education.
    \item Sicart, M. (2009) The Ethics of Computer Games. MIT Press.
    \item \url{https://www.igda.org/?page=codeofethics}
    \item \url{http://www.bcs.org/category/6030}
    \item \url{https://www.acm.org/about-acm/acm-code-of-ethics-and-} \\ \url{professional-conduct}
\end{itemize}

\rubricyeartwo

\begin{markingrubric}
%
    \firstcriterion{Basic Proficiency Threshold}{40\% (Threshold)}
        \gradespan{1}{\fail Parts A---C have not been submitted, are incomplete, or are unsatisfactory.}
        \gradespan{5}{Parts A---C are complete. 
        \par At least ten relevant sources have been referenced.
        \par		Where appropriate, all sources report scholarly research.
        \par		Some appropriate seminal and highly reputed sources have been referenced.}
%
    \criterion{Relevance of and Focus on the Research Question}{10\%}
%        \grade\fail 	No focus on the research question.
        \grade  \fail	Little to no focus on a specific research question.
        \grade 		Some focus on specific research questions.
            \par 		Appropriate references to the games industry are used to cite the relevance of the research question.
        \grade 		Much focus on specific research questions.
            \par 		Research questions are explicitly defined.
            \par 		Appropriate references to the games industry are used to argue the relevance of the research question.
        \grade 		Considerable focus on specific research questions.
            \par 		Research questions are explicitly defined.
            \par 		Conclusion explicitly refers back to the question.
            \par 		Appropriate references to the games industry are used to argue the relevance of the research question.
        \grade 		Significant focus on a single, specific research question.
            \par 		The research question is explicitly defined.
            \par 		Conclusion explicitly refers back to the question.
            \par 		Appropriate references to case studies and examples of issues in the games industry are used to argue and demonstrate the relevance of the research question.
        \grade 		Extensive focus on a single, specific research question.
            \par 		The research question is explicitly defined.
            \par 		Conclusion explicitly refers back to the question.
            \par 		Appropriate references to case studies and examples of issues in the games industry are used to argue and demonstrate the relevance of the research question.
%
    \criterion{Depth of Insight into Ethics and Professionalism}{15\%}
%        \grade\fail 	No depth of insight into software engineering principles.
        \grade  \fail	Little to no depth of insight into ethical conduct and professional values.
        \grade 		Some depth of insight into ethical conduct and professional values.
        \par 		Insights highlight a specific ethical and/or professional challenge in digital games development.
        \grade 		Much depth of insight into ethical conduct and professional values.
        \par 		Insights highlight the pertinence of a specific ethical and/or professional challenge in digital games development.
        \grade 		Considerable depth of insight into ethical conduct and professional values.
        \par 		Insights explore, in detail, the pertinence of a specific ethical and/or professional challenge in digital games development.
        \grade 		Significant depth of insight into ethical conduct and professional values.
        \par 		Critical insights explore, in detail, the pertinence of a specific and important ethical and/or professional challenge in digital games development.
         \par 		Insights highlight a potential solution to address the challenge.
        \grade 		Extensive depth of insight into ethical conduct and professional values.
        \par 		Critical insights explore, in detail, the pertinence of a specific and important ethical and/or professional challenge in digital games development.
         \par 		Insights highlight a potentially effective solution to address the challenge.
%
    \criterion{Specificity, Verifiability, \& Accuracy of Claims}{5\%}
%        \grade\fail 	No citations to evidence to claims.
%        \par 		Substantial errors and/or misinterpretations.
        \grade  \fail	Few claims have a clear source of evidence.
        \par 		Significant errors and/or misinterpretations.
        \grade 		Some claims have a clear source of evidence.
        \par 		Many errors and/or misinterpretations.
        \grade 		Many claims have a clear source of evidence.
        \par 		Some errors and/or misinterpretations.
        \grade 		Most claims have a clear source of evidence.
        \par 		Few errors and/or misinterpretations.
        \grade 		All claims have a clear source of evidence.
        \par 		Almost no errors and/or misinterpretations.
        \par 		Appropriate forms of evidence are used to support some claims.
        \grade 		All claims have a clear source of evidence.
        \par 		Almost no errors and/or misinterpretations.
      \par 		Appropriate forms of evidence are used to support many claims.
%
    \criterion{Adequacy of Analysis of Research Articles}{15\%}
%        \grade\fail 	No analysis has been presented.
        \grade  \fail	Little to no analysis has been presented.
        \grade 		Some analysis has been presented. 
        \grade 		Much analysis has been presented.
        \grade 		Considerable analysis has been presented.
        \grade 		Significant analysis has been presented.
         \par		The limitations of most analyses is explicitly acknowledged.
         \grade 		Extensive analysis has been presented.
         \par		The limitations of most analyses is explicitly acknowledged.
	\par		Some limitations are addressed.
%
    \criterion{Appropriateness of Academic Writing}{5\%}
%        \grade\fail 	Little or no evidence of partial-mastery of academic writing.
%        \par 		The reference section is missing.
        \grade  \fail	Little to no evidence for mastery of academic writing.
        \par 		The reference section is incomplete and/or malformed.
        \grade 		Some evidence of mastery of academic writing.
        \par 		The reference section is nearly complete and mostly well-formed in either ACM or IEEE format.
        \par 		Most in-text citations and quotations are correct.
        \grade 		Some evidence of mastery of academic writing.
        \par 		The reference section is complete and well-formed in either ACM or IEEE format.
        \par 		All in-text citations and quotations are correct.
        \grade 		Much evidence of mastery of academic writing.
        \par 		The reference section is complete and well-formed in either ACM or IEEE format.
        \par 		All in-text citations and quotations are correct.
        \grade 		Considerable evidence of mastery of academic writing.
        \par 		The reference section is complete and well-formed in either ACM or IEEE format.
        \par 		All in-text citations and quotations are correct.
        \grade 		Significant evidence of mastery of academic writing.
        \par 		The reference section is complete and well-formed in either ACM or IEEE format.
        \par 		All in-text citations and quotations are correct.
%
    \criterion{Appropriateness of Spelling \& Grammar}{5\%}
%        \grade\fail 	Substantial spelling and/or grammar errors.
        \grade  \fail	Many spelling and/or grammar errors.
        \grade 		Some spelling and/or grammar errors.  
        \grade 		Few spelling and/or grammar errors.
        \grade 		Almost no spelling and/or grammar errors.
        \grade 		No spelling or grammar errors.
        \par 		Active voice is prevalent.
        \grade 		No spelling or grammar errors.
        \par 		Active voice is prevalent.
        \par 		Grammar is leveraged deliberately to draw attention to salient points.     
%
    \criterion{Appropriateness of Essay Structure}{5\%}
%    \grade\fail 	There is no structure, or the structure is unclear.
        \grade \fail	There is little to no structure.
        \par 		Only a few sentences and paragraphs are well constructed.
        \grade 		There is some structure.
        \par 		Some sentences and paragraphs are well constructed.
        \grade 		There is much structure.
        \par 		Many sentences and paragraphs are well constructed.
        \par 		There is a clear introduction and conclusion.
        \grade 		There is considerable structure.
        \par 		Most sentences and paragraphs are well constructed.
        \par 		There is a clear and well-constructed introduction and conclusion.
        \grade 		There is significant structure, leveraged to effectively highlight the argument and key takeaway points.
        \par 		Nearly all sentences and paragraphs are well constructed.
        \par 		There is a clear and well-constructed introduction and conclusion.
        \grade 		There is extensive structure, leveraged to effectively highlight the argument and key takeaway points.
        \par 		All sentences and paragraphs are well constructed.
        \par 		There is a clear and well-constructed introduction and conclusion.
\end{markingrubric}

\end{document}