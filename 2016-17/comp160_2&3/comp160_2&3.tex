\documentclass{../fal_assignment}
\graphicspath{ {../} }

\usepackage{enumitem}
\usepackage[T1]{fontenc} % http://tex.stackexchange.com/a/17858
\usepackage{url}
\usepackage{todonotes}

\usepackage{listings}
\lstset{
    basicstyle=\ttfamily,
	frame=single,
	tabsize=4,
	showstringspaces=false,
	breaklines=true,
    language=C++
}

\title{Production Tasks \& Game Demo}
\author{Dr Michael Scott}
\module{COMP160}

\begin{document}

\maketitle

\begin{marginquote}
    ``It seems that perfection is attained not when there is nothing more to add,
        but when there is nothing more to remove.''
    
    --- Antoine de Saint-Exup\'ery
    
    \marginquoterule
    
    ``Good judgment comes from experience and experience comes from bad judgment!''
    
    --- Fred Brooks Jr
    
    \marginquoterule
    
    ``Debugging is twice as hard as writing the code in the first place.
     Therefore, if you write the code as cleverly as possible, you are, by definition, not smart enough to debug it.'' 
 
         --- Brian Kernighan
     
\end{marginquote}
\marginpicture{flavour_pic}{
    A poster and technical demo is a common format for presenting novel techniques at conferences such as \textit{ACM Multimedia}.
}

\section*{Introduction}

In this assignment, you will prepare a production prototype of your game. This version of your game will be written in C++ using the Simple DirectMedia Layer (SDL). You will continue to work in small groups (typically, 3--4 students).

The success of a digital game is not just determined by the fun of its design, but also by the soundness of its architecture. Put simply, games with serious bugs do not sell. As such, the value of software engineering principles cannot be emphasised enough! Reflecting upon these principles, through the lens of a development project that mirrors industry practice, is therefore important. Further to this, commercial considerations do not end. Engagement with the press will need to persist throughout production, so as to ensure market interest at launch.

This assignment is formed of several parts:

\begin{enumerate}[label=(\Alph*)]
    \item \textbf{Launch}, as a \textbf{group}, a GitHub Pages website which will:
    	\begin{enumerate}[label=\roman*.]
    		\item \textbf{describe} the game;
    		\item as well as \textbf{promote} the game's key features and aesthetics.
	\end{enumerate}
    \item \textbf{Prepare}, as a \textbf{group}, a plan which will:
    	\begin{enumerate}[label=\roman*.]
    	    	\item \textbf{identify} the skills and time available to complete the final prototype;
    		\item \textbf{describe} the key user stories that will comprise the final prototype;
    		\item and \textbf{highlight} the stories most critical to the technical demo.
	\end{enumerate}
    \item \textbf{Implement}, as a \textbf{group}, an initial production prototype in SDL which will:
    	\begin{enumerate}[label=\roman*.]
    		\item \textbf{illustrate} the core game mechanic;
    		\item and \textbf{apply} at least \textbf{three} non-trivial algorithms.
	\end{enumerate}
    \item \textbf{Write}, as a \textbf{group}, a press kit for the website which will:
    	\begin{enumerate}[label=\roman*.]
    		\item \textbf{generate} press interest in the game.
	\end{enumerate}	
    \item \textbf{Implement}, as a \textbf{group}, a final production prototype  in SDL which will:
    	\begin{enumerate}[label=\roman*.]
    		\item \textbf{revise} any issues raised by your tutor and/or your peers.
	\end{enumerate}
    \item \textbf{Prepare}, as an \textbf{individual}, an A3 research-style poster that:
    	\begin{enumerate}[label=\roman*.]
    		\item \textbf{outlines and discusses} the engineering of the final prototype;
    		\item and \textbf{describes one} algorithm that \textbf{you yourself} implemented. 
	\end{enumerate}
    \item \textbf{Present}, as a \textbf{group}, a `technical demo' which will:
    	\begin{enumerate}[label=\roman*.]
    		\item \textbf{clarify} the technical content of your own poster;
    		\item and \textbf{show} the final production prototype of the game.
	\end{enumerate}
\end{enumerate}
  
\subsection*{Assignment Setup}

This assignment is a \textbf{product development task}. Fork the GitHub repository at the following URL:

\indent \url{https://github.com/Falmouth-Games-Academy/comp160-game}

Use the existing directory structure and, as required, extend this structure with sub-directories. Ensure that you maintain the \texttt{readme.md} file.

Modify the \texttt{.gitignore} to the defaults for \textbf{Visual Studio}. Please, also ensure that you add editor-specific files and folders to \texttt{.gitignore}. 

\subsection*{Part A}

Part A consists of a \textbf{single formative submission}. This work is \textbf{collaborative} and will be assessed on a \textbf{threshold} basis. The following criteria are used to determine a pass or fail:

\begin{enumerate}[label=(\alph*)]
	\item The website is live;
	\item The game is adequetely described;
	\item The aesthetic is illustrated with at least four screenshots;
	\item The core game mechanics are illustrated with at least one video.
\end{enumerate}

To complete Part A, setup GitHub Pages and populate a single web page with content. Show it to your tutor in the first scheduled sprint review session.

You will receive immediate \textbf{informal feedback} from your \textbf{tutor}.

\subsection*{Part B}

Part B consists of a \textbf{single formative submission}. This work is \textbf{collaborative} and will be assessed on a \textbf{threshold} basis. The following criteria are used to determine a pass or fail:

\begin{enumerate}[label=(\alph*)]
	\item Only well-formed user stories are included;
	\item The user stories form a comprehensive plan;
	\item The plan has reasonable scope and is feasible.
\end{enumerate}

To complete Part B, setup and populate the team Trello board. Ensure that all members of the team are added to the board. Show it to your tutor in the scheduled CPD catch-up tutorial.

You will receive immediate \textbf{informal feedback} from your \textbf{tutor}.

\subsection*{Part C}

Part C is a \textbf{multiple formative submissions}. This work is \textbf{collaborative} and will be assessed on a \textbf{threshold} basis. The following criteria are used to determine a pass or fail:

\begin{enumerate}[label=(\alph*)]
	\item Submission is timely;
	\item Enough work is available to conduct a meaningful review;
	\item A broadly appropriate review of another team's work is submitted.
\end{enumerate}

To complete Part C, prepare a draft version of the production prototype. Ensure that the source code and related assets are pushed to GitHub and a pull request is made prior to \textit{each} scheduled sprint review session. Ensure that you attend \textit{each} scheduled sprint review session.

You will receive immediate \textbf{informal feedback} from your \textbf{tutor} and \textbf{peers}.

\subsection*{Part D}

Part D consists of a \textbf{single formative submission}. This work is \textbf{collaborative} and will be assessed on a \textbf{threshold} basis. The following criteria are used to determine a pass or fail:

\begin{enumerate}[label=(\alph*)]
	\item The press kit is available online;
	\item The development team are introduced and credited for their work;
	\item Key flavour text and unique selling points are included;
	\item The aesthetic is illustrated with at least four screenshots;
	\item The core game mechanics are illustrated with at least one video.
\end{enumerate}

To complete Part D, incorporate the press kit into a second web page. Show it to your tutor in \textit{each} scheduled sprint review session after the first.

You will receive immediate \textbf{informal feedback} from your \textbf{tutor}.


\subsection*{Part E}

Part E is a \textbf{single summative submission}. This work is \textbf{collaborative} and will be assessed on a \textbf{criterion-referenced} basis. Please refer to the marking rubric at the end of the brief for details on the criteria.

To complete Part E, revise the production prototype based on the feedback you have received and tidy up any incomplete features. Please also ensure that you include appropriate screen-shots of the Trello board in a separate folder. Then, upload the source code to the LearningSpace. Please note, the LearningSpace will only accept a single \texttt{.zip} file. The recommended way of generating the \texttt{.zip} file is using the \textit{[Download Zip]} button on the GitHub website.

You will receive \textbf{formal feedback} three weeks after the final deadline.

\subsection*{Part F}

Part F is a \textbf{single summative submission}. This work is \textbf{individual} and will be assessed on a \textbf{threshold} basis. The following criteria are used to determine a pass or fail:

\begin{enumerate}[label=(\alph*)]
	\item Submission is timely;
	\item At least \textbf{one} non-trivial algorithm is described
	\item The algorithm was implemented by the person submitting the work
	\item The description of the algorithm is detailed and accurate
	\item There is sufficient detail to show how the algorithm fits into the overall architecture of the game
	\item There is little to no overlap with other members of the team
\end{enumerate}

To complete Part F, prepare the poster using any word processing and/or presentation tool. Then, upload the relevant files to the LearningSpace. Please note, the LearningSpace will only accept a single \texttt{.zip} file.

You will receive \textbf{informal feedback} from your tutor three days prior to Part E.

\subsection*{Part G}

Part G is a \textbf{single summative submission}. This work is \textbf{collaborative} and will be assessed on a \textbf{criterion-referenced} basis. Please refer to the marking rubric at the end of the brief for details on the criteria.

To complete Part G, look over your notes to prepare yourself to answer questions. Ensure that you are comfortable with the content of your poster and discuss any concerns with your tutor. Then, attend the scheduled technical demo session. Please ensure that you print your poster ahead of time and bring it with you. It should be printed on A3 paper and in portrait. This is your individual responsibility. Please, also ensure that one member of the team is able to setup a laptop with the production prototype ahead of time.

You will receive \textbf{formal feedback} three weeks after the final deadline.

\section*{Additional Guidance}

Carefully select the algorithm that you will take ownership of and implement. This algorithm will need to interface with other game components, and therefore affected by work of your peers. Aim for high cohesion and low coupling! It is also important that the main requirements are firmly specified and are not too broad. This will ensure that there is little overlap with the work of your peers and help ensure that you do not overburden yourself with too much work. 

Please remember to commit frequently and to push your source code and related assets to the GitHub repository. This will make it easier for you to maintain a backup of your work. It will also help you to measure your productivity. GitHub should be an essential part of your workflow, not merely a place to submit your work for feedback. You will be expected to maintain an archive of playable builds to demo your work at any time.

Poor planning and poor time management can have a significant impact on this assignment. As some of you may have already discovered, programming is quite unlike other subjects in that it cannot be ``crammed'' into a last minute deluge. Sustain a steady pace across the duration of the course. Do a little programming every day, if you can!

For the most part, your work will be marked as a group effort.
However we want to avoid the situation where students try to ``coast'' through the assignment
on their fellow group members' work,
and equally the situation where one member of the group takes the lion's share of the work
and prevents the others from contributing effectively.
Marks will be weighted by a multiplier for \textbf{individual contribution},
which aims to penalise both of these behaviours.
We assess this by several means, including but not limited to: sprint reviews; individual vivas; feedback from your peers;
attribution in the source code; and GitHub commit logs.
Any student who has contributed their \textit{fair share} of effort to the project will receive a fair \% for their effort,
so any student who is putting in the appropriate level of effort has no need to worry.
Note that effort is not the same as productivity.

Your code will be assessed on \textbf{functional coherence}:
how well the finished game corresponds to the user stories,
and whether the game has any obvious bugs.
Correspondence to user stories runs both ways:
implementing features that were not present in the design (``feature creep'')
is just as bad as neglecting to implement features.

Your code will also be assessed on \textbf{sophistication}.
To succeed on a project of this size and complexity,
you will need to make use of appropriate algorithms, data structures, libraries, and object oriented programming concepts.
Appropriateness to the task at hand is key:
you will \textbf{not} receive credit for complexity  
where something simpler would have sufficed.

\textbf{Maintainability} is important in all programming projects,
but doubly so when working in a team.
Use \textbf{comments} liberally to improve code comprehension,
and carefully choose the \textbf{names} for your files, classes, functions and variables.
Use a well-established commenting convention
for \textbf{high-level documentation}.
The open-source tool Doxygen supports several such conventions.
Also ensure that all code corresponds to a sensible and consistent \textbf{formatting style}:
indentation, whitespace, placement of curly braces, etc.
Hard-coded \textbf{literals} (numbers and strings) within the source should be avoided,
with values instead defined as constants together in a single place.
Consider allowing some literal values, where appropriate, to be ``tinkered'' without changing the source code,
e.g.\ by defining them in an external file read by the game on startup.

\section*{FAQ}

\begin{itemize}
	\item 	\textbf{What is the deadline for this assignment?} \\ 
    		Falmouth University policy states that deadlines must only be specified on the MyFalmouth system.
    		
	\item 	\textbf{What should I do to seek help?} \\ 
    		You can email your tutor for informal clarifications. For informal feedback, make a pull request on GitHub. 
    		
    	\item 	\textbf{Is this a mistake?} \\ 	
    		If you have discovered an issue with the brief itself, the source files are available at: \\
    		\url{https://github.com/Falmouth-Games-Academy/bsc-assignment-briefs/issues}.\\
    		 Please create an issue and comment accordingly.
\end{itemize}

\section*{Additional Resources}

\begin{itemize}
    \item Stroustrup, B. (2014) Programming: Principles and Practice using C++. Second Edition. Addison Wesley.
    \item Spraul, V. (2012) Think Like a Programmer. O'Reilly Media.
    \item Keith, C. (2010) Agile Game Development with Scrum. Pearson Education.
    \item Sims, C. and Johnson, H.L. (2012) SCRUM: A Breathtakingly Brief and Agile Introduction. Dymaxicon.
    \item \url{https://www.mountaingoatsoftware.com/agile/user-stories}
    \item \url{https://literateprogramming.com}
    \item \url{http://gameprogrammingpatterns.com/}
    \item \url{https://blog.codinghorror.com/}
    \item \url{https://git-scm.com/book/en/v2}
    \item \url{http://martinfowler.com/articles/continuousIntegration.html}
    \item \url{https://travis-ci.org}
    \item \url{https://doxygen.org}
    \item \url{http://dopresskit.com/}
    \item \url{http://www.binpress.com/blog/2015/04/06/}\\ \url{guide-launching-indie-games-part-three-getting-press/}
    \item \url{https://c9.io}
    \item \url{http://www.gamasutra.com/blogs/RogerPaffrath/20131115/204871/What_NOT_to_do_when_starting_as_an_indie_game_developer.php}
\end{itemize}

\rubrictitle{Marking Rubric (Production Prototype)}
\rubrichead{Criteria marked with a $\dagger$ are weighted by individual contribution to a shared deliverable. All other criteria are individual.}
\begin{markingrubric}
    \firstcriterion{Sprint Reviews}{40\%}
        \gradespan{1}{\fail The student fails to participate in at least one sprint review, or has not contributed to promotional material.}
        \gradespan{5}{The student participates in all sprint reviews.
             \par All sprint reviews result in a playable build.}
%
    \criterion{Appropriateness of User Stories and Sprint Plans}{5\% $\dagger$}
        \grade\fail No user stories and/or sprint plans are provided.
        \grade Few user stories are distinguishable and easily measured.
            \par Sprint plans provide little support for the project.
        \grade Some user stories are distinguishable and easily measured.
            \par Sprint plans provide some support for the project.
        \grade Most user stories are distinguishable and easily measured.
            \par User stories correspond to the game design.
            \par Sprint plans provide much support for the project.
        \grade Nearly all user stories are distinguishable and easily measured.
            \par User stories clearly correspond to the game design.
            \par Sprint plans provide considerable support for the project.
        \grade All user stories are distinguishable and easily measured.
            \par User stories clearly and comprehensively correspond to the game design.
            \par Sprint plans provide significant support for the project.
%
    \criterion{Functional Coherence}{5\% $\dagger$}
        \grade\fail No gameplay elements have been implemented and/or the code fails to compile or run.
        \grade Few gameplay elements have been implemented.
            \par There are many obvious and serious bugs.
        \grade Some gameplay elements have been implemented.
            \par There are some obvious bugs.
        \grade Many gameplay elements have been implemented.
            \par There is some evidence of feature creep.
            \par There are few obvious bugs.
        \grade Almost all gameplay elements have been implemented.
            \par There is little evidence of feature creep.
            \par There are some minor bugs.
        \grade All gameplay elements have been implemented.
            \par There is no evidence of feature creep.
            \par Bugs, if any, are purely cosmetic and/or superficial.
%            
    \criterion{Promotional Coherence}{4\% $\dagger$}
        \grade\fail There is no website or press kit, or the game does not resemble either.
        \grade Few promised gameplay elements are in-game.
            \par The website and press kit have little clarity.
        \grade Some promised gameplay elements are in-game.
            \par The website and press kit have some clarity.
        \grade Many promised gameplay elements are in-game.
            \par The website and press kit have much clarity.
            \par Promotional material evokes some excitement.
        \grade Almost all promised gameplay elements are in-game.
            \par The website and press kit have considerable clarity.
            \par Promotional material evokes much excitement.
        \grade All promised gameplay elements are in-gamed.
            \par The website and press kit have significant clarity.
            \par Promotional material evokes considerable excitement.
%
    \criterion{Sophistication}{10\% $\dagger$}
        \grade\fail No insight into the appropriate use of programming constructs is evident from the source code.
            \par No attempt to structure the program is evident (e.g. one monolithic source file).
        \grade Little insight into the appropriate use of programming constructs is evident from the source code.
            \par The program structure is poor.
        \grade Some insight into the appropriate use of programming constructs is evident from the source code.
            \par The program structure is adequate.
        \grade Much insight into the appropriate use of programming constructs is evident from the source code.
            \par The program structure is appropriate.
        \grade Considerable insight into the appropriate use of programming constructs is evident from the source code.
            \par The program structure is effective. There is high cohesion and low coupling.
        \grade Significant insight into the appropriate use of programming constructs is evident from the source code.
            \par The program structure is very effective. There is high cohesion and low coupling.
%
    \criterion{Maintainability}{20\% $\dagger$}
        \grade\fail There are no comments, or comments are misleading.
            \par Most variable names are unclear or inappropriate.
            \par Code formatting hinders readability.
        \grade The code is only sporadically commented, or comments are unclear.
            \par Few identifier names are clear or inappropriate.
        \grade The code is well commented.
            \par Some identifier names are descriptive and appropriate.
            \par An attempt has been made to adhere to a consistent formatting style.
             \par There is little obvious duplication of code or of literal values.           
        \grade The code is reasonably well commented.
            \par Most identifier names are descriptive and appropriate.
            \par Most code adheres to the Google C++ formatting style.
             \par There is almost no obvious duplication of code or of literal values.   
        \grade The code is reasonably well commented, with Doxygen-compatible module documentation.
            \par Almost all identifier names are descriptive and appropriate.
            \par Almost all code adheres to the Google C++ formatting style.
             \par There is no obvious duplication of code or of literal values. Some literal values can be easily ``tinkered''. 
        \grade The code is very well commented, with comprehensive Doxygen-compatible module documentation.
            \par All identifier names are descriptive and appropriate.
            \par All code adheres to the Google C++ formatting style.
             \par There is no obvious duplication of code or of literal values. Most literal values are, where appropriate, easily ``tinkered'' outside of the source.  
%
    \criterion{Portability and Navigability}{1\% $\dagger$}
        \grade\fail Game will not execute at all on another machine for reasons related to code portability which cannot be fixed easily due to its poor structure.
            \par The provided template has not been followed.
        \grade There were challenges executing the game, but these were resolvable.
            \par The directory structure inside the submitted zip file is unclear.
            \par The provided template has not been followed.
        \grade Several portability issues are present.
            \par The directory structure inside the submitted zip file is somewhat confusing.
            \par The provided template has mostly been followed.
        \grade Some portability issues are present.
            \par The directory structure inside the submitted zip file is adequate.
            \par The provided template has been followed.
        \grade Few portability issues are present.
            \par The directory structure inside the submitted zip file is mostly sensible.
            \par The provided template has been followed.
        \grade Almost no portability issues are present.
            \par The directory structure inside the submitted zip file is sensible.
            \par The provided template has been followed.
%
    \criterion{Team Cohesion}{5\% $\dagger$}
        \grade\fail The group's professional conduct has been unacceptable,
            and/or the group has failed to function at all as a team.
            \par Agile working practices have not been used.
        \grade The group has demonstrated little professionalism.
            \par Agile working practices have provided little support for the project.
        \grade The group has demonstrated some professionalism,
            functioning adequately as a team.
            \par Agile working practices have provided some support for the project.
        \grade The group has demonstrated much professionalism,
            functioning somewhat effectively as a team.
            \par Agile working practices have provided much support for the project.
        \grade The group has demonstrated considerable professionalism,
            functioning effectively as a cohesive team.
            \par Agile working practices have provided considerable support for the project.
            \par There is evidence of some use of Travis CI to support a continuous integration approach.
        \grade The group has demonstrated significant professionalism,
            functioning highly effectively as a cohesive team.
            \par Agile working practices have provided significant support for the project.
            \par Travis CI has been used to effectively support a continuous integration approach.
%
    \criterion{Use of Version Control}{10\%}
        \grade\fail GitHub has not been used.
        \grade Material has been checked into GitHub less frequently than once per sprint.
            \par All code has been checked into the Master branch.
        \grade Code has been checked into GitHub at least once per sprint.
            \par An attempt has been made to use branches.
        \grade Code has been checked into GitHub several times per sprint.
            \par Commit messages are clear, concise and relevant.
            \par Branches are used sensibly.
        \grade Code has been checked into GitHub several times per sprint.
            \par Commit messages are clear, concise and relevant.
            \par Branches are used somewhat effectively.
            \par There is evidence of engagement with peers (e.g.\ code review).
        \grade Code has been checked into GitHub several times per sprint.
            \par Commit messages are clear, concise and relevant.
            \par Branches are used effectively.
            \par There is significant evidence of engagement with peers (e.g.\ code review).
%
    \criterion{Individual Contribution}{Multiplier for criteria marked $\dagger$}
        \gradespan{5}{\fail The student has failed to contribute their ``fair share'' to the project,
            or has actively prevented others from doing so.}
        \grade The student has contributed their ``fair share'' to the project,
            and has facilitated others in doing so.
\end{markingrubric}

\rubrictitle{Marking Rubric (Technical Demo)}
\rubrichead{Criteria marked with a $\ddagger$ are shared by the group. All other criteria are individual.}
\begin{markingrubric}
    \firstcriterion{Basic Competency Threshold}{40\%}
        \gradespan{1}{\fail No individual poster and/or production prototype is delivered, or either are inappropriate.}
        \gradespan{5}{A broadly appropriate individual poster and group tech demo are delivered in a timely fashion. There is no evidence of academic misconduct.}
%
    \criterion{Communication Skills}{20\% $\ddagger$}
        \grade\fail Delivered with no enthusiasm. 
           \par The technology behind the game has not been articulated with clarity.
        \grade Delivered with little enthusiasm. 
            \par Little connection with the audience.
            \par The technology behind the game has been articulated with little clarity.
        \grade Delivered with some enthusiasm, conveying technical confidence. 
            \par Some connection with the audience.
            \par The technology behind the game has been articulated with some clarity.
        \grade Delivered with much enthusiasm, conveying technical confidence. 
            \par Much connection with the audience.
            \par The technology behind the game has been articulated with much clarity.
        \grade Delivered with considerable enthusiasm,conveying technical confidence. 
            \par Considerable connection with the audience.
            \par The technology behind the game has been articulated with considerable clarity.
        \grade Delivered with significant enthusiasm, conveying technical confidence and passion.
            \par Significant connection with the audience.
            \par The technology behind the game has been articulated with significant clarity.
%
    \criterion{Poster Quality}{10\%}
        \grade\fail There is no poster or it does not describe the engineering of the software. 
        \grade The engineering of the software (e.g., class designs) is described with little adequacy.
        \grade The engineering of the software (e.g., class designs) is described with some adequacy.
        \grade The engineering of the software (e.g., class designs) is concisely described with much adequacy.
            \par The use of UML diagrams and source code excerpts is somewhat effective.
        \grade The engineering of the software (e.g., class designs) is concisely described with considerable adequacy.
            \par The use of UML diagrams and source code excerpts is quite effective.
        \grade The engineering of the software (e.g., class designs) is concisely described with significant adequacy.
            \par The use of UML diagrams and source code excerpts is very effective.
%
    \criterion{Technical Insight}{10\%}
        \grade\fail No individual algorithm has been contributed or it is trivial.
        \grade Little insight into the technical qualities of the individual algorithm.
            \par Little ability to explain how the algorithm fits into the game's components and architecture.
        \grade Some insight into the technical qualities of the individual algorithm.
            \par Some ability to explain how the algorithm fits into the game's components and architecture.
        \grade Much insight into the technical qualities of the individual algorithm.
            \par Much ability to explain how the algorithm fits into the game's components and architecture.
            \par The relevance of the contribution is justified.
        \grade Considerable insight into the technical qualities of the individual algorithm.
            \par Considerable ability to explain how the algorithm fits into the game's components and architecture.
            \par The relevance and value of the individual algorithm are justified.
        \grade Significant insight into the technical qualities of the individual algorithm.
            \par Significant ability to explain how the algorithm fits into the game's components and architecture.
            \par The relevance and value of the  individual algorithm are justified.
            \par The  individual algorithm is somewhat important to the design of the game.
%
    \criterion{Demo Quality}{20\% $\ddagger$}
        \grade\fail There is no demo, or it is non-functional.
        \grade The demo demonstrates some mechanics and interfaces.
        \grade The demo demonstrates the key mechanics and interfaces.
        \grade The demo demonstrates the core game mechanic.
            \par Although there may be a backup video, at least some aspect of the demo is live using the production prototype.
        \grade The demo demonstrates the core game mechanic.
            \par Although there may be a backup video, much of the demo is live using the production prototype.
        \grade The demo demonstrates the core game mechanic.
            \par Although there may be a backup video, a considerable part of the demo is live using the production prototype.
            \par There is some innovation in terms of technology in the demo.
\end{markingrubric}

\end{document}
