\documentclass{../fal_assignment}
\graphicspath{ {../} }

\usepackage{enumitem}
\setlist{nosep} % Make enumerate / itemize lists more closely spaced
\usepackage[T1]{fontenc} % http://tex.stackexchange.com/a/17858
\usepackage{url}
\usepackage{todonotes}

\title{Hacking }
\author{Alcwyn Parker}

\begin{document}

\maketitle

\section*{Introduction}

\begin{marginquote}
Hacker definition: ``A person who enjoys exploring the details of programmable systems and stretching their capabilities, as opposed to most users, who prefer to learn only the minimum necessary.''

--- Jargon File

\end{marginquote}
\marginpicture{flavour_pic}{
    A twitterbot sorting a picture by its pixels.
}
In this assignment, you will work as an individual, to design and prototype a novel game controller. Your prototype should function as an input device for either one of the games being developed by students on the BA Digital Games course; or the game you developed in COMP150 last semester. Your prototype should use a hardware platform such as Arduino or Raspberry Pi etc, to convert user actions into game inputs. 

Computing for games embraces the core values of the hacker movement.  Experimentation, ingenuity and creativity are at the heart of everything we do. Custom game controllers are the perfect place to flex your creative flare and problem solving skills, whilst gaining invaluable experience working with hardware. 

This assignment is formed of several components:

\begin{enumerate}[label=(\Alph*)]
    \item \textbf{Write}, a proposal for a novel game controller that:
    	\begin{enumerate}[label=\roman*.]
		\item \textbf{assesses} the game controller market;
    		\item \textbf{states} which game is the basis for your interface;
    		\item \textbf{justifies} your choice of game;
		\item \textbf{outlines} your design in detail;
	\end{enumerate}
    \item \textbf{Implement}, three physical prototypes which:
    	\begin{enumerate}[label=\roman*.]
		\item \textbf{improves} the design over each iteration;
	\end{enumerate}
    \item \textbf{implement}, a final physical prototype which:
    	\begin{enumerate}[label=\roman*.]
		\item \textbf{revises} the design based on the results of a heuristic analysis.
	\end{enumerate}
    \item \textbf{Present} a practical demo of the game controller to your tutor that will:
    	\begin{enumerate}[label=\roman*.]
    		\item \textbf{demonstrate} your academic integrity;
    		\item as well as \textbf{demonstrate} your \textbf{individual} programming knowledge \textbf{and} communication skills.
	\end{enumerate}
\end{enumerate}

\subsection*{Assignment Setup}

Fork the GitHub repository at:

\indent \url{https://github.com/Falmouth-Games-Academy/comp140-hardware }

Use the existing directory structure and, as required, extend this structure with sub-directories. Ensure that you maintain the \texttt{readme.md} file.

Modify the \texttt{.gitignore} to the defaults for \textbf{Python}. Please, also ensure that you add editor-specific files and folders to \texttt{.gitignore}. 

\subsection*{Part A}

Part A consists of a \textbf{single formative submission}. This work is \textbf{collaborative} and will be assessed on a \textbf{threshold} basis. The following criteria are used to determine a pass or fail:

\begin{enumerate}[label=(\alph*)]
	\item research is thorough;
	\item concept is appropriate and distinctive;
	\item approach is considered and justified.
\end{enumerate}

To complete part A, on GitHub, edit the \textbf{readme.md} file to contain a description of your proposed game controller. Your proposal should include details of any background research you have done to assess commercial viability. On Trello, create a task board that defines the key requirements (in terms of components and user stories) of the controller. 

\textbf{Formative submission:} Arrange a meeting with your tutor to discuss your concept and task board. 

\subsection*{Part B}

Part B is a \textbf{continual formative assessment}. This is individual work will be assessed on a threshold basis. The following criteria are used to determine a pass or fail:

\begin{enumerate}[label=(\alph*)]
	\item version Control used effectively;
	\item Sufficient progress each week;
	\item Reflective practice.
\end{enumerate}

You will build your \textbf{prototype} controller over a period of \textbf{4 weeks} utilising a fast, iterative development process. Each week should see a vast improvement in the quality of design and development working towards a shippable product to demo in the fourth week. that is, a prototype which does not have any major flaws or half-finished features that prevent it from being tested, and that can be used (even if lacking some features) as a controller in the game. 

Use the forked repository to store any digital artefacts (including but not limited to design sketches, photographs, art assets, source code, electronic circuit designs). 

Feedback will be given on a \textbf{week-by-week} basis. 

\subsection*{Part C}

Part C is a single formative submission. This work is collaborative and will be assesses on a threshold basis. The following criteria are used to determine a pass or fail:
\begin{enumerate}[label=(\alph*)]
	\item Submission is timely;
	\item Enough work is available to conduct a meaningful review;
	\item A broadly appropriate review of a peer?s work is submitted.
\end{enumerate}

To complete Part C, prepare you game controller for review. It must be fully functional and integrated into your chosen game.  Ensure that the source code and related assets are pushed to GitHub and a pull request is made prior to the scheduled peer-review session.

\subsection*{Part D}

Part D is a \textbf{single summative} submission. This work is individual and will be assessed on a threshold basis. The following criteria are used to determine a pass or fail: 

\begin{enumerate}[label=(\alph*)]
	\item enough work is available to hold a meaningful discussion; 
	\item Clear evidence of programming knowledge and communication skills; 
	\item No breaches of academic integrity. 
\end{enumerate}

To complete Part D, prepare a practical demonstration of the game controller. Ensure that the source code and related assets are pushed to GitHub and a pull request is made prior to the scheduled viva session. Then, attend the scheduled viva session. 

You will receive \textbf{immediate informal} feedback from your tutor.

\section*{Additional Guidance}
Falmouth University is nationally and internationally renowned as an arts institution. Despite the fact that you are studying for a Bachelor of Science degree in a technical discipline, you are still expected to strive for the same level of innovation and creative flair as your fellow students in other departments. All assignments on this course involve a mix of technical and creative activities; this assignment is more heavily weighted towards the creative than the assignments you have completed thus far. On this assignment, a competent execution of an unimaginative idea is unlikely to achieve higher than a C grade overall, as opposed to an imperfect execution of a unique and ambitious concept ? bear this in mind when working on your design. One approach to promoting creativity is divergent thinking: generation of ideas by exploring many possible solutions. Often the most interesting ideas are subversive: they deliberately go against the accepted or most obvious solution 

The history of video games is littered with failed peripherals which consumers simply did not want, which were perceived as expensive gimmicks rather than legitimate enhancements to gameplay. Your creativity should be balanced by commercial awareness: your design should be informed by your research into products that have succeeded and failed in the past, and what underexploited niches exist in the present. An A? project would be a highly divergent idea, but one that has clear commercial viability. Do not be too discouraged if you fall short of this: this is a tall order even for the professionals! 

We have given you some of the materials you need: an Arduino and other useful components. You will need to add your own materials to produce a functional physical prototype. A ?Blue Peter? style prototype made from household items is fine, as is something made out of modeling clay, construction toys etc. However you should still choose your materials carefully, as overly flimsy construction may lose you marks on the functionality criterion. 

You may also wish to connect electronic components such as LEDs, buzzers, photoresistors etc to the Arduino, or even use a different, more flexible hard- ware platform such as RaspberryPi. However you are discouraged from spending large sums of money on extra hardware, and doing so is not required to achieve a high mark. If you choose to go down this route, it is possible to purchase a RaspberryPi and other useful peripheral online for around the price of a textbook (up to \pounds40). 

You should aim to demonstrate a high level of sophistication in the technical execution of your prototype. An important part of sophistication is having the insight to choose the right tool for the job: if a simpler technique fulfils all the requirements, use it. The use of unnecessarily complicated techniques, serving only to showcase one?s own cleverness, is a dangerous habit for a software developer. 
The sole purpose of the video demonstration is to aid moderators and external examiners, who are not present for the demo session, in assessing your work. Your video does not need to be entertaining or highly polished: a smartphone or webcam video of you or someone else using the controller is sufficient. 

\section*{FAQ}

\begin{itemize}
	\item 	\textbf{What is the deadline for this assignment?} \\ 
    		Falmouth University policy states that deadlines must only be specified on LearningSpace. Please examine the assignment area where you located this document.
    		
	\item 	\textbf{What should I do to seek help?} \\ 
    		You can email your tutor for informal clarifications. For informal feedback, make a pull request on GitHub. 
    		
    	\item 	\textbf{Is this a mistake?} \\ 	
    		If you have discovered an issue with the brief itself, the source files are available at: \\
    		\url{https://github.com/Falmouth-Games-Academy/bsc-assignment-briefs}.\\
    		 Please raise an issue and comment accordingly.
\end{itemize}

\section*{Additional Resources}

\begin{itemize}
     \item Wilkinson, K. and Petrich, M. (2014) The Art of Tinkering: Meet 150 Markers Working at the Intersection of Art, Science \& Technology. Weldon Owen: London.
    \item Alicia Gibb. Building Open Source Hardware: DIY Manufacturing for Hackers and Makers. Addison Wesley, 2014. 
    \item Jeremy Blum. Exploring Arduino: Tools and Techniques for Engineering Wizardry. John Wiley, 2013. 
    \item Kelly, K. (2014) Cool Tools: A Catalogue of Possibilities. Cool Tools.
    \item Hatch, M. (2013) The Maker Movement Manifesto: Rules for Innovation in the New World of Creators, Hackers, and Tinkerers. McGraw Hill: New York.
    \item https://www.sitepoint.com/heuristic-evaluation-guide/
    \item https://www.usability.gov/how-to-and-tools/methods/heuristic-evaluation.html
    
\end{itemize}

\begin{markingrubric}
    \firstcriterion{Iterative development process}{5\%}
        \gradespan{2}{\fail There is little or no evidence of an iterative development process and no improvement over time in regards to the quality of the design and build of the prototype. }
        \gradespan{2}{A `potentially shippable' prototype is produced at the end of development period.
            \par There is evidence of a `reasonable' iterative development process but the prototype suffers from 'lock in' in regards to the original concept.}
        \gradespan{2}{A `potentially shippable' prototype is produced at the end of the development period.
            \par The project has benefitted from an iterative development process and many improvement have been made to the original concept}    		\criterion{Design of the solution}{15\%}
        \grade\fail No user stories are provided, or the design does not correspond to the user stories.
        \grade Some user stories are distinguishable and easily measured.
            \par The correspondence between design and user stories is tenuous.
        \grade Little user stories are distinguishable and easily measured.
            \par The design somewhat corresponds to the user stories.
        \grade Most user stories are distinguishable and easily measured.
            \par The design corresponds to the user stories.
        \grade Nearly all user stories are distinguishable and easily measured.
            \par The design clearly corresponds to the user stories.
        \grade All user stories are distinguishable and easily measured.
            \par The design clearly and comprehensively corresponds to the user stories.
    \criterion{Commercial awareness}{10\%}
        \grade\fail No commercial awareness is demonstrated.
        \grade Some commercial awareness is demonstrated.
            \par There is no evidence of market research.
        \grade Little commercial awareness is demonstrated.
            \par Market research is present, but with significant gaps.
        \grade Much commercial awareness is demonstrated.
            \par Market research is extensive, but with some gaps.
        \grade Considerable commercial awareness is demonstrated.
            \par Market research is comprehensive.
        \grade Significant commercial awareness is demonstrated.
            \par Market research is comprehensive and insightful.
    \criterion{Innovation and creative flair}{5\%}
        \grade\fail No evidence of innovation and/or creativity.
        \grade Some evidence of emerging innovation and/or creativity.
            \par The solution is purely derivative of existing products.
            \par There is no evidence of divergent thinking.
        \grade Little evidence of emerging innovation and/or creativity.
            \par The solution is mostly derivative, with some attempts at innovation.
            \par There is evidence of an attempt at divergent thinking.
        \grade Much evidence of emerging innovation and/or creativity.
            \par The solution is an interesting and somewhat innovative product.
            \par There is some evidence of divergent thinking.
        \grade Considerable evidence of mastery of innovative and creative practice.
            \par The solution is a novel and innovative product.
            \par There is much evidence of divergent thinking.
        \grade Significant evidence of mastery of innovative and creative practice.
            \par The solution is a unique and innovative product.
            \par There is significant evidence of divergent thinking.
    \criterion{Functionality of physical prototype}{15\%}
        \grade\fail A physical prototype is not produced, or the prototype is completely non-functional.
        \grade The physical prototype has no functionality.
            \par There are serious technical and/or constructional flaws.
        \grade The physical prototype has some functionality.
            \par There are obvious technical and/or constructional flaws.
        \grade The physical prototype has much functionality.
            \par There are minor technical and/or constructional flaws.
        \grade The physical prototype has considerable functionality.
            \par There are superficial technical and/or constructional flaws.
        \grade The physical prototype has significant functionality.
            \par The technical execution and physical construction are flawless.
    \criterion{Sophistication: \par Software \par Electronics \par Physical construction}{20\%}
        \grade\fail The solution lacks even a basic level of sophistication in any of the three areas.
        \grade The solution evidences some sophistication in one or more of the three areas.
            \par Some insight has been demonstrated in any area.
        \grade The solution evidences little sophistication in one or more of the three areas.
            \par Little insight has been demonstrated in at least one of the areas.
        \grade The solution evidences much sophistication in two or more of the three areas.
            \par Much insight has been demonstrated in at least one of the areas.
        \grade The solution evidences considerable sophistication in all three areas.
            \par Considerable insight has been demonstrated in at least two of these areas.
        \grade The solution evidences significant sophistication in all three areas..
            \par Significant insight has been demonstrated in all three areas.
    \criterion{Use of version control -- use template}{5\%}
        \grade\fail GitHub has not been used.
        \grade Some material has been checked into GitHub.
            \par Mostly before the deadline.
        \grade Little material has been checked into GitHub.
        	   \par  At least once per sprint.
        \grade Much material has been checked into GitHub.
        	   \par Several times per sprint.
        \grade Considerable material has been checked into GitHub.
        	    \par Several times per sprint.
            \par Commit messages are clear, concise and relevant.
        \grade Significant material has been checked into GitHub.
        	    \par Several times per sprint.
            \par Commit messages are clear, concise and relevant.
            \par There is evidence of engagement with peers (e.g.\ voluntary code review).
    \criterion{Basic Competency Threshold}{40\%}
        \gradespan{1}{\fail merge with iterative. At least one part is missing or is unsatisfactory. }
        \gradespan{5}{Submission is timely.
        	\par Enough work is available to hold a meaningful discussion.
	\par Clear evidence of programming knowledge and communication skills.
	\par Clear evidence of reflection on own performance and contribution.
	\par Only constructive criticism of pair-programming partner is raised.
	\par No breaches of academic integrity.}
	
\end{markingrubric}

\end{document}