\documentclass{../fal_assignment}
\graphicspath{ {../} }

\usepackage{enumitem}
\setlist{nosep} % Make enumerate / itemize lists more closely spaced
\usepackage[T1]{fontenc} % http://tex.stackexchange.com/a/17858
\usepackage{url}
\usepackage{todonotes}

\title{Software Engineering Essay}
\author{Dr Michael Scott}

\begin{document}

\maketitle

\section*{Introduction}

\begin{marginquote}
  ``Program testing can be used to show the presence of bugs, but never to show their absence!''
  
   --- Edsger Dijkstra
   
    \marginquoterule
    
    ``Let us change our traditional attitude to the construction of programs. Instead of imagining that our main task is to instruct a computer what to to, let us concentrate rather on explaining to human beings what we want a computer to do.''
    
    --- Donald Knuth
    
    \marginquoterule
    
    ``A good way to stay flexible is to write less code.''
    
    --- Andy Hunt \& Dave Thomas
      
\end{marginquote}
\marginpicture{flavour_pic}{
    Software Engineering is the systematic application of scientific and technological knowledge, methods, and experience to software development practice.
}

In this assignment, you will research software engineering principles to bring an academic perspective to your working practice. Specifically, to explore: 

\begin{enumerate}[label=(\roman*)]
    \item Which engineering challenges are prominent in game development;
    \item and how software engineering principles are applied to digital games.
\end{enumerate}

Working practices are important to employers in the games industry. Problems arising from poor software design and ineffective quality assurance practices are largely avoidable. As such, the key principles of software engineering are widely used in the games industry. They are essential to learn. Additionally, research skills will help you with your professional development. Most critically, moving beyond textbooks and websites to the academic literature, which will help you to keep your skills up to date in the future.

This assignment is formed of several parts:

\begin{enumerate}[label=(\Alph*)]
    \item \textbf{Write} a 200-word proposal \textbf{with} references which must:
    	\begin{enumerate}[label=\roman*.]
    		\item \textbf{state} a problem associated with the given challenge area;
    		\item \textbf{describe} the way in which you intend to address the question;
    		\item and then \textbf{list} at least \textbf{10} appropriate academic references to use to support your research.
	\end{enumerate}
    \item \textbf{Present}, as a \textbf{group}, a 15-minute summary of your research that will:
    	\begin{enumerate}[label=\roman*.]
    		\item \textbf{clarify} each person's challenge area \textbf{and} final research question;
    		\item \textbf{debate} the most important principles based on your own research;
    		\item and \textbf{discuss how} these principles apply to working practice.
	\end{enumerate}
    \item \textbf{Write} a draft 1000-word essay which will:
    	\begin{enumerate}[label=\roman*.]
    		\item  \textbf{address} the research queston;
    		\item and \textbf{analyse} which software engineering principles are the most important for your research area.
	\end{enumerate}
    \item \textbf{Write} a final 1000-word essay which will:
    	\begin{enumerate}[label=\roman*.]
    		\item \textbf{revise} any issues raised by your tutor and/or your peers.
	\end{enumerate}
\end{enumerate}

\todo[inline]{\textbf{Note:} All research questions must be distinctive. Members of the same development group must \textbf{not} target the same research question.}

\subsection*{Assignment Setup}

This assignment is an \textbf{academic writing task}. Fork the GitHub repository at the following URL:

\indent \url{https://github.com/Falmouth-Games-Academy/comp160-engineering}

Use the existing directory structure and, as required, extend this structure with sub-directories. Ensure that you maintain the \texttt{readme.md} file.

Modify the \texttt{.gitignore} to the defaults for \textbf{TeX}. Please, also ensure that you add editor-specific files and folders to \texttt{.gitignore}. 

\subsection*{Part A}

Part A consists of a \textbf{single formative submission}. This work is \textbf{individual} and will be assessed on a \textbf{threshold} basis. The following criteria are used to determine a pass or fail:

\begin{enumerate}[label=(\alph*)]
	\item Submission is timely;
	\item Research question is appropriate and distinctive;
	\item At least six academic peer-reviewed sources are cited.
\end{enumerate}

To complete Part A, review the challenge area announced in class. Write your proposal in the \texttt{readme.md} document and then prepare the reference list using a \texttt{*.bib} file.  Show these to your tutor.  If acceptable, this will be signed-off. 

You will receive immediate \textbf{informal feedback}.

\subsection*{Part B}

Part B is a \textbf{single formative submission}. This work is \textbf{collaborative} and will be assessed on a \textbf{threshold} basis. The following criteria are used to determine a pass or fail:

\begin{enumerate}[label=(\alph*)]
	\item Research questions are adequetely addressed;
	\item Some evidence of academic rigor;
	\item Some insight into the relationship between theory and practice.
\end{enumerate}

To complete Part B, prepare a presentation, and practice your debate and discussion. Prepare your slideshow collaboratively in TeX. Use the combined reference list of the group to broadly discuss each individual research question. Help each other. Ensure that the source code and related assets are pushed to GitHub prior to the scheduled session. Then, attend the scheduled session.

You will receive \textbf{peer feedback} within 3 working days after the session.

\subsection*{Part C}

Part C is a \textbf{single formative submission}. This work is \textbf{individual} and will be assessed on a \textbf{threshold} basis. The following criteria are used to determine a pass or fail:

\begin{enumerate}[label=(\alph*)]
	\item Submission is timely;
	\item Enough work is available to conduct a meaningful review;
	\item A broadly appropriate review of a peer's work is submitted.
\end{enumerate}

To complete Part C, prepare a draft version of the essay. Ensure that the source code and related assets are pushed to GitHub and a pull request is made prior to the scheduled session. Then, attend the scheduled session.

You will receive \textbf{peer feedback} within 3 working days after the session.

\subsection*{Part D}

Part D is a \textbf{single summative submission}. This work is \textbf{individual} and will be assessed on a \textbf{criterion-referenced} basis. Please refer to the marking rubric at the end of this document for further detail.

To complete Part D, revise the essay based on the feedback you have received. Then, upload the essay to the LearningSpace. Please note, the LearningSpace will only accept a single \texttt{.pdf} file.

You will receive \textbf{formal feedback} three weeks after the final deadline.

\section*{Additional Guidance}

Developing the research question is the most challenging aspect of this assignment. It is very unlikely that you will settle on the first research question that you propose. This is because the question will often arise out of your individual research and reading efforts. Furthermore, the question should relate to working practices for game developers. An example might be: ``which design patterns are most appropriate for localising dialogue?''. You will need to discuss your question with your tutor and your peers to help focus it.

Areas where students tend to lose marks are: depth of insight; analytical skill; and evaluative skill. Depth of insight implies rigorous research, addressing one key challenge in much detail, rather than several challenges with weaker research and/or in less detail. Adequete analysis implies going beyond mere descrption, perhaps through: performing calculatons, comparing sources, or even deploying reasoning to generate new insights. Adequete evaluation implies making appropriate reference to evidence and ensuring that evidence is of appropriate quality. Further to this, sound and valid arguments are constructed, criticising the claims made by other authors.

Focus on answering your research question. You have but 1000-words! Depth over breadth. Quality over quantitiy. Write concisely. Your ability to recall facts is not under assessment, your ability to construct an argument through critical analysis and making it relevant to practice is.

\section*{FAQ}

\begin{itemize}
	\item 	\textbf{What is the deadline for this assignment?} \\ 
    		Falmouth University policy states that deadlines must only be specified on LearningSpace. Please examine the assignment area where you located this document.
    		
	\item 	\textbf{What should I do to seek help?} \\ 
    		You can email your tutor for informal clarifications. For informal feedback, make a pull request on GitHub. 
    		
    	\item 	\textbf{Is this a mistake?} \\ 	
    		If you have discovered an issue with the brief itself, the source files are available at: \\
    		\url{https://github.com/Falmouth-Games-Academy/bsc-assignment-briefs}.\\
    		 Please make a pull request and comment accordingly.
\end{itemize}

\section*{Additional Resources}

\begin{itemize}
    \item Keith, C. (2010) Agile Game Development with Scrum. Pearson Education.
\end{itemize}

\begin{markingrubric}
%
    \firstcriterion{Parts A---C}{10\% (Threshold)}
        \gradespan{1}{\fail Parts A---C have not been submitted, are incomplete, or are unsatisfactory.}
        \gradespan{2}{Two parts incomplete.}
        \gradespan{2}{One part incomplete.}
        \gradespan{1}{Parts A---C are complete.}
%
    \criterion{Appropriateness of Referenced Articles}{10\%}
        \grade\fail 	No relevant article is referenced.
        \grade 		At least three relevant sources are referenced.
        \grade 		At least six relevant sources have been referenced.
        \par		Where appropriate, some sources report scholarly research.
        \grade 		At least eight relevant sources have been referenced.
        \par		Where appropriate, most articles report scholarly research.
        \grade 		At least ten relevant sources have been referenced.
        \par		Where appropriate, all sources report scholarly research.
        \par		Some appropriate seminal and highly reputed sources have been referenced.      
        \grade 		At least ten relevant sources have been referenced.
        \par		Where appropriate, all articles report scholarly research.
        \par		Many appropriate seminal and highly reputed sources have been referenced.   
%
    \criterion{Relevance to and Focus on the Research Question}{5\%}
        \grade\fail 	No focus on the research question.
        \grade 		Little focus on the research question.
        \grade 		Some focus on the research question.
        \grade 		Much focus on the research question.
            \par 		Research questions are explicitly defined.
        \grade 		Considerable focus on the research question.
            \par 		Research question is explicitly defined.
            \par 		Conclusion explicitly refers back to the question.
        \grade 		Significant focus on the research question.
            \par 		Research question is explicitly defined.
            \par 		Conclusion explicitly refers back to the question.
%
    \criterion{Depth of Insight into Software Engineering Principles}{20\%}
        \grade\fail 	No depth of insight into software engineering principles.
        \grade 		Little depth of insight into software engineering principles.
        \grade 		Some depth of insight into software engineering principles.
        \par 		Insight highlights a specific engineering challenge in digital games development.
        \grade 		Much depth of insight into software engineering principles.
        \par 		Insight highlights a specific and relevant engineering challenge in digital games development.
        \grade 		Considerable depth of insight into software engineering principles.
        \par 		Insight explores, in detail, a specific and relevant engineering challenge in digital games development.
        \grade 		Significant depth of insight into software engineering principles.
        \par 		Critical insight that explores and/or addresses, in detail, a specific and pertinent engineering challenge in digital games development.
%
    \criterion{Specificity, Verifiability, \& Accuracy of Claims}{5\%}
        \grade\fail 	No citations to evidence to claims.
        \par 		Substantial errors and/or misinterpretations.
        \grade 		Few claims have a clear source of evidence.
        \par 		Significant errors and/or misinterpretations.
        \grade 		Some claims have a clear source of evidence.
        \par 		Many errors and/or misinterpretations.
        \grade 		Many claims have a clear source of evidence.
        \par 		Some errors and/or misinterpretations.
        \grade 		Most claims have a clear source of evidence.
        \par 		Few errors and/or misinterpretations.
        \grade 		All claims have a clear source of evidence.
        \par 		Almost no errors and/or misinterpretations.
%
    \criterion{Adequacy of Analysis of Research Articles}{20\%}
        \grade\fail 	No analysis has been presented.
        \grade 		Little analysis has been presented.
        \grade 		Some analysis has been presented. 
        \grade 		Much analysis has been presented.
        \grade 		Considerable analysis has been presented.
        \grade 		Significant analysis has been presented.
%
    \criterion{Adequacy of Discussion on Transfer to the Games Industry}{15\%}
        \grade\fail 	No transfer to the games industry.
        \grade 		Little transfer to the games industry.
        \grade 		Some transfer to the games industry. 
        \par 		Appropriate references to the games industry and/or game development practice. 
        \grade 		Much transfer to the games industry.
        \par 		Appropriate argument suggesting effective game development practice. 
        \grade 		Considerable transfer to the games industry.
        \par 		Relevant criticism of game development practices, demonstrating insight into pitfalls and arguing for possible solutions. 
        \grade 		Significant transfer to the games industry.
        \par 		Relevant criticism of game development practices, demonstrating insight into key pitfalls and effectively defending appropriate solutions with evidence. 
%
    \criterion{Appropriateness of Academic Writing}{5\%}
        \grade\fail 	Little or no evidence of partial-mastery of academic writing.
        \par 		The reference section is missing.
        \grade 		Evidence of partial-mastery of academic writing.
        \par 		The reference section is incomplete and/or malformed.
        \grade 		Evidence of partial-mastery of academic writing.
        \par 		The reference section is complete and well-formed in either ACM or IEEE format.
        \par 		Most in-text citations and quotations are correct.
        \grade 		Some evidence of mastery of academic writing.
        \par 		The reference section is complete and well-formed in either ACM or IEEE format.
        \par 		All in-text citations and quotations are correct.
        \grade 		Much evidence of mastery of academic writing.
        \par 		The reference section is complete and well-formed in either ACM or IEEE format.
        \par 		All in-text citations and quotations are correct.
        \grade 		Considerable evidence of mastery of academic writing.
        \par 		The reference section is complete and well-formed in either ACM or IEEE format.
        \par 		All in-text citations and quotations are correct.
%
    \criterion{Appropriateness of Spelling \& Grammar}{5\%}
        \grade\fail 	Substantial spelling and/or grammar errors.
        \grade 		Many spelling and/or grammar errors.
        \grade 		Some spelling and/or grammar errors.  
        \grade 		Few spelling and/or grammar errors.
        \grade 		Almost no spelling and/or grammar errors.
        \grade 		No spelling or grammar errors.
%
    \criterion{Appropriateness of Essay Structure}{5\%}
        \grade\fail 	There is no structure, or the structure is unclear.
        \grade 		There is little structure.
        \grade 		There is some structure.
        \par 		A few sentences and paragraphs are well constructed.
        \grade 		There is much structure.
        \par 		Some sentences and paragraphs are well constructed.
        \par 		There is a clear introduction and conclusion.
        \grade 		There is much structure, highlighting the argument.
        \par 		Most sentences and paragraphs are well constructed.
        \par 		There is a clear and well-constructed introduction and conclusion.
        \grade 		There is much structure, highlighting the argument.
        \par 		All sentences and paragraphs are well constructed.
        \par 		There is a clear and well-constructed introduction and conclusion.
\end{markingrubric}

\end{document}