\documentclass{../fal_assignment}
\graphicspath{ {../} }

\usepackage{enumitem}
\setlist{nosep} % Make enumerate / itemize lists more closely spaced
\usepackage[T1]{fontenc} % http://tex.stackexchange.com/a/17858
\usepackage{url}
\usepackage{todonotes}

\title{API Hacking}
\author{Alcwyn Parker}

\begin{document}

\maketitle
%\begin{marginquote}
%    ``Students come into programming classes with a broad range of backgrounds ---
%    some have experience in several programming languages, others have never programmed before in their life.
%    
%    Being able to engage with the community and support each other is important.
%    Upload your code to GitHub and receive feedback from experienced peers.
%    Review your peers' work yourself and really consider what `quality' actually means
%    and what `good' source code looks like.
%    Debate, argue, and question others about it ---
%    an open and sustained discourse is an excellent way to learn ---
%    for both beginners and adepts!''
%\end{marginquote}
\marginpicture{flavour_pic}{
    \emph{ANGELINA}, by Falmouth University researcher Michael Cook, uses fifteen different APIs and web services to automatically design entire games.}

\section*{Introduction}

In this assignment, you are required to design and implement a component for a game.
The component must be based on one or more \textbf{existing third-party APIs}:
for example web services, hardware device SDKs, open-source libraries, or open dataset formats.
Your component must be integrated into either:
\begin{enumerate}[label=(\alph*)]
    \item A game developed by \textbf{BA Digital Games} students (liaise with your chosen BA team before choosing this option); or
    \item Your \textbf{COMP130} Kivy game project; or
    \item Your \textbf{COMP150} group game project (\textbf{limit one per team}: liaise with the rest of your team before choosing this option); or
    \item \label{item:game-opensource} An \textbf{open-source game}; or
    \item \label{item:game-mod} A \textbf{commercial game} which has a well-established (official or community-made) modding interface that allows mods to be developed in one of the permitted languages.
\end{enumerate}
Whichever option you choose, you may implement your component in \textbf{C++}, \textbf{C\#}, \textbf{Python}, or a combination of these.
Other languages may be permitted at the discretion of your tutor, if you can argue convincingly that it is appropriate for your proposed project.
Note that options~\ref{item:game-opensource} and~\ref{item:game-mod} above are likely to be more technically challenging than the others,
so are recommended only for those comfortable with advanced programming concepts.

You may work \textbf{individually} or \textbf{in pairs}.
This project is assessed individually, so ensure that \textbf{your individual contribution is clearly indicated}.

This assignment is formed of \textbf{three} parts.
There is a \textbf{single summative submission} via LearningSpace.
Begin by forking the GitHub project at the following URL:
\begin{center}
\url{https://github.com/Falmouth-Games-Academy/comp140-api-hacking}
\end{center}

\subsection*{A. \emph{Propose} a game component}

On GitHub, edit the \texttt{readme.md} file to contain a description of your proposed game component.
Your proposal should:
\begin{itemize}
    \item \textbf{Identify} the game into which your component will be integrated;
    \item \textbf{Describe} the component that will be created;
    \item \textbf{Show} the key user stories, in the form of a Trello task board;
    \item \textbf{Identify and justify} which technologies (implementation language, APIs, etc.) you will use.
\end{itemize}
Your proposal will be assessed on:
\begin{itemize}
    \item \textbf{Scope}:
        is the proposed component non-trivial, but feasible given the duration of the project?
    \item \textbf{Appropriateness of component}:
        will the proposed component enhance the selected game?
    \item \textbf{Appropriateness of technologies}:
        is the choice of implementation language and APIs sensible and well-justified?
\end{itemize}

\textbf{Formative submission:} Discuss your proposal and task board with tutors in class.

\subsection*{B. \emph{Implement} the component}

You will build your prototype component over \textbf{three weeks}.
You should aim to have a `potentially shippable' component at the end of each week;
that is, a component which does not have any major flaws or half-finished features that prevent it from being tested.
For the first two weeks, the component should be working either within the game or in a separate testing environment;
for the final week, the component should be fully integrated into the game.

Your implementation will be assessed on the criteria listed in the marking rubric.

\textbf{Formative submission:} 
Check your code into GitHub regularly (either your forked \texttt{comp140-api-hacking} repository or a branch of another appropriate repository),
and make a pull request whenever you require assistance or feedback.

\subsection*{C. \emph{Demonstrate} your component}

Bring an executable of the game, with your component integrated, to the demo session in class.
Be prepared to discuss it with tutors and peers.

This part is not assessed, but you will receive feedback which will be useful to you on future projects.

\textbf{Formative submission:} Participate in the demo session.

\begin{marginquote}
    ``There are two ways of constructing a software design.
        One way is to make it so simple that there are obviously no deficiencies.
        And the other way is to make it so complicated that there are no obvious deficiencies.''
    
    --- C.\ A.\ R.\ Hoare
\end{marginquote}

\subsection*{Summative submission (electronic)}

Create a zip file containing the following:

\begin{itemize}
\item A markdown file named \texttt{readme.md} containing your \textbf{proposal} from part~A,
    and any other \textbf{documentation} you feel is appropriate
    (e.g.\ special instructions for testing the component).
\item Screenshots of your \textbf{Trello task board} to evidence your project management.
    and screenshots documenting any checklists or other information within your cards.
\item All \textbf{source code} created for part~B.
    You do \textbf{not} need to provide the entire source code for the game.
    Any code which was \textbf{not written by you} must be \textbf{clearly identified} as such,
    in the \texttt{readme.md} file and/or as comments in the code files themselves.
\item A \textbf{Windows executable} version of the game with your component integrated.
    Or, in the case of a mod for a commercial game, an executable version of the mod with clear installation instructions.
    \textbf{Do not include copyright material which you do not have permission to redistribute}.
\end{itemize}

The recommended way to produce this zip file is to check all of the above into your GitHub repository
throughout the course of the project,
and then use the ``Download Zip'' function on the GitHub website.
Any material you do not wish to upload to GitHub (e.g.\ the executable)
must be added to the zip file manually before uploading.

Upload your zip file to the appropriate submission queue on LearningSpace.
Note that LearningSpace accepts only a single zip file per submission,
with a maximum file size of 1GB.
If the game executable is larger than this, contact your tutors to make alternative arrangements.

\section*{Additional Guidance}

Creating new software solutions from scratch is an important skill for software developers.
However it can lead to ``reinventing the wheel'': spending much time and effort solving problems which have already been solved by others.
Thus an equally important skill is being able to bring together existing code from different sources to produce a cohesive whole.
Ensure that your \textbf{choice of technologies} is appropriate.
%Your choice of implementation language may be somewhat constrained:
%for example a component for a Unity game will need to be implemented in C\#, or in C++ using Unity's native plugin interface.
%Depending on your proposed component, you may have a wider choice in terms of which API or APIs you use.
Read up on the alternatives to determine which is the best fit for your proposed component.

Falmouth University is nationally and internationally renowned as an arts institution.
Despite the fact that you are studying for a Bachelor of Science degree in a technical discipline,
you are still expected to strive for the same level of \textbf{innovation and creative flair}
as your fellow students in other departments.
All assignments on this course involve a mix of technical and creative activities;
cultivating both of these skills in tandem will stand you in good stead for a career in the games industry.
One approach to promoting creativity is
\textbf{divergent thinking}: generation of ideas by exploring many possible solutions.
Often the most interesting ideas are \textbf{subversive}: they deliberately go against the
conventional or most obvious solution.

The first step in planning your implementation should be to break your proposed component down into \textbf{user stories}. 
Your user stories should be \textbf{distinguishable} (i.e.\ there should be little overlap between them)
and \textbf{easily measured} (i.e.\ it should be easy to tell when each user story has been implemented).
They should also be \textbf{comprehensive}, i.e.\ the user stories should completely capture the
desired functionality of the component, with no gaps.
Your code will be assessed on \textbf{functional coherence}:
how well the finished component corresponds to the user stories,
and whether there are any obvious bugs.
Correspondence to user stories runs both ways:
implementing features that were not present in the design (``feature creep'')
is just as bad as neglecting to implement features.

Your code will also be assessed on \textbf{sophistication}.
To succeed on a project of this size and complexity,
you will need to make use of appropriate algorithms, data structures, library features, and object oriented programming concepts.
Appropriateness to the task at hand is key, however:
you will \textbf{not} receive credit for shoehorning a complicated solution into your program
where a simpler one would have sufficed.

\textbf{Maintainability} is important in all programming projects,
but doubly so when working in a team.
Use \textbf{comments} liberally to improve comprehension of your code,
and choose carefully the \textbf{names} for your files, classes, functions and variables.
For high marks you should use a well-established commenting convention
for \textbf{high-level documentation} of your files, classes and functions.
The open-source tool Doxygen supports several such conventions.
Also ensure that all code corresponds to a sensible and consistent \textbf{formatting style}:
indentation, whitespace, placement of curly braces, etc.
Hard-coded \textbf{literals} (numbers and strings) within the source should be avoided,
with values instead defined as constants together in a single place.
%Consider allowing some literal values, where appropriate, to be ``tinkered'' without changing the source code,
%e.g.\ by defining them in an external file read by the game on startup.

\begin{marginquote}
    ``Measuring programming progress by lines of code is like measuring aircraft building progress by weight.''
    
    --- Bill Gates
\end{marginquote}
\marginpicture{wheel}{
    Don't reinvent this.
}
\section*{Additional Resources}

\begin{itemize}
    \item M. G. Friberger et al. (2013) Data Games. Proceedings of Procedural Content Generation Workshop. FDG 2013.
        \url{http://julian.togelius.com/Friberger2013Data.pdf}
    \item Cook, M. and Colton, S. (2014) Ludus Ex Machina: Building a 3D game designer that competes alongside humans. Proceedings of ICCC 2014.
        \url{http://ccg.doc.gold.ac.uk/papers/cook_iccc2014.pdf}
    \item \url{http://docs.unity3d.com/Manual/NativePlugins.html}
\end{itemize}

\begin{markingrubric}
    \firstcriterion{Sprint reviews}{Threshold \par 5\% + 5\%}
        \gradespan{2}{\fail None of the sprints are delivered, or no `reasonable' peer reviews are submitted.}
        \gradespan{2}{A `potentially shippable' component is produced at the end of at least one sprint.
            \par A `reasonable' review of at least one peer's work is provided in at least one of the review sessions.}
        \gradespan{2}{A `potentially shippable' component is produced at the end of all three sprints.
            \par A `reasonable' review of at least one peer's work is provided in each of the review sessions.}
%
    \criterion{Choice of technologies (implementation language and APIs)}{10\%}
        \grade\fail The choice of technologies is inappropriate.
            \par Justification is inappropriate or not provided.
        \grade The choice of technologies is appropriate.
            \par Little appropriate justification is provided.
        \grade The choice of technologies is appropriate.
            \par Justification is appropriate, but no alternatives have been considered.
        \grade The choice of technologies is appropriate.
            \par Justification is appropriate, and alternatives have been considered.
        \grade The choice of technologies is highly appropriate.
            \par Justification is strong, showing a working knowledge of the chosen technology and its alternatives.
        \grade The choice of technologies is highly appropriate.
            \par Justification is exemplary, showing insight into the chosen technology and its alternatives.
%
    \criterion{Design of the solution}{10\%}
        \grade\fail User stories are not provided, or the design does not correspond to the user stories.
        \grade Few user stories are distinguishable and easily measured.
            \par The correspondence between design and user stories is tenuous.
        \grade Some user stories are distinguishable and easily measured.
            \par The design somewhat corresponds to the user stories.
        \grade Most user stories are distinguishable and easily measured.
            \par The design corresponds to the user stories.
        \grade Nearly all user stories are distinguishable and easily measured.
            \par The design clearly corresponds to the user stories.
        \grade All user stories are distinguishable and easily measured.
            \par The design clearly and comprehensively corresponds to the user stories.
%
    \criterion{Functional coherence}{10\%}
        \grade\fail No user stories have been implemented.
            \par There are critical bugs that prevent the code from compiling or running.
        \grade Few user stories have been implemented.
            \par There are many obvious and serious bugs.
        \grade Some user stories have been implemented.
            \par There are some obvious bugs.
        \grade Many user stories have been implemented.
            \par There is some evidence of feature creep.
            \par There are few obvious bugs.
        \grade Almost all user stories have been implemented.
            \par There is little evidence of feature creep.
            \par There are some minor bugs.
        \grade All user stories have been implemented.
            \par There is no evidence of feature creep.
            \par Bugs, if any, are purely cosmetic and/or superficial.
%
    \criterion{Sophistication}{20\%}
        \grade\fail No insight into the appropriate use of programming constructs is evident from the source code.
            \par APIs are used inappropriately.
            \par No attempt to structure the program is evident (e.g. one monolithic source file).
        \grade Little insight into the appropriate use of programming constructs is evident from the source code.
            \par APIs are used somewhat appropriately.
            \par The program structure is poor.
        \grade Some insight into the appropriate use of programming constructs is evident from the source code.
            \par APIs are used mostly appropriately.
            \par The program structure is adequate.
        \grade Much insight into the appropriate use of programming constructs is evident from the source code.
            \par APIs are used appropriately, with some mastery evident.
            \par The program structure is appropriate.
        \grade Significant insight into the appropriate use of programming constructs is evident from the source code.
            \par APIs are used appropriately, with much mastery evident.
            \par The program structure is effective. There is high cohesion and low coupling.
        \grade Exemplary insight into the appropriate use of programming constructs is evident from the source code.
            \par APIs are used appropriately, with exemplary mastery evident.
            \par The program structure is very effective. There is high cohesion and low coupling.
%
    \criterion{Maintainability}{20\%}
        \grade\fail There are no comments, or comments are misleading.
            \par Most variable names are unclear or inappropriate.
            \par Code formatting hinders readability.
        \grade The code is only sporadically commented, or comments are unclear.
            \par Some identifier names are unclear or inappropriate.
            \par Code formatting is inconsistent or does not aid readability.
        \grade The code is well commented.
            \par Some identifier names are descriptive and appropriate.
            \par An attempt has been made to adhere to a consistent formatting style.
             \par There is little obvious duplication of code or of literal values.           
        \grade The code is reasonably well commented.
            \par Most identifier names are descriptive and appropriate.
            \par Most code adheres to a consistent formatting style.
             \par There is almost no obvious duplication of code or of literal values.   
        \grade The code is reasonably well commented, with some Doxygen-compatible module documentation.
            \par Almost all identifier names are descriptive and appropriate.
            \par Almost all code adheres to a consistent formatting style.
             \par There is no obvious duplication of code or of literal values. %Some literal values can be easily ``tinkered''. 
        \grade The code is very well commented, with comprehensive Doxygen-compatible module documentation.
            \par All identifier names are descriptive and appropriate.
            \par All code adheres to a consistent formatting style.
             \par There is no obvious duplication of code or of literal values. %Most literal values are, where appropriate, easily ``tinkered'' outside of the source.  
%
    \criterion{Innovation and creative flair}{10\%}
        \grade\fail Demonstrates no evidence of innovation and/or creativity.
        \grade Demonstrates evidence of emerging innovation and/or creativity.
            \par The solution is purely derivative of existing products.
            \par There is no evidence of divergent thinking.
        \grade Demonstrates evidence of progressing innovation and/or creativity.
            \par The solution is mostly derivative, with some attempts at innovation.
            \par There is evidence of an attempt at divergent thinking.
        \grade Demonstrates evidence of partial mastery of innovative and creative practice.
            \par The solution is an interesting and somewhat innovative product.
            \par There is some evidence of divergent thinking.
        \grade Demonstrates some evidence of mastery of innovative and creative practice.
            \par The solution is a novel and innovative product.
            \par There is much evidence of divergent thinking.
        \grade Demonstrates much evidence of mastery of innovative and creative practice.
            \par The solution is a unique and innovative product.
            \par There is significant evidence of divergent thinking.
%
    \criterion{Portability and navigability}{5\%}
        \grade\fail Game will not execute at all on another machine for reasons related to code portability which cannot be fixed easily due to its poor structure.
            \par The provided template has not been followed.
        \grade There were challenges executing the game, but these were resolvable.
            \par The directory structure inside the submitted zip file is unclear.
            \par The provided template has not been followed.
        \grade Several portability issues are present.
            \par The directory structure inside the submitted zip file is somewhat confusing.
            \par The provided template has mostly been followed.
        \grade Some portability issues are present.
            \par The directory structure inside the submitted zip file is adequate.
            \par The provided template has been followed.
        \grade Few portability issues are present.
            \par The directory structure inside the submitted zip file is mostly sensible.
            \par The provided template has been followed.
        \grade Almost no portability issues are present.
            \par The directory structure inside the submitted zip file is sensible.
            \par The provided template has been followed.
%
    \criterion{Use of version control}{5\%}
        \grade\fail GitHub has not been used.
        \grade Material has only been checked into GitHub a few times before the deadline.
        \grade Material has been checked into GitHub at least once per sprint.
        \grade Material has been checked into GitHub several times per sprint.
        \grade Material has been checked into GitHub several times per sprint.
            \par Commit messages are clear, concise and relevant.
        \grade Material has been checked into GitHub several times per sprint.
            \par Commit messages are clear, concise and relevant.
            \par There is evidence of engagement with peers (e.g.\ voluntary code review).
\end{markingrubric}

\end{document}
