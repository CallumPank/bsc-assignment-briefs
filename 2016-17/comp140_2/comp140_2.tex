\documentclass{../fal_assignment}
\graphicspath{ {../} }

\usepackage{enumitem}
\setlist{nosep} % Make enumerate / itemize lists more closely spaced
\usepackage[T1]{fontenc} % http://tex.stackexchange.com/a/17858
\usepackage{url}
\usepackage{todonotes}

\title{API Hacking}
\author{Alcwyn Parker}

\begin{document}

\maketitle

\section*{Introduction}

\begin{marginquote}
``There are two "rules of three" in [software] reuse:

It is three times as difficult to build reusable components as single use components, and

a reusable component should be tried out in three different applications before it will be sufficiently general to accept into a reuse library.'
\par ---  Robert L. Glass

\end{marginquote}

\marginpicture{flavour_pic}{
    @UncharteredAtlas is a twitter bot that replies to tweets by creating generative terrain maps styled to look like kind you find in fantasy novels.
}

In this assignment, you are required to design and implement a component for a game. The component must be based on one or more existing third-party APIs: for example web services, hardware device SDKs, open-source libraries, or open dataset formats. Your prototype should function as an add-on to either one of the games being developed by students on the BA Digital Games course; or the game that you developed in COMP150. Whichever option you choose, you may implement your component in C++, C\#, Python, or a combination of these. 

Application programming interfaces (API) allow programmers access to a vast array of third-party services and data-sources from social networks to real-time sensor networks. They are the conduits for communication between applications. Most software projects will involve API integration of some kind and so the focus of this assignment is to help you familiarise yourself with the concepts and techniques needed to integrate third-party services into your own projects.

This assignment is formed of several parts:

\begin{enumerate}[label=(\Alph*)]
    \item \textbf{Write} a proposal for a game component that will:
        	\begin{enumerate}[label=\roman*.]
    		\item \textbf{state and justify} the game that will be the basis for your comonent;
    		\item \textbf{outline} the initial concept in detail;
		\item \textbf{show} the key user stories, in the form of a Trello task board;
		\item and \textbf{list} the key requirements the component must fulfil;
		\item including which technologies (implementation languages, APIs, etc.) you will use.  
	\end{enumerate}
    \item \textbf{Implement} a draft version of your your game component :
    	\begin{enumerate}[label=\roman*.]
		\item \textbf{improve} upon the design over the course of two weeks;
	\end{enumerate}
    \item \textbf{Implement} the final version of your game component that will:
    	\begin{enumerate}[label=\roman*.]
    		\item \textbf{revise} the design based on feedback from your peers;
	\end{enumerate}
     \item \textbf{Present} a practical demo of the game controller to your tutor that will:
    	\begin{enumerate}[label=\roman*.]
    		\item \textbf{demonstrate} your academic integrity;
    		\item and \textbf{show} your programming knowledge \textbf{and} communication skills.
	\end{enumerate}
\end{enumerate}    

\subsection*{Assignment Setup}

This assignment is a \textbf{pair programming task}. Fork the GitHub repository at:

\indent \url{https://github.com/Falmouth-Games-Academy/comp140-api-hacking }

Use the existing directory structure and, as required, extend this structure with sub-directories. Ensure that you maintain the \texttt{readme.md} file.

Modify the \texttt{.gitignore} to the defaults for \textbf{Python}. Please, also ensure that you add editor-specific files and folders to \texttt{.gitignore}. 

\subsection*{Part A}

Part A consists of a \textbf{single formative submission}. This work is \textbf{individual} and will be assessed on a \textbf{threshold} basis. The following criteria are used to determine a pass or fail:

\begin{enumerate}[label=(\alph*)]
	\item Submission is timely;
	\item Design is feasible, distinctive and has creative merit.
\end{enumerate}

To complete part A, write your proposal in the \texttt{readme.md} document. Show this to your tutor in-class. If acceptable, this will be signed-off.

\subsection*{Part B}

Part B is formed of \textbf{multiple formative submissions}. This work is \textbf{individual} and will be assessed on a \textbf{threshold} basis. The following criteria are used to determine a pass or fail:

\begin{enumerate}[label=(\alph*)]
	\item Submission is timely;
	\item Enough progress is made to conduct a meaningful review each week;
\end{enumerate}

You will build your prototype component over \textbf{three weeks}. You should aim to have a `potentially shippable' component at the end of each week; that is, a component which does not have any major flaws or half-finished features that prevent it from being tested. For the first two weeks, the component should be working either within the game or in a separate testing environment; for the final week, the component should be fully integrated into the game. Ensure that any digital artefacts (including but not limited to design sketches, photographs, art assets, source code) are pushed to GitHub prior to each weekly session. Also ensure that you bring your prototype with you each week. 

You will receive immediate \textbf{informal feedback} from your \textbf{tutor} and \textbf{peers}.

\subsection*{Part C}

Part C is a \textbf{single summative submission}. This work is \textbf{individual} and will be assessed on a \textbf{criterion-referenced} basis. Please refer to the marking rubric at the end of this document for further detail.

To complete Part C, revise the game component based on the feedback you have received. Then, upload the source code to the LearningSpace. Please note, the LearningSpace will only accept a single \texttt{.zip} file.

You will receive \textbf{formal feedback} from your \textbf{tutor} three weeks after the final submission deadline.

\subsection*{Part D}

Part D is a \textbf{single summative submission}. This is \textbf{individual} work will be assessed on a \textbf{threshold} basis. The following criteria are used to determine a pass or fail: 

\begin{enumerate}[label=(\alph*)]
	\item Enough work is available to hold a meaningful discussion; 
	\item Clear evidence of programming knowledge and communication skills; 
	\item No breaches of academic integrity. 
\end{enumerate}

To complete Part D, prepare a practical demonstration of the game compnent. Ensure that the source code and related assets are pushed to GitHub and a pull request is made prior to the scheduled viva session. Then, attend the scheduled viva session. 

You will receive \textbf{immediate informal} feedback from your \textbf{tutor}.


\section*{Additional Guidance}

Creating new software solutions from scratch is an important skill for software developers. However it can lead to `reinventing the wheel': spending much time and effort solving problems which have already been solved by others. Thus an equally important skill is being able to bring together existing code from different sources to produce a cohesive whole. Ensure that your choice of technologies is appropriate. Read up on the alternatives to determine which is the best fit for your proposed component. 

Falmouth University is nationally and internationally renowned as an arts institution. Despite the fact that you are studying for a Bachelor of Science degree in a technical discipline, you are still expected to strive for the same level of innovation and creative flair as your fellow students in other departments. All assignments on this course involve a mix of technical and creative activities; cultivating both of these skills in tandem will stand you in good stead for a career in the games industry. One approach to promoting creativity is divergent thinking: generation of ideas by exploring many possible solutions. Often the most interesting ideas are subversive: they deliberately go against the conventional or most obvious solution. 

The first step in planning your implementation should be to break your proposed component down into user stories. Your user stories should be distinguishable (i.e. there should be little overlap between them) and easily measured (i.e. it should be easy to tell when each user story has been implemented). They should also be comprehensive, i.e. the user stories should completely capture the desired functionality of the component, with no gaps. Your code will be assessed on functional coherence: how well the finished component corresponds to the user stories, and whether there are any obvious bugs. Correspondence to user stories runs both ways: implementing features that were not present in the design (`feature creep') is just as bad as neglecting to implement features. 

Your code will also be assessed on sophistication. To succeed on a project of this size and complexity, you will need to make use of appropriate algorithms, data structures, library features, and object oriented programming concepts. Appropriateness to the task at hand is key, however: you will not receive credit for shoehorning a complicated solution into your program where a simpler one would have sufficed. 

Maintainability is important in all programming projects, but doubly so when working in a team. Use comments liberally to improve comprehension of your code, and choose carefully the names for your files, classes, functions and variables. For high marks you should use a well-established commenting convention for high-level documentation of your files, classes and functions. The open-source tool Doxygen supports several such conventions. Also ensure that all code corresponds to a sensible and consistent formatting style: indentation, whitespace, placement of curly braces, etc. Hard-coded literals (numbers and strings) within the source should be avoided, with values instead defined as constants together in a single place.

\section*{FAQ}

\begin{itemize}
	\item 	\textbf{What is the deadline for this assignment?} \\ 
    		Falmouth University policy states that deadlines must only be specified on LearningSpace. Please examine the assignment area where you located this document.
    		
	\item 	\textbf{What should I do to seek help?} \\ 
    		You can email your tutor for informal clarifications. For informal feedback, make a pull request on GitHub. 
    		
    	\item 	\textbf{Is this a mistake?} \\ 	
    		If you have discovered an issue with the brief itself, the source files are available at: \\
    		\url{https://github.com/Falmouth-Games-Academy/bsc-assignment-briefs}.\\
    		 Please raise an issue and comment accordingly.
\end{itemize}

\section*{Additional Resources}

\begin{itemize}
    \item A list of useful API
        \url{https://github.com/toddmotto/public-apis}
    \item Restful API Tutorial
        \url{http://www.restapitutorial.com/}
    \item \url{http://docs.unity3d.com/Manual/NativePlugins.html}
\end{itemize}


\begin{markingrubric}
%
    \firstcriterion{Choice of technologies (implementation language and APIs)}{5\%}
        \grade\fail The choice of technologies is inappropriate.
            \par Justification is inappropriate or not provided.
        \grade The choice of technologies is appropriate.
            \par Little appropriate justification is provided.
        \grade The choice of technologies is appropriate.
            \par Justification is appropriate, but no alternatives have been considered.
        \grade The choice of technologies is appropriate.
            \par Justification is appropriate, and alternatives have been considered.
        \grade The choice of technologies is highly appropriate.
            \par Justification is strong, showing a working knowledge of the chosen technology and its alternatives.
        \grade The choice of technologies is highly appropriate.
            \par Justification is exemplary, showing insight into the chosen technology and its alternatives.
%
    \criterion{Design of the solution}{10\%}
        \grade\fail No user stories are provided, or the design does not correspond to the user stories.
        \grade Some user stories are distinguishable and easily measured.
            \par The correspondence between design and user stories is tenuous.
        \grade Little user stories are distinguishable and easily measured.
            \par The design somewhat corresponds to the user stories.
        \grade Most user stories are distinguishable and easily measured.
            \par The design corresponds to the user stories.
        \grade Nearly all user stories are distinguishable and easily measured.
            \par The design clearly corresponds to the user stories.
        \grade All user stories are distinguishable and easily measured.
            \par The design clearly and comprehensively corresponds to the user stories.
%
    \criterion{Functional coherence}{10\%}
        \grade\fail No user stories have been implemented.
            \par There are critical bugs that prevent the code from compiling or running.
        \grade Some user stories have been implemented.
            \par There are many obvious and serious bugs.
        \grade A little number of user stories have been implemented.
            \par There are some obvious bugs.
        \grade Many user stories have been implemented.
            \par There is some evidence of feature creep.
            \par There are few obvious bugs.
        \grade A considerable number of user stories have been implemented.
            \par There is little evidence of feature creep.
            \par There are some minor bugs.
        \grade A significant number of user stories have been implemented.
            \par There is no evidence of feature creep.
            \par Bugs, if any, are purely cosmetic and/or superficial.
%
    \criterion{Sophistication}{10\%}
        \grade\fail No insight into the appropriate use of programming constructs is evident from the source code.
            \par No attempt to structure the program (e.g. one monolithic function).
        \grade Little insight into the appropriate use of programming constructs is evident from the source code.
            \par The program structure is poor.
        \grade Some insight into the appropriate use of programming constructs is evident from the source code.
            \par The program structure is adequate.
        \grade Much insight into the appropriate use of programming constructs is evident from the source code.
            \par The program structure is appropriate.
        \grade Considerable insight into the appropriate use of programming constructs is evident from the source code.
            \par The program structure is effective. There is high cohesion and low coupling.
        \grade Significant insight into the appropriate use of programming constructs is evident from the source code.
            \par The program structure is very effective. There is high cohesion and low coupling.
%
     \criterion{Maintainability}{15\%}
        \grade\fail There are no comments in the source code, or comments are misleading.
            \par Most variable names are unclear or inappropriate.
            \par Code formatting hinders readability.
        \grade The source code is only sporadically commented, or comments are unclear.
            \par Some identifier names are unclear or inappropriate.
            \par Code formatting is inconsistent or does not aid readability.
        \grade The source code is somewhat well commented.
            \par Some identifier names are descriptive and appropriate.
            \par An attempt has been made to adhere to an appropriate formatting style standard.
             \par There is little obvious duplication of code or of literal values.           
        \grade The source code is reasonably well commented.
            \par Most identifier names are descriptive and appropriate.
            \par Most code adheres to an appropriate formatting style standard.
             \par There is almost no obvious duplication of code or of literal values.   
        \grade The source code is reasonably well commented.
            \par Almost all identifier names are descriptive and appropriate.
            \par Almost all code adheres to an appropriate formatting style standard.
             \par There is no obvious duplication of code or of literal values. Some literal values can be easily ``tinkered'' in the source code. 
        \grade The source code is very well commented.
            \par All identifier names are descriptive and appropriate.
            \par All source code adheres to an appropriate formatting style standard.
             \par There is no obvious duplication of code or of literal values. Most literal values are, where appropriate, easily ``tinkered'' outside of the source code.  
%
    \criterion{Portability and navigability}{5\%}
        \grade\fail Game will not execute at all on another machine for reasons related to code portability which cannot be fixed easily due to its poor structure.
            \par The provided template has not been followed.
        \grade There were some challenges executing the game, but these were resolvable.
            \par The directory structure inside the submitted zip file is unclear.
            \par The provided template has not been followed.
        \grade Several portability issues are present.
            \par The directory structure inside the submitted zip file is somewhat confusing.
            \par The provided template has mostly been followed.
        \grade Some portability issues are present.
            \par The directory structure inside the submitted zip file is adequate.
            \par The provided template has been followed.
        \grade Few portability issues are present.
            \par The directory structure inside the submitted zip file is mostly sensible.
            \par The provided template has been followed.
        \grade Almost no portability issues are present.
            \par The directory structure inside the submitted zip file is sensible.
            \par The provided template has been followed.
%
    \criterion{Use of Version Control}{5\%}
        \grade\fail GitHub has not been used.
        \grade Source code has rarely been checked into GitHub.
        \grade Source code  has been checked into GitHub at least once per week.
            \par Commit messages are present.
            \par There is evidence of engagement with peers (e.g.\ code review).
        \grade Source code  has been checked into GitHub several times per week.
            \par Commit messages are clear, concise and relevant.
            \par There is evidence of somewhat meaningful engagement with peers (e.g.\ code review).
        \grade Source code has been checked into GitHub several times per week.
            \par Commit messages are clear, concise and relevant.
            \par There is evidence of meaningful engagement with peers (e.g.\ code review).
        \grade Source code has been checked into GitHub several times per week.
            \par Commit messages are clear, concise and relevant.
            \par There is evidence of effective engagement with peers (e.g.\ code review).
%
   \criterion{Basic Competency Threshold}{40\%}
        \gradespan{1}{\fail At least one part is missing or is unsatisfactory. 
        
        There is little or no evidence of an iterative development process and no improvement over time in regards to the quality of the design and build of the prototype.}
        \gradespan{5}{Submission is timely.
        	\par Enough work is available to hold a meaningful discussion.
	\par Clear evidence of a `reasonable' iterative development process
	\par Clear evidence of programming knowledge and communication skills.
	\par Clear evidence of reflection on own performance and contribution.
	\par Only constructive criticism of pair-programming partner is raised.
	\par No breaches of academic integrity.}
	
\end{markingrubric}

\end{document}