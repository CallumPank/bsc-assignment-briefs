\documentclass{../fal_assignment}
\graphicspath{ {../} }

\usepackage{enumitem}
\setlist{nosep} % Make enumerate / itemize lists more closely spaced
\usepackage[T1]{fontenc} % http://tex.stackexchange.com/a/17858
\usepackage{url}
\usepackage{todonotes}

\title{Client-Server Game Task}
\author{Dr Michael Scott}

\begin{document}

\maketitle

\section*{Introduction}

\begin{marginquote}
``Essentially everyone, when they first build a distributed application, makes the following eight assumptions:

\par - The network is reliable;
\par - Latency is zero;
\par - Bandwidth is infinite;
\par - The network is secure;
\par - Topology doesn't change;
\par - There is one administrator;
\par - Transport cost is zero;
\par - The network is homogeneous.

All prove to be false in the long run and all cause big trouble and painful learning experiences.''

\par --- Peter Deutsch
\end{marginquote}
\marginpicture{flavour_pic}{
    \textit{MUD1}, written by Roy Trubshaw and Richard Bartle---now, Prof. Bartle, the external examiner---was the first virtual world to run multiplayer over a network. It was written in 1978, pre-dating the Internet and operating over ARPANET.
}

In this assignment, you are required to implement a multiplayer online text-based adventure game using a client-server architecture. You have the option of using either the Python or Java programming language to implement the game server. You will use C++ to implement the gamer client.

Knowledge of computer networking is essential. You will need to work with highly sophisticated network technologies to satisfy increasing demand for social experiences in games. Experience working with client-server architectures, in particular, is highly desired by employers even outside games.

This assignment is formed of several parts:

\begin{enumerate}[label=(\Alph*)]
    \item \textbf{Design} a software architecture for your game in UML, that will:
    	\begin{enumerate}[label=\roman*.]
    		\item \textbf{support} design elements of the game \textit{Colossal Cave Adventure};
    		\item and \textbf{incorporate} both a client \textbf{and} a server.
	\end{enumerate}
    \item \textbf{Implement} a prototype of the game, in Java/Python/C++, that will:
    	\begin{enumerate}[label=\roman*.]
    		\item \textbf{illustrate} the key features of the game design.
	\end{enumerate}
    \item \textbf{Write} a final prototype of the game that will:
    	\begin{enumerate}[label=\roman*.]
    		\item \textbf{revise} any issues raised by your tutor and/or your peers.
	\end{enumerate}
    \item \textbf{Present} a practical demo of the game prototype to your tutor that will:
    	\begin{enumerate}[label=\roman*.]
    		\item \textbf{show} your academic integrity;
    		\item as well as \textbf{demonstrate} your knowledge of computer networking \textbf{and} your technical communication skills.
	\end{enumerate}
\end{enumerate}

\subsection*{Assignment Setup}

This assignment is a \textbf{programming task}. Fork the GitHub repositories at:

\indent \url{https://github.com/Falmouth-Games-Academy/comp260-mud-client}
\\ \indent \url{https://github.com/Falmouth-Games-Academy/comp260-mud-server}

Use the existing directory structure and, as required, extend this structure with sub-directories. Ensure that you maintain the \texttt{readme.md} file.

Modify the \texttt{.gitignore} to the defaults for \textbf{Python}, \textbf{Eclipse}, and \textbf{Visual Studio} as appropriate. Please, also ensure that you add editor-specific files and folders to \texttt{.gitignore}. 

\subsection*{Part A}

Part A consists of a \textbf{single formative submission}. This work is \textbf{individual} and will be assessed on a \textbf{threshold} basis. 

To complete Part A, write about your contract in the \texttt{readme.md} document.  Show this to your tutor in-class.  If acceptable, this will be signed-off. 

You will receive immediate \textbf{informal feedback} from your \textbf{tutor}.

\subsection*{Part B}

Part B is a \textbf{single formative submission}. This work is \textbf{individual} and will be assessed on a \textbf{threshold} basis. The following criteria are used to determine a pass or fail:

\begin{enumerate}[label=(\alph*)]
	\item Submission is timely;
	\item Enough work is available to conduct a meaningful review;
	\item A broadly appropriate review of a peer's work is submitted.
\end{enumerate}

To complete Part B, prepare draft versions of the computer program. Ensure that the source code and related assets are pushed to GitHub and a pull request is made prior to the scheduled peer-review session. Then, attend the scheduled peer-review session.

You will receive immediate \textbf{informal feedback} from your \textbf{peers}.

\subsection*{Part C}

Part C is a \textbf{single summative submission}. This work is \textbf{individual} and will be assessed on a \textbf{criterion-referenced} basis. Please refer to the marking rubric at the end of this document for further detail.

To complete Part C, revise the computer program based on the feedback you have received. Then, upload it to the LearningSpace. Please note, the LearningSpace will only accept a single \texttt{.zip} file.

You will receive \textbf{formal feedback} from your \textbf{tutor} three weeks after the final submission deadline.

\subsection*{Part D}

Part D is a \textbf{single summative submission}. This work is \textbf{individual} and will be assessed on a \textbf{threshold} basis.  The following criteria are used to determine a pass or fail:

\begin{enumerate}[label=(\alph*)]
	\item Enough work is available to hold a meaningful discussion;
	\item Clear evidence of programming knowledge \textbf{and} communication skills;
	\item No breaches of academic integrity.
\end{enumerate}

To complete Part D, prepare a practical demonstration of the computer programs. Ensure that the source code and related assets are pushed to GitHub and a pull request is made prior to the scheduled viva session. Then, attend the scheduled viva session.

You will receive immediate \textbf{informal feedback} from your \textbf{tutor}.

\section*{Additional Guidance}

It is critically important that you do not neglect your individual roles in the development process. Programming in pairs means that you work together on the same computer---switching between driver and navigator. It is a great opportunity to develop your technical communication skills and overcome common misconceptions about programming. It should not, however, be treated as a 'free ride'---you will get to review each others' progress. 

A common pitfall is poor planning or time management. Often, students underestimate how much work is involved in first learning programming concepts and then actually applying them. Programming is quite unlike other subjects in that it cannot be crammed into a last minute deluge just before a deadline. It is, therefore, very important that you begin work early and sustain a consistent pace: little and often.

The first deadline is quite close to the start of the course and not much material will have been covered by this point. Please rest assured. This first formative submission is supposed to be a simple analysis of requirements. We expect there to be errors. However, it is very important to make a start on this project so you recieve early feedback to give you some direction and to encourage you to practice your programming skills across the entire duration of the course. Ideally, you should be programming every day!

The peer-review component of this work does sometimes raise alarm. However, the only way to learn how to review code is by reviewing code. Your tutor will guide you through the process and provide advice. With practice, it will become clear what is satisfactory by discussing the quality of work with your peers and your tutor during the peer review sessions. 

\section*{FAQ}

\begin{itemize}
	\item 	\textbf{What is the deadline for this assignment?} \\ 
    		Falmouth University policy states that deadlines must only be specified on the MyFalmouth system.
    		
	\item 	\textbf{What should I do to seek help?} \\ 
    		You can email your tutor for informal clarifications. For informal feedback, make a pull request on GitHub. 
    		
    	\item 	\textbf{Is this a mistake?} \\ 	
    		If you have discovered an issue with the brief itself, the source files are available at: \\
    		\url{https://github.com/Falmouth-Games-Academy/bsc-assignment-briefs}.\\
    		 Please raise an issue and comment accordingly.
\end{itemize}

\section*{Additional Resources}

\begin{itemize}
    \item TBC
\end{itemize}

\rubrichead{Criteria marked with a $\ddagger$ are shared by the group. All other criteria are individual.}
\begin{markingrubric}
%
    \firstcriterion{Functional Coherence}{5\% $\ddagger$}
        \grade\fail 	No algorithm has been implemented successfully.
            \par 		The source code does not compile or there are serious logical errors.
        \grade 		At least one algorithm has been  implemented successfully.
            \par 		There are many obvious logical errors, more than one of which is significant.   
        \grade 		At least two algorithms have been  implemented successfully.
            \par 		There are several obvious logical errors, at least one of which is significant. 
        \grade 		At least three algorithms have been  implemented successfully.
            \par 		There are some obvious logical errors, which are not significant. 
            \par		The brief has been satisfied.
        \grade 		At least three algorithms have been  implemented successfully.
            \par 		There are few obvious logical errors, which are cosmetic and/or superficial.
            \par		The brief has been satisfied.     
        \grade 		At least three algorithms have been  implemented successfully.
            \par		There are no obvious logical errors.
            \par		The brief has been satisfied.
%
    \criterion{Sophistication}{15\% $\ddagger$}
        \grade\fail No insight into the appropriate use of programming constructs is evident from the source code.
            \par No attempt to structure the program (e.g. one monolithic function).
        \grade Little insight into the appropriate use of programming constructs is evident from the source code.
            \par The program structure is poor.
        \grade Some insight into the appropriate use of programming constructs is evident from the source code.
            \par The program structure is adequate.
        \grade Much insight into the appropriate use of programming constructs is evident from the source code.
            \par The program structure is appropriate.
        \grade Considerable insight into the appropriate use of programming constructs is evident from the source code.
            \par The program structure is effective. There is high cohesion and low coupling.
        \grade Significant insight into the appropriate use of programming constructs is evident from the source code.
            \par The program structure is very effective. There is high cohesion and low coupling.
%
    \criterion{Maintainability}{20\% $\ddagger$}
        \grade\fail There are no comments in the source code, or comments are misleading.
            \par Most variable names are unclear or inappropriate.
            \par Code formatting hinders readability.
        \grade The source code is only sporadically commented, or comments are unclear.
            \par Some identifier names are unclear or inappropriate.
            \par Code formatting is inconsistent or does not aid readability.
        \grade The source code is somewhat well commented.
            \par Some identifier names are descriptive and appropriate.
            \par An attempt has been made to adhere to thhe PEP-8 formatting style.
             \par There is little obvious duplication of code or of literal values.           
        \grade The source code is reasonably well commented.
            \par Most identifier names are descriptive and appropriate.
            \par Most code adheres to the PEP-8 formatting style.
             \par There is almost no obvious duplication of code or of literal values.   
        \grade The source code is reasonably well commented, with Python doc-strings.
            \par Almost all identifier names are descriptive and appropriate.
            \par Almost all code adheres to the PEP-8 formatting style.
             \par There is no obvious duplication of code or of literal values. Some literal values can be easily ``tinkered'' in the source code. 
        \grade The source code is very well commented, with Python doc-strings.
            \par All identifier names are descriptive and appropriate.
            \par All source code adheres to the PEP-8 formatting style.
             \par There is no obvious duplication of code or of literal values. Most literal values are, where appropriate, easily ``tinkered'' outside of the source code.  
%
    \criterion{Creative Flair}{10\% $\ddagger$}
        \grade\fail No creativity.
            \par The work is a clone of an existing work with mere cosmetic alterations.
        \grade Little creativity.
            \par The work is derivative of existing works, with only minor alterations.
        \grade Some creativity.
            \par The work is derivative of existing works, demonstrating little divergent and/or subversive thinking.
        \grade Much creativity.
            \par The work is somewhat novel, demonstrating some divergent and/or subversive thinking.
        \grade Considerable creativity.
            \par The work is novel, demonstrating significant divergent and/or subversive thinking.
        \grade Significant creativity.
            \par The work is highly original, with strong evidence of divergent and/or subversive thinking.
%
    \criterion{Use of Version Control}{10\%}
        \grade\fail GitHub has not been used.
        \grade Source code has rarely been checked into GitHub.
        \grade Source code  has been checked into GitHub at least once per week.
            \par Commit messages are present.
            \par There is evidence of engagement with peers (e.g.\ code review).
        \grade Source code  has been checked into GitHub several times per week.
            \par Commit messages are clear, concise and relevant.
            \par There is evidence of somewhat meaningful engagement with peers (e.g.\ code review).
        \grade Source code has been checked into GitHub several times per week.
            \par Commit messages are clear, concise and relevant.
            \par There is evidence of meaningful engagement with peers (e.g.\ code review).
        \grade Source code has been checked into GitHub several times per week.
            \par Commit messages are clear, concise and relevant.
            \par There is evidence of effective engagement with peers (e.g.\ code review).
%
    \criterion{Basic Competency Threshold}{40\%}
        \gradespan{1}{\fail At least one part is missing or is unsatisfactory.}
        \gradespan{5}{Submission is timely.
        	\par Enough work is available to hold a meaningful discussion.
	\par Clear evidence of programming knowledge and communication skills.
	\par Clear evidence of reflection on own performance and contribution.
	\par Only constructive criticism of pair-programming partner is raised.
	\par No breaches of academic integrity.}
	
\end{markingrubric}

\end{document}