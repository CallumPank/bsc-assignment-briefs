\documentclass{../fal_assignment}
\graphicspath{ {../} }

\usepackage{enumitem}
\usepackage[T1]{fontenc} % http://tex.stackexchange.com/a/17858
\usepackage{url}
\usepackage{todonotes}

\title{Interface Hacking}
\author{Alcwyn Parker}

\begin{document}

\maketitle
%\begin{marginquote}
%    ``Students come into programming classes with a broad range of backgrounds ---
%    some have experience in several programming languages, others have never programmed before in their life.
%    
%    Being able to engage with the community and support each other is important.
%    Upload your code to GitHub and receive feedback from experienced peers.
%    Review your peers' work yourself and really consider what `quality' actually means
%    and what `good' source code looks like.
%    Debate, argue, and question others about it ---
%    an open and sustained discourse is an excellent way to learn ---
%    for both beginners and adepts!''
%\end{marginquote}
\marginpicture{flavour_pic}{
    The \emph{Arduino} is an open-source prototyping platform that makes designing and developing interfaces fun \& easy.}
\section*{Introduction}

In this assignment, you are required to design and prototype a novel game controller device.
Your prototype should function as an input device for \textbf{either}
one of the games being developed by students on the BA Digital Games course; \textbf{or}
the game you developed in COMP130 last semester.
Your prototype should use a hardware platform such as \emph{Arduino} or \emph{Raspberry~Pi} etc,
to convert user actions into game inputs.

%Computing for Games combines technical and creative skills in equal parts.
%All of your assignments involve a mixture of the two;
%in this assignment the emphasis is more on creativity.
%You will build upon the technical skills you have learned so far,
%combined with your own creativity and innovation,
%to produce a unique creative artefact.

This assignment is formed of \textbf{three} parts.
The summative submission has \textbf{two} parts: an electronic component and a physical component.
Begin by forking the GitHub project at the following URL:
\begin{center}
\url{https://github.com/Falmouth-Games-Academy/comp140-hardware}
\end{center}

\subsection*{A. \emph{Design} a novel game controller device}

On GitHub, create a Markdown file (or edit \texttt{readme.md})
defining the high concept of your controller.
Include details of any background research you have done to assess commercial viability.

On Trello, create a task board that defines the key requirements
(in terms of components and user stories) of the controller.

\textbf{Formative submission:} Arrange a meeting with your tutor to discuss your concept and task board.

\subsection*{B. \emph{Build} and  \emph{integrate} a prototype of your game controller}

You will build your prototype controller over a period of \textbf{4 weeks} utilising a \textbf{fast}, \textbf{iterative} development process.
Each week should see a vast \textbf{improvement} in the quality of design and development working towards a \textbf{shippable product} to demo in the fourth week. that is, a prototype which does not have any major flaws or half-finished features that prevent it from being tested, and that can be used (even if lacking some features) as a controller in the game.

Use this forked repository to store any digital artefacts (including but not limited to
design sketches, photographs, art assets, source code, electronic circuit designs).

\subsection*{C. \emph{Heuristic analysis} by peers}

Bring your prototype controller to the session where it will be subject to heuristic analysis by your peers in class.

For submission, you must also prepare a short \textbf{video demonstration} of your controller.
The maximum length of this video is \textbf{two minutes};
longer videos will be subject to penalties in line with the course word count policy available on LearningSpace.
In your video, briefly demonstrate the functionality of the controller.
You may narrate if you wish, but it is not required.

\textbf{Formative submission:} Participate in the heuristic analysis session.

\subsection*{Summative submission (electronic)}

Create a zip file containing the following:

\begin{itemize}
\item A markdown file named \texttt{readme.md} giving a high-level overview of the project,
    including your concept and background research from part~A.
\item Screenshots of your Trello task board from part~A.
\item All source code and other digital artefacts created for part~B.
    You should include source code files or snippets demonstrating the integration of your controller into the game,
    but you do \textbf{not} need to provide the entire source code for the game.
\item The video demonstration produced for part~C, compressed in AVI or MP4 format.
\end{itemize}

The recommended way to produce this zip file is to check all of the above into your GitHub repository
throughout the course of the project,
and then use the ``Download Zip'' function on the GitHub website.

\textbf{Material not included in the zip file will not be marked},
even if it is available online, and even if tutors have seen it in class.
Please ensure that all material you want the markers to consider is included in your submission.

Upload your zip file to the appropriate submission queue on LearningSpace.

\subsection*{Summative submission (physical)}

Hand in your prototype controller to the Academic Office in Tremough House.
Pack your controller in a suitably sized box, with your student number
and ``BSc Computing for Games -- COMP140 Assignment 1'' written clearly on it.

Your prototype will be returned to you after marking,
however any components that are property of the university (e.g.\ Arduinos \& RaspberryPi boards)
will first be removed.
Thus please do not glue, solder or otherwise permanently affix anything to them.

\begin{marginquote}
    ``Classic engineering relies on a strict process for getting from A to B; the Arduino Way delights in the possibility of getting lost on the way and finding C instead.``
    
   
    --- Massimo Banzi - Co-founder of Arduino
\end{marginquote}

\section*{Additional Guidance}

Falmouth University is nationally and internationally renowned as an arts institution.
Despite the fact that you are studying for a Bachelor of Science degree in a technical discipline,
you are still expected to strive for the same level of \textbf{innovation and creative flair}
as your fellow students in other departments.
All assignments on this course involve a mix of technical and creative activities;
this assignment is more heavily weighted towards the creative than the assignments you have completed thus far.
On this assignment, a competent execution of an unimaginative idea is unlikely to achieve higher than a C grade overall,
as opposed to an imperfect execution of a unique and ambitious concept
--- bear this in mind when working on your design.
One approach to promoting creativity is
\textbf{divergent thinking}: generation of ideas by exploring many possible solutions.
Often the most interesting ideas are \textbf{subversive}: they deliberately go against the
accepted or most obvious solution

The history of video games is littered with failed peripherals which consumers simply did not want,
which were perceived as expensive gimmicks rather than legitimate enhancements to gameplay.
Your creativity should be balanced by \textbf{commercial awareness}:
your design should be informed by your research into products that have succeeded and failed
in the past, and what underexploited niches exist in the present.
An A$^*$ project would be a highly divergent idea, but one that has clear commercial viability.
Do not be too discouraged if you fall short of this: this is a tall order even for the professionals!

We have given you some of the materials you need: an Arduino and other useful components.
You will need to add your own materials to produce a \textbf{functional} physical prototype.
A ``Blue Peter'' style prototype made from household items is fine,
as is something made out of modelling clay, construction toys etc.
However you should still choose your materials carefully, as overly flimsy construction may
lose you marks on the functionality criterion.

You may also wish to connect electronic components such as LEDs, buzzers, photoresistors etc to the Arduino,
or even use a different, more flexible hardware platform such as RaspberryPi.
However you are discouraged from spending large sums of money on extra hardware,
and doing so is \textbf{not required} to achieve a high mark.
If you choose to go down this route,
it is possible to purchase an RaspberryPi and other useful peripheral online for 
around the price of a textbook (\textsterling 20 -- \textsterling 30).

You should aim to demonstrate a high level of \textbf{sophistication}
in the technical execution of your prototype.
An important part of sophistication is having the insight to choose the right tool for the job:
if a simpler technique fulfils all the requirements, use it.
The use of unnecessarily complicated techniques, serving only to showcase one's own cleverness,
is a dangerous habit for a software developer.

The sole purpose of the \textbf{video demonstration} is to aid moderators and external examiners,
who are not present for the demo session,
in assessing your work.
Your video does \textbf{not} need to be entertaining or highly polished:
a smartphone or webcam video of you or someone else using the controller is sufficient.

%Your \textbf{weekly reports} should document the iterations you make on your design and prototyping.
%The emphasis in this assignment is on creativity and rapid iteration,
%so do not be afraid to ``go back to the drawing board'' if a prototyped idea does not work as well as anticipated.
%However it is important to document (and learn from) your failures, even more so than your successes.
%
%You are strongly encouraged to make use of sketches, diagrams, photographs, screenshots
%and short videos to document your design and prototyping process.
%Many indie game developers use such work-in-progress material as an important tool for promotion
%and community engagement; for example, search Twitter for the hashtag \texttt{\#screenshotsaturday}.
%Videos should ideally be short (5--30 second) demonstrations of functions of your prototype;
%you may narrate if you wish, but it is not required.

\begin{marginquote}
    ``The first 90 percent of the code accounts for the first 90 percent of the development time.
    
    ``The remaining 10 percent of the code accounts for the other 90 percent of the development time.''
    
    --- Tom Cargill
    
    \marginquoterule
    
    ``Hofstadter's Law:
    
    ``It always takes longer than you expect, even when you take into account Hofstadter's Law.''
    
    --- Douglas Hofstadter
\end{marginquote}

\section*{Additional Resources}

\begin{itemize}
    \item Wilkinson, K. and Petrich, M. (2014) The Art of Tinkering: Meet 150 Markers Working at the Intersection of Art, Science \& Technology. Weldon Owen: London.
    \item Alicia Gibb. Building Open Source Hardware: DIY Manufacturing for Hackers and Makers. Addison Wesley, 2014. 
    \item Jeremy Blum. Exploring Arduino: Tools and Techniques for Engineering Wizardry. John Wiley, 2013. 
    \item Kelly, K. (2014) Cool Tools: A Catalogue of Possibilities. Cool Tools.
    \item Hatch, M. (2013) The Maker Movement Manifesto: Rules for Innovation in the New World of Creators, Hackers, and Tinkerers. McGraw Hill: New York.
    \item \url{http://arduino.cc}
    \item \url{https://vimeo.com/18539129} Arduino The Documentary (2010)
\end{itemize}

\begin{markingrubric}
    \firstcriterion{Iterative development process}{Threshold \par 5\% + 5\%}
        \gradespan{2}{\fail There is little or no evidence of an iterative development process and no improvement over time in regards to the quality of the design and build of the prototype. }
        \gradespan{2}{A `potentially shippable' prototype is produced at the end of development period.
            \par There is evidence of a `reasonable' iterative development process but the prototype suffers from 'lock in' in regards to the original concept.}
        \gradespan{2}{A `potentially shippable' prototype is produced at the end of the development period.
            \par The project has benefitted from an iterative development process and many improvement have been made to the original concept}
    \criterion{Design of the solution}{15\%}
        \grade\fail User stories are not provided, or the design does not correspond to the user stories.
        \grade Few user stories are distinguishable and easily measured.
            \par The correspondence between design and user stories is tenuous.
        \grade Some user stories are distinguishable and easily measured.
            \par The design somewhat corresponds to the user stories.
        \grade Most user stories are distinguishable and easily measured.
            \par The design corresponds to the user stories.
        \grade Nearly all user stories are distinguishable and easily measured.
            \par The design clearly corresponds to the user stories.
        \grade All user stories are distinguishable and easily measured.
            \par The design clearly and comprehensively corresponds to the user stories.
    \criterion{Commercial awareness}{10\%}
        \grade\fail No commercial awareness is demonstrated.
        \grade Emerging commercial awareness is demonstrated.
            \par There is no evidence of market research.
        \grade Some commercial awareness is demonstrated.
            \par Market research is present, but with significant gaps.
        \grade Much commercial awareness is demonstrated.
            \par Market research is extensive, but with some gaps.
        \grade Significant commercial awareness is demonstrated.
            \par Market research is comprehensive.
        \grade Exemplary commercial awareness is demonstrated.
            \par Market research is comprehensive and insightful.
    \criterion{Innovation and creative flair}{30\%}
        \grade\fail Demonstrates no evidence of innovation and/or creativity.
        \grade Demonstrates evidence of emerging innovation and/or creativity.
            \par The solution is purely derivative of existing products.
            \par There is no evidence of divergent thinking.
        \grade Demonstrates evidence of progressing innovation and/or creativity.
            \par The solution is mostly derivative, with some attempts at innovation.
            \par There is evidence of an attempt at divergent thinking.
        \grade Demonstrates evidence of partial mastery of innovative and creative practice.
            \par The solution is an interesting and somewhat innovative product.
            \par There is some evidence of divergent thinking.
        \grade Demonstrates some evidence of mastery of innovative and creative practice.
            \par The solution is a novel and innovative product.
            \par There is much evidence of divergent thinking.
        \grade Demonstrates much evidence of mastery of innovative and creative practice.
            \par The solution is a unique and innovative product.
            \par There is significant evidence of divergent thinking.
    \criterion{Functionality of physical prototype}{10\%}
        \grade\fail A physical prototype is not produced, or the prototype is completely non-functional.
        \grade The physical prototype is barely functional.
            \par There are serious technical and/or constructional flaws.
        \grade The physical prototype is somewhat functional.
            \par There are obvious technical and/or constructional flaws.
        \grade The physical prototype is mostly functional.
            \par There are minor technical and/or constructional flaws.
        \grade The physical prototype is functional.
            \par There are superficial technical and/or constructional flaws.
        \grade The physical prototype is functional.
            \par The technical execution and physical construction are flawless.
    \criterion{Sophistication: \par Software \par Electronics \par Physical construction}{20\%}
        \grade\fail The solution lacks even a basic level of sophistication in any of the three areas.
        \grade The solution is basic and unsophisticated in all three areas.
            \par Little insight has been demonstrated in any area.
        \grade The solution is moderately sophisticated in one of the areas, but lacking in the other two.
            \par Emerging insight has been demonstrated in at least one of the areas.
        \grade The solution is moderately sophisticated in two of the noted areas, but lacking in the third.
            \par Much insight has been demonstrated in at least one of the areas.
        \grade The solution combines somewhat sophisticated software, electronics and physical construction.
            \par Significant insight has been demonstrated in at least two of these areas.
        \grade The solution combines highly sophisticated software, electronics and physical construction.
            \par Exemplary insight has been demonstrated in all three areas.
    \criterion{Professional practice}{5\%}
        \grade\fail GitHub has not been used.
        \grade Material has only been checked into GitHub a few times before the deadline.
        \grade Material has been checked into GitHub at least once per week.
        \grade Material has been checked into GitHub several times per week.
        \grade Material has been checked into GitHub several times per week.
            \par Commit messages are clear, concise and relevant.
        \grade Material has been checked into GitHub several times per week.
            \par Commit messages are clear, concise and relevant.
            \par There is evidence of engagement with peers (e.g.\ voluntary code review).
\end{markingrubric}

\end{document}
