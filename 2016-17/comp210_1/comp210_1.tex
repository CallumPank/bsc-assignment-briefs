\documentclass{../fal_assignment}
\graphicspath{ {../} }

\usepackage{enumitem}
\usepackage[T1]{fontenc} % http://tex.stackexchange.com/a/17858
\usepackage{url}
\usepackage{todonotes}

\title{Interfaces \& Interaction}
\author{Alcwyn Parker}

\begin{document}

\maketitle

\marginpicture{flavour_pic}{
    The \emph{Arduino} is an open-source prototyping platform that makes designing and developing interfaces fun \& easy.}
\section*{Introduction}

In this assignment, you are required to choose an existing screen based game interface to evaluate with a focus on usability and user-experience. Although the choice of game interface is down to you, careful consideration must be given to select an interface that is complex and interesting enough to warrant interrogation. How you evaluate the interface is down to you, although your approach must be appropriate and you will need to provide justification in your documentation. To ensure a thorough evaluation it is advised that you implement at least two qualitative and two quantitative methods. Some suggested methods are: cognitive walkthrough, task analysis, user-story mapping, analytic tools or any other methods that you feel are relevant. 

Human-centred design (HCD) puts the end-user at the heart of the design process with a focus on usability and user-experience (UX). It relies on a fast-paced, iterative approach to the design and development process where evaluation and testing are built into every iteration. The majority of methods involve either the intended users themselves or experts in the field. This allows the designer to learn from each iteration and form goals and objectives for the next. It is vital that you familiarise yourself with the various qualitative and quantitive evaluation methods so that you can apply them to all of your future projects.  

This assignment is formed of \textbf{one} part.
The summative submission has \textbf{one} part: a single digital portfolio. 

Begin by forking the GitHub project at the following URL:
\begin{center}
\url{https://github.com/Falmouth-Games-Academy/comp210-interface-evaluation}
\end{center}


\begin{enumerate}[label=(\alph*)]
    \item \textbf{Implement} a thorough evaluation of a screen based game interface of your choice. write an evaluation document, which must:
    	\begin{enumerate}[label=\roman*.]
    		\item \textbf{justify} your choice of screen based game interface;
    		\item \textbf{list} and \textbf{justify} your choice of evaluation methods;
		\item \textbf{describe} in great detail, the findings from the evaluation task.
	\end{enumerate}
    \item \textbf{Present}, a ten minute summary of your findings that will:
    	\begin{enumerate}[label=\roman*.]
    		\item \textbf{clarify} your approach to the task;
    		\item \textbf{describe} the strengths and weaknesses of your chosen interface;
    		\item and \textbf{discuss how} strategies derived from your findings that might improve the usability and user-experience of the interface in question.
	\end{enumerate}
	\item \textbf{Write} a conclusion that will:
    	\begin{enumerate}[label=\roman*.]
    		\item \textbf{synthesise} your findings into a clear and concise list of strengths and weaknesses for your chosen interface. 
    		\item \textbf{reflect} on the process and consider the strengths and weaknesses of your approach;
	\end{enumerate}
\end{enumerate}

This assignment is an \textbf{academic writing task}. Fork the GitHub repository at the following URL:

\indent \url{https://github.com/Falmouth-Games-Academy/comp210-evaluation}

Use the existing directory structure and, as required, extend this structure with sub-directories. Ensure that you maintain the \texttt{readme.md} file.

Modify the \texttt{.gitignore} to the defaults for \textbf{TeX}. Please, also ensure that you add editor-specific files and folders to \texttt{.gitignore}. 


\subsection*{Part A}

Part A is formed of \textbf{multiple formative submissions}. This is \textbf{individual} work will be assessed on a \textbf{threshold} basis. The following criteria are used to determine a pass or fail:

\begin{enumerate}[label=(\alph*)]
	\item Submission is timely;
	\item Enough progress is made to conduct a meaningful review each week;
\end{enumerate}

To complete Part A, carry out a thorough evaluation of your chosen interface. Demonstrate your progress to your tutor each week in class. You are expected to implement at least four different methods of evaluation (two qualitative and two quantitative). Ensure that any digital artefacts (including but not limited to sketches, photographs, diagrams, raw data, and any other documentation) are pushed to GitHub prior to each weekly session. 

You will receive immediate \textbf{informal feedback} from your \textbf{tutor} and \textbf{peers}.

\subsection*{Part B}

Part B is a \textbf{single summative submission}. This is \textbf{individual} work will be assessed on a \textbf{threshold} basis. The following criteria are used to determine a pass or fail: 

\begin{enumerate}[label=(\alph*)]
	\item Enough work is available to hold a meaningful discussion; 
	\item Clear evidence of usability testing knowledge and communication skills; 
	\item No breaches of academic integrity. 
\end{enumerate}

To complete Part B, prepare a ten minute presentation that explains your approach to the task and summarises your findings. Ensure that all related assets are pushed to GitHub and a pull request is made prior to the scheduled viva session. Then, attend the scheduled viva session. 

You will receive \textbf{immediate informal} feedback from your \textbf{tutor}.

\subsection*{Part C}

Part C is a \textbf{single summative submission}. This work is \textbf{individual} and will be assessed on a \textbf{criterion-referenced} basis. Please refer to the marking rubric at the end of this document for further detail.

To complete Part C, write a conclusion that provides a synthesis  of your findings and reflects on the strengths and weakness of your approach to the task. Then, upload all the documentation to the LearningSpace. Please note, the LearningSpace will only accept a single \texttt{.zip} file.

You will receive \textbf{formal feedback} from your \textbf{tutor} three weeks after the final submission deadline.

\section*{Additional Guidance}
Choosing an appropriate game interface is critical to this task. Your choice of game interface should not only be complex and interesting enough to warrant interrogation but also be relevant to your interests and your aspirations as a game developer. The selection process might involve choosing multiple games and using rapid and heavily discounted evaluation methods to identify the game interface that will produce the most insightful results. Before you begin the task you are encourages to research existing case studies/evaluations to inform your approach. 

Your evaluation must find a balance between expert reviews vs. usability testing. As mentioned previously, human-centred design puts the user at the centre of the design process and thus, relying solely on expert review will not produce results conducive to a HCD process. The purpose of usability testing is to evaluate the users behaviour when interacting with an interface and identify the aspects of the interface that are most regularly a source of frustration and confusion. Tests should be designed around tasks and scenarios that represent typical end-user goals. Participants in your studies must span a range of skills and experiences for your results to be meaningful. It is important that you go beyond your course cohort to find participants. 

Poor planning and poor time management can have a significant impact on this assignment. A comprehensive evaluation cannot be `crammed' into a last minute deluge. Sustain a steady pace across the four weeks. At a minimum, you should aim to implement one method of evaluation a week.

Areas where students tend to lose marks are: depth of insight; analytical skill; and evaluative skill. Depth of insight implies rigorous testing of each task in great. Adequate analysis implies going beyond mere description, perhaps through: researching UI/UX, comparing interface, or even deploying reasoning to generate new insights. Adequate evaluation implies making appropriate reference to evidence and ensuring that evidence is of appropriate quality. Further to this, sound and valid arguments are constructed based on common usability principles. 

\begin{marginquote}
    ``The first 90 percent of the code accounts for the first 90 percent of the development time.
    
    ``The remaining 10 percent of the code accounts for the other 90 percent of the development time.''
    
    --- Tom Cargill
    
    \marginquoterule
    
    ``Hofstadter's Law:
    
    ``It always takes longer than you expect, even when you take into account Hofstadter's Law.''
    
    --- Douglas Hofstadter
\end{marginquote}

\section*{Additional Resources}

\begin{itemize}
    \item Wilkinson, K. and Petrich, M. (2014) The Art of Tinkering: Meet 150 Markers Working at the Intersection of Art, Science \& Technology. Weldon Owen: London.
    \item Alicia Gibb. Building Open Source Hardware: DIY Manufacturing for Hackers and Makers. Addison Wesley, 2014. 
    \item Jeremy Blum. Exploring Arduino: Tools and Techniques for Engineering Wizardry. John Wiley, 2013. 
    \item Kelly, K. (2014) Cool Tools: A Catalogue of Possibilities. Cool Tools.
    \item Hatch, M. (2013) The Maker Movement Manifesto: Rules for Innovation in the New World of Creators, Hackers, and Tinkerers. McGraw Hill: New York.
    \item \url{http://arduino.cc}
    \item \url{https://vimeo.com/18539129} Arduino The Documentary (2010)
\end{itemize}

\begin{markingrubric}
    \firstcriterion{Iterative development process}{Threshold \par 5\% + 5\%}
        \gradespan{2}{\fail There is little or no evidence of an iterative development process and no improvement over time in regards to the quality of the design and build of the prototype. }
        \gradespan{2}{A `potentially shippable' prototype is produced at the end of development period.
            \par There is evidence of a `reasonable' iterative development process but the prototype suffers from 'lock in' in regards to the original concept.}
        \gradespan{2}{A `potentially shippable' prototype is produced at the end of the development period.
            \par The project has benefitted from an iterative development process and many improvement have been made to the original concept}
    \criterion{Design of the solution}{15\%}
        \grade\fail User stories are not provided, or the design does not correspond to the user stories.
        \grade Few user stories are distinguishable and easily measured.
            \par The correspondence between design and user stories is tenuous.
        \grade Some user stories are distinguishable and easily measured.
            \par The design somewhat corresponds to the user stories.
        \grade Most user stories are distinguishable and easily measured.
            \par The design corresponds to the user stories.
        \grade Nearly all user stories are distinguishable and easily measured.
            \par The design clearly corresponds to the user stories.
        \grade All user stories are distinguishable and easily measured.
            \par The design clearly and comprehensively corresponds to the user stories.
    \criterion{Commercial awareness}{10\%}
        \grade\fail No commercial awareness is demonstrated.
        \grade Emerging commercial awareness is demonstrated.
            \par There is no evidence of market research.
        \grade Some commercial awareness is demonstrated.
            \par Market research is present, but with significant gaps.
        \grade Much commercial awareness is demonstrated.
            \par Market research is extensive, but with some gaps.
        \grade Significant commercial awareness is demonstrated.
            \par Market research is comprehensive.
        \grade Exemplary commercial awareness is demonstrated.
            \par Market research is comprehensive and insightful.
    \criterion{Innovation and creative flair}{30\%}
        \grade\fail Demonstrates no evidence of innovation and/or creativity.
        \grade Demonstrates evidence of emerging innovation and/or creativity.
            \par The solution is purely derivative of existing products.
            \par There is no evidence of divergent thinking.
        \grade Demonstrates evidence of progressing innovation and/or creativity.
            \par The solution is mostly derivative, with some attempts at innovation.
            \par There is evidence of an attempt at divergent thinking.
        \grade Demonstrates evidence of partial mastery of innovative and creative practice.
            \par The solution is an interesting and somewhat innovative product.
            \par There is some evidence of divergent thinking.
        \grade Demonstrates some evidence of mastery of innovative and creative practice.
            \par The solution is a novel and innovative product.
            \par There is much evidence of divergent thinking.
        \grade Demonstrates much evidence of mastery of innovative and creative practice.
            \par The solution is a unique and innovative product.
            \par There is significant evidence of divergent thinking.
    \criterion{Functionality of physical prototype}{10\%}
        \grade\fail A physical prototype is not produced, or the prototype is completely non-functional.
        \grade The physical prototype is barely functional.
            \par There are serious technical and/or constructional flaws.
        \grade The physical prototype is somewhat functional.
            \par There are obvious technical and/or constructional flaws.
        \grade The physical prototype is mostly functional.
            \par There are minor technical and/or constructional flaws.
        \grade The physical prototype is functional.
            \par There are superficial technical and/or constructional flaws.
        \grade The physical prototype is functional.
            \par The technical execution and physical construction are flawless.
    \criterion{Sophistication: \par Software \par Electronics \par Physical construction}{20\%}
        \grade\fail The solution lacks even a basic level of sophistication in any of the three areas.
        \grade The solution is basic and unsophisticated in all three areas.
            \par Little insight has been demonstrated in any area.
        \grade The solution is moderately sophisticated in one of the areas, but lacking in the other two.
            \par Emerging insight has been demonstrated in at least one of the areas.
        \grade The solution is moderately sophisticated in two of the noted areas, but lacking in the third.
            \par Much insight has been demonstrated in at least one of the areas.
        \grade The solution combines somewhat sophisticated software, electronics and physical construction.
            \par Significant insight has been demonstrated in at least two of these areas.
        \grade The solution combines highly sophisticated software, electronics and physical construction.
            \par Exemplary insight has been demonstrated in all three areas.
    \criterion{Professional practice}{5\%}
        \grade\fail GitHub has not been used.
        \grade Material has only been checked into GitHub a few times before the deadline.
        \grade Material has been checked into GitHub at least once per week.
        \grade Material has been checked into GitHub several times per week.
        \grade Material has been checked into GitHub several times per week.
            \par Commit messages are clear, concise and relevant.
        \grade Material has been checked into GitHub several times per week.
            \par Commit messages are clear, concise and relevant.
            \par There is evidence of engagement with peers (e.g.\ voluntary code review).
\end{markingrubric}

\end{document}
